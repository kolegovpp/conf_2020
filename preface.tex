\thispagestyle{empty}

\noindent УДК 001(571.56+571.65)(063) \\
ББК 72(2Р55)$_\rs{я}$4 \\
\indent \hspace{0.2cm} H 34

\vfill


Ответственный редактор:
чл.-корр., д.\,г.-м.\,н. \textbf{В.\,В.\,Акинин}.
\smallskip

\tolerance=500
Редакционная коллегия:
д.\,г.\,н., профессор \textbf{В.\,Н.\,Смирнов} (председатель),
к.\,б.\,н.~\textbf{О.\,П.\,Бар\-тош},
д.\,г.-м.\,н.~\textbf{А.\,С.\,Бя\-ков},
к.\,г.-м.\,н.~\textbf{И.\,С.\,Го\-лу\-бен\-ко},
к.\,г.-м.\,н.~\textbf{Е.\,Е.\,Колова},
к.\,и.\,н.~\textbf{А.\,И.\,Ле\-бе\-динцев},
д.\,б.\,н.~\textbf{В.\,П.\,Ни\-ки\-шин},
к.\,т.\,н., доцент \textbf{А.\,В.\,Сироткин},
д.\,б.\,н., доцент \textbf{А.\,А.\,Смир\-нов},
к.\,г.-м.\,н.~\textbf{И.\,М.\,Хаса\-нов},
к.\,э.\,н.~\textbf{О.\,А.\,Шарыпова}.

\bigskip
\tolerance=500
\noindent Выпуск электронной версии утвержден к печати Учёным советом СВКНИИ ДВО~РАН, протокол №~?~(???) от~??.??.2020~г.

\vfill

\begin{minipage}[t][8cm][t]{0.10\textwidth}
  \smallskip
Н 34 \hfill
\end{minipage}
\begin{minipage}[t][8cm][t]{0.85\textwidth}
  \hspace{0.6cm} \textbf{Научная молодёжь~--- Северо-Востоку России}~: Материалы
  VIII~Межрегиональной конференции молодых учёных, приуроченной к~60\nobreakdash-летнему юбилею
  Северо-Восточного комплексного научно-исследовательского института им.~Н.~А.~Шило ДВО РАН (Магадан, 26--27~ноября 2020\,г.).~---
  Магадан~: СВКНИИ ДВО РАН, 2020.~--- Вып.~8.~--- \pageref{LastPage}~с.

  \bigskip
\noindent ISBN 978-5-6040134-5-8
  \bigskip

  \small
  \hspace{0.6cm}Представлены доклады участников VIII Межрегиональной конференции молодых учёных
  <<Научная молодёжь~--- Северо-Востоку России>>, состоявшейся 26--27~ноября
  2020\,г. в СВКНИИ~ДВО~РАН.
   Отражены фундаментальные научные исследования по следующим
   направлениям: анализ и состояние объектов окружающей среды,
   проблемы рационального природопользования,
   освоение минерально-сырьевых ресурсов,
   история освоения и развития Северо-Востока России,
   особо охраняемые природные территории и экология культуры,
   медико-экологические проблемы,
   социально-экономическое и инновационное развитие северных территорий,
   биоразнообразие и состояние экосистем,
   региональная геология и геофизические методы исследований,
   физико-математические и компьютерные методы исследований.
\end{minipage}

\vfill
\vfill
Содержат тезисы докладов молодых учёных, присланные в адрес Программного комитета конференции и прошедшие отбор на соответствие объявленной тематике. В конце сборника приведены аннотации стендовых докладов IV конкурса-выставки учащихся 2--11-х классов <<Будущие начинается сегодня>>, которые прошли проверку в системе <<Антиплагиат>> и имеют уникальность текста 70\,\% и выше.
\vfill
  \bigskip
\noindent ISBN 978-5-6040134-5-8
\hfill \begin{minipage}[t][3cm][t]{0.38\textwidth}
\copyright СВКНИИ ДВО РАН, 2020
\end{minipage}
