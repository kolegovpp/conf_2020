\procTitle{Социокультурный и этноисторический контекст развития Магаданской области (на примере эвенов)}
\procAuthor{Егян~А.\,А.}
\procEmail{Artemio01091939@gmail.com}
\procOrganization{СПбГУ} \procCity{Санкт-Петербург}

\makeProcTitleRazdel
\index{e@Егян~А.\,А.}


Коренным населением Магаданской области являются эвены, народ тунгусо-маньчжурского происхождения, связанный историко-культурными и хозяйственными связями с соседними народами: якутами, коряками, юкагирами, чукчами, эвенками и так далее. Однако на территории Магаданской области эвены в количественном измерении абсолютно доминируют во всех районах над другими коренными малочисленными народами Севера (КМНС) РФ.

     Социокультурные связи эвенов неразрывным образом связаны с древностями, эпохой колонизации и новейшей историей Магаданской области, в большинстве контекстов отображая историю её освоения и развития.

     Этноисторический контекст эвенов является близкой по ряду факторов к другим народами Северо-Востока России, особенно коряков, эвенков, частично юкагирам и в меньшей степени якутам. Этническая и историческая форма восприятия эвенов сквозь призму вклада в генезис культуры Магаданской области, которая является отличительной чертой данного субъекта на фоне соседей, имеющих другой бэкграунд, позволяет раскрыть культурно-антропологический пласт бассейна Колымы.

      Контекст современной экологии и аспектов её влияния на этносоциальный контекст эвенской жизни неоднозначен. Историческая деятельность колымских разработок, особенно золотоносных, связанная также и с деятельностью Дальстроя, повлияла на физико-географические свойства локальной природы, например, на рыбохозяйственную сторону деятельности, ибо в бассейне Колымы (верхней) пришел в упадок естественный прирост и развитие рыбы, что в суммарном эквиваленте бассейного измерения превышает 150~рек [1].

      Проблемы касаются и с площадью потерянных для хозяйственного контекста (нарушенного) для оленеводства (оленьих пастбищ), характерного эвенам исторически.

      Волны проблем коснулись и группы эвен Северо-Эвенского района Магаданской области (Кубакинское, Пеледонское), которые столкнулись с золоторудной добычей и сложностей с привычным использованием в традиционной нише для оленеводства.

      Районы, находящиеся в непосредственной близости от трассы в Якутию, также испытывают влияние на традиционный быт эвенов данного региона: как Магаданской области, так и собственно Якутии (Усть-Янский улус; Булунский, Аллайховский в контексте деградации пастбищ). В то же время Момский улус, где расположены эвенские исторические поселения, считается заповедной; в ней технологические разработки запрещены и ограничены [2].

      Традиционным контекстом деятельностью эвенов, связанный с древностью Магаданской области, является оленеводство (кочевого происхождения), а также охота на белку. Эвены Охотского моря также связаны с ловлей рыбы и морского зверя. Между эвенами происходил обмен товарами натурального хозяйства (например, мясо оленей на шкуру нерпы).

      В XX веке, в эпоху коллективизации, стада оленей обобществились, крупнейшие стада включали в себя более двадцати пяти тысяч оленей в эвенских хозяйствах, которые отличались традиционно высокой продуктивностью. В эпоху 1990-хх гг., связанной с трансформацией экономического направления, появились формы общин (родовых, также семейных), которые получили определённую территорию традиционного пользования (этноисторически). Опыт реорганизации совхозов и близких к ним форм промысловых объединений показывает, что традиционное эвенское хозяйство нуждается в коллективной форме, а также в государственной поддержки [3].

      Эвены относятся к коренному населению Магаданской области и частично соседних субъектов РФ (Хабаровский край, Якутию, Чукотский АО и Камчатская области). Прежде в русском языке существовало название <<ламуты>> (от эвенкийского названия моря).

      Эвенский язык генетически восходит к тунгусо-маньчжурской ветви алтайской семьи языков. В эвенском языке существуют диалекты и говоры, более двух десятков, связанных с географией: восточный, западный и средний. Диалект эвенов Магаданской области является литературной нормой, а письменность в начале 1930-х гг. была на латинице, которая через к середине декады сменилась кириллицей. Менее половины эвенов владеют эвенским языком, что связано с русификацией и индустриализацией, религиозными и политико-правовым бэкграундом в XVIII--XX веках. Генезис части эвенов, особенно связанных с приграничными районами этноисторической зоной расселения (с юкагирами, чукчами, якутами и другими) [4].

      Проблемой для многих малочисленных народов Севера языковая тематика характерна и для эвенов, но говорить об утрате эвенского языка преждевременно. По оценке специалистов, хорошо владеют родным языком лица старшего и многие представители среднего поколений.

      Языковой вопрос эвенов является одной из проблематичных сторон, равно как и для многих народов, входящих в состав КМНС РФ. Однако полная потеря эвенского языка исключается. Владение эвенского языка, как правило, фиксируется у людей старшего и в меньшей степени среднего поколения.

      Что касается возрастной когорты молодых людей и представителей среднего возраста, на эвенском языке говорит от менее половины населения, приблизительно 1/5 всех людей, не достигших 18~лет [5].

      Преподавание эвенского языка ограничивается четвёртым классом, в части школ он является факультативным. Использование языка ограничено преимущественно семьёй (реже производством) в случае абсолютного эвенского этнически большинства. Эвенский преподаётся в Магаданском Международном Педагогическом Университете, а также в Хабаровске (Педагогический Университет) и Санкт-Петербурге (РГПУ им. Герцена). Существуют случаи проблемных контекстов (сокращения из-за проблем финансирования): Быстринский район Камчатки, где издаётся газета <<Айдит>>. Художественные направления на эвенском языке издаются Магаданским книжным издательством.

      Эвены являются одним из самых крупных народов, включённых в КМНС РФ. Сложности в реализации традиционных контекстов хозяйства по ряду исторических и других причин, а также социально-экономические трудности в Магаданской области, выраженные в демографических показателях, тем не менее, не препятствуют этносоциальному развитию в дальнейшей перспективе. Разрешение демографических, этнолингвистических, социокультурных и финансовых сторон способствует в будущем развитию социокультурного и этноисторического начала [1].

     Контекст статуса заключается в том, что в Северо-Эвенском районе Магаданском области, равно как и в Быстринском районе Камчатки, де-факто существует институт национального района, что также характерно и для других.

      Органы местного самоуправления взаимодействуют с органами власти и коммерческими структурами (например, по контестам рыбного хозяйства, трудоустройства эвенов). Ассоциация эвенов в Ольском районе распределяет среди эвенов лимиты на улов лосося. В вопросе золоторудных компаний, например, Ассоциация выступает соучредителем, что увеличивает финансовое наполнение и позволяет расходовать деньги по статьям нужд оленеводов, культурной жизни, подготовки специалистов и так далее.

      На эвенов распространяются общие контексты юриспруденции КМНС РФ. В 1998~г. в Магаданской области реализовано положение (временное) <<О~территориях традиционного природопользования малочисленных народов Севера>>, тогда как в соседних субъектах РФ были реализованы иные законы, связанные, помимо эвенов, ещё и с другими народами, преимущественно с относительным большинством из числа народов Северо-Востока:  в Якутии это <<О~территориях традиционного природопользования малочисленных народов Севера>>; в Камчатской области это закон <<О~территориально-хозяйственных общностях КМНС>> и <<О ТТП (1998)>> ; в Чукотском АО это положения (временные) <<О порядке передачи земельных участков под фермерские оленеводческие хозяйства>> [6].

      Что касается этносоциальных контекстов, традиционно выделяемыми проблемами эвенов, прежде всего социального кластера, это безработица и здоровья, особенно что связано с наследием экономических реформ и социально-экономических преобразований, выраженных в показателях безработицы и здоровья. Однако этот контекст связан также и с возвращением части эвенов на родовые места в данный период, форма возврата к натуральному хозяйству (охота, рыболовство, сбор растений) в силу финансовых трудностей.

      Что касается заболеваний, связанных со здоровьем у эвенов Магаданской области, отличаются следующие: проблемы с органами дыхания, сердцем, алкоголизмом и связанные с ним отравления, туберкулёзом [7].

      В XVII веке эвены имели географическое распространение по отрогам Верхоянского хребта, по Колыме и притокам, омолонским и индигирским бассейнам. Так, сужение привычного ареала хозяйственного и другие причины, вытекающих из колонизации со стороны Юго-Запада, привели к миграции также и в зону Уды и Амгуни.

      В Магаданской области (наиболее эвенскими) по этническому составу районами проживания являются следующие: Ольский и Северо-Эвенский. Историческими событиями для эвенов является так называемое <<восстановление национальности>>, прежде всего в Якутии и в меньшей степени в Хабаровском крае.

      Таким образом, в данной статье раскрываются социокультурный аспект формирования и развития эвенов в общем дискурсе Магаданской области на материалах этнической и социальной истории, а также хозяйственных, правовых, культурологических, социально-экономических аспектов.

     Эвены являются одним из крупнейших народов по численности, включённых в состав КМНС РФ. Однако контексты массового притока русскоязычного населения в 20 веке (различного по этническому и культурно-социальному происхождению), активное вовлечение в образование в русском среде, промышленного освоения, деятельности Дальстроя и др., сформировали особенную социокультурную и этнолингвистическую ситуацию, уникальную для Северо-Востока России.

      Значительная вовлеченность в процессы XX~века, особенно связанной с притоком населения, оторванного от прежней среды, принципиально различной по происхождению, а также связь с древними культурно-антропологическими пластами бассейна Колымы, повлияли на формирование современных контекстов в жизни эвенов Магаданской области.


\begin{thebibliography}{99}

\bibitem{}\BibAuthor{Бацаев~И.~Д.}  Очерки истории Магаданской области (начало 20-х~--- середина 60-х~гг. XX~в.).~--- Магадан~: СВКНИИ ДВО РАН, 2007.~--- 17~с.
\bibitem{}\BibAuthor{Козлов~А.~Г.} Магадан: возникновение, становление и развитие административного центра Дальстроя (1929--1945).~--- Магадан~: СВНЦ ДВО РАН, 2007.~--- 126~с.
\bibitem{}\BibAuthor{Левин~М.~Г.} Эвены // Этнографические очерки. Народы Сибири / Под. ред. М.~Г.~Левина и Л.~П.~Потапова.~--- Москва, 1956.~--- С.~26.
\bibitem{}\BibAuthor{Попова~У.~Г.} Эвены Магаданской области. Очерки истории, хозяйства и культуры эвенов Охотского побережья, 1917--1977.~--- М.~: Наука, 1981~--- 304~с.
\bibitem{}\BibAuthor{Сирина~А.~А.} Эвены // Большая российская энциклопедия.~--- Москва, 2017.~--- 204~с.
\bibitem{}\BibAuthor{Суляндзига~Р.~В., Кудряшова~Д.~А., Суляндзига~П.~В.} Коренные малочисленные народы Севера, Сибири и Дальнего Востока Российской Федерации. Обзор современного положения.~--- Москва, 2003.~--- С.~15.
\bibitem{}\BibAuthor{Туголуков~В.~А.} Тунгусы (эвенки и эвены) Средней и Западной Сибири.~--- Москва, 1985.~--- С.~26.

\end{thebibliography}
