
\begin{center}
\begin{minipage}[c]{1.0\textwidth}
 \begin{table}[H]
 \begin{center}
 \caption{\bfseries Параметры тензоров напряжений расчитанных на основе выборки из 12 зеркал скольжения замеренных в обнажении находящемся в зоне влияния Ланково-Омолонскогой зоны разломов в раойне оз.\,Мельдек}


 \label{tab:kondratev-tab}
 \medskip\footnotesize

 \begin{tabularx}{1\linewidth}{ c c c c c l}
 \toprule
 $\sigma_1$ &  $\sigma_3$ & $\mu_\sigma$ & $P$ & $\tau$ & Тип напряженного состояния \\
 \midrule

 270\dg\,\angl66  & 96\dg\,\angl23 & \hspace*{0.9em}$0,12$  & $1$  &  $0,2$ & горизонтальный сдвиг \\


 \bottomrule
 \end{tabularx}
 \end{center}

\footnotesize
Примечание:
$\sigma_1$~--- ориентация вектора растяжения (азимут и угол погружения),
$\sigma_3$~--- ориентация вектора сжатия (азимут и угол погружения),
$\mu_\sigma$~--- коэффициент Лоде~--- Надаи,
$P$~--- редуцированные значения эффективного давления,
$\tau$~--- редуцированные значения максимального касательного напряжения.


 \end{table}
\end{minipage}
\end{center}
