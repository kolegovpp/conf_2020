\procTitle{Двойная культурная идентичность как инструмент Российского освоения Арктического региона}
\procAuthor{Пиляк~С.\,А.}
\procEmail{s.pilyak@mail.ru}
\procOrganization{Музей <<Смоленская крепостная стена>>} \procCity{Смоленск}

\makeProcTitle
\index{p@Пиляк~С.\,А.}

Обращение к культурному наследию также является одним из векторов самоотождествления человека с гранями историко-культурной реальности [5, С.~6]. Сопричастность определённой эпохе, близость собственных духовных воззрений мировоззрению человека иной эпохи к настоящему моменту является модным явлением, причём в следовании этой моде невозможно обойтись без обращения к объектам культурного наследия как маркерам иных эпох. При закреплении в сознании человека образы, связанные с объектами культурного наследия, обретают ценностно-смысловую значимость и эмоциональную окраску. При этом в сознании человека данные образы неминуемо <<\textit{вступают во взаимодействие с границами его идентичности}>> [5, С.~8]. Объекты культурного наследия выступают некими зримыми символами эпох, верстовыми столбами истории. Как отмечают современные исследователи, <<\textit{в результате одни грани культуры могут позиционироваться человеком как чуждые или ,,иные‘‘, а другие восприниматься как близкие, сопричастные и в итоге тождественные его идентичности}>> [5, С.~6]. Культурная идентичность является важнейшим отражением специфики человека, эпохи, региона и государства.

Ценность культурной идентичности, культурного наследия в целом, осознаётся лишь в случае утраты. Формирование понятия было связано с процессом коренной перестройки культурно-национальной карты Европы в ходе масштабных социально-экономических процессов XIX--XX вв.: <<\textit{Поиск национальной идентичности является одной из ярких особенностей европейской культуры постнаполеоновского времени. В этом же историко-культурном контексте происходит закрепление России в роли одного из наиболее значительных игроков на политической арене континента}>> [2, С.~425]. Ощущение постепенной утраты культурных скреп народа, региона или местного сообщества привело к научному осознанию ценности идентичности. Одним из основоположников концепции идентичности является американский психолог Э.~Эриксон. Исследователь, анализируя междисциплинарную природу понятия идентичности, отмечает: <<\textit{С одной стороны, её можно отнести к сознательному ощущению личной идентичности, с другой~--- это бессознательное стремление к целостности личного характера, с третьей~--- это критерий для процесса синтеза Эго. И наконец, внутренняя солидарность с групповыми идеалами и групповой идентичностью}>> [14, С.~206]

Как отмечают исследователи, <<\textit{несмотря на единство цивилизационного типа, культуры всех регионов страны уникальны в силу географических, природно-климатических особенностей, наличных природных богатств, существующей инфраструктуры, исторического развития, демографической ситуации}>> [9, С.~40]. При этом наиболее характерными отличительными культурными особенностями регионов среди многих традиционно называют фольклор, ремесла, традиционную архитектуру, в том числе и музеефицированные комплексы под открытым небом [9, С.~43]. Такие маркеры региональной идентичности составляют наиболее очевидный набор категорий, транслирующих своеобразие региона, что в немалой степени упрощает понимание региональной специфики.

Региональная специфика, представленная в региональном культурном наследии, приобретает в условиях ускоряющейся глобализации особое значение. Как известно, <<\textit{каждый регион имеет свой культурный ландшафт, природно-климатические условия, место расположения территории, что не может не отразиться на процессах социокультурной сферы}>> [4]. При этом интерпретация культурного наследия во многом зависит от признания причастности культурному наследию определённого региона и уровня взаимопроникновения идентичности человека и культуры [5, С.~7]. Р.~Бретон [15] формулирует три качества идентичности. Во-первых, идентичность обладает потенциалом в реализации интересов людей. Во-вторых, идентичность способна увеличить уровень взаимозависимости. В-третьих, идентичность является опорой на общую историю, наследие, веру и язык.

Нередко культурную идентичность связывают с национальным менталитетом. Исследователи отмечают что <<\textit{уникальность и оригинальность культурного творчества каждого народа (этноса) обусловлены тем, как в сознании, психологии отражаются особенности его исторического развития, вся совокупность общественных и природных условий}>> [1, С.~27]. Дискуссии на тему того, какие~--- общественные или природные особенности превалируют в формировании идентичности, до сих пор не принесли однозначного решения.

Как отмечают исследователи, <<\textit{регионализация по своей внутренней сути является обособлением и отграничиванием по принципу ,,наше~--- не~наше‘‘, ,,мы~--- они‘‘, что, в свою очередь, связано с самоидентификацией, самоопределением и самоотнесением совокупности людей к определённой общности ,,региональных‘‘ материальных условий и духовных (национальных, конфессиональных и др.) ценностей}>> [3, С.~50]. В этом случае <<\textit{Чужак предпринимает попытку истолковать культурный образец (pattern) социальной группы, с которой он сближается, и сориентироваться в нем}>> [12, С.~533]. Стоит отметить, что собственное культурное наследие вполне может стать для человека наименее известным по сравнению с популярным, ярким, грамотно интерпретируемым наследием иных сообществ, в том числе и государств.

Поскольку любая общность, к которой может отнести себя человек, способна сформировать идентичность, признаки, по которым формируется эта общность, и будут свойствами идентичности. Тем не менее, <<\textit{поскольку структура идентичности имеет многоуровневый и гибкий характер, все виды идентичности взаимосвязаны между собой и имеют особый характер ,,преломления‘‘ в социокультурном пространстве, а потому говорить о каком-либо конкретном виде идентичности изолированно от других практически невозможно}>> [11, С.~654].

Процессы освоения заселённых пространств, маркировки этих территорий, к примеру, храмами, представляют не столько процесс экономического или военного характера, сколько культурную трансформацию среды. К.~Кокшенева именует церкви <<знаком веры>> [8, С.~14]. Также храмы могут служить <<знаком религии>>, то есть неким верстовым камнем, отмечающим поле присутствия приверженцев определённой религии. Весомое значение в формировании культурной идентичности имеют духовный и религиозный аспекты. <<\textit{Религиозность раскрывает глубину человеческой души в её поступках как актах нравственного самоопределения}>> [10, С.~177].

С учётом вышеизложенного в условиях России особое внимание следует уделить понятию двойной идентичности, разработанному К.~Хюбнером [13, С.~12]: <<\textit{После того, как Россия вобрала в себя Великую Степь и разнородные культуры своих южных соседей, она сформировала двойную идентичность~--- этническую и общенациональную}>> [7, С.~285-286]. Именно эта особенность российского менталитета объясняет исключительный характер российского государства и масштаб освоения различных территорий, населённых сотнями народов.

Как отмечают исследователи, <<\textit{\dotsвследствие глобализации и усиления миграционных процессов на современной карте мира практически не осталось моноэтнических государств}>> [6, С.~48]. В России при этом создание государственной многонациональной идентичности возможно и реализуется при условии сохранения национальной идентичности каждым регионом, что заметно отражается и на региональной культурной политике в целом. Особенностью интерпретации культурного наследия является высокое прикладное значение фундаментальных исследований по данной теме. Наиболее лаконичным определением наследия остаётся его понимание как субстрата национальной идентичности. Соответственно, культурное наследие становится одновременно главным инструментом и капиталом реализации государственной культурной политики.

Именно двойная культурная идентичность, выраженная в форме внимания к культуре и религии как отдельного региона, так и государства в целом, не только стала залогом освоения пространств Арктического региона, но и остаётся важнейшим инструментом государственной культурной политики и в настоящее время.

\begin{thebibliography}{99}
\bibitem{}\BibAuthor{Бабаков~В.~Г., Семенов~В.~М.} Национальное сознание н национальная культура (методологические проблемы).~--- М., 1996.~--- 70~с.
\bibitem{}\BibAuthor{Будрина~Л.~А.} <<Новый порфир>> или национальный символ? Трансформация образа малахита  // Актуальные проблемы теории и истории искусства~---2018: Тезисы докладов VIII~Международной конференции / под ред. С.~В.~Мальцевой, Е.~Ю.~Станюкович-Денисовой, А.~В.~Захаровой.~--- СПб.~: НП-Принт, 2018.~--- С.~425--426.
\bibitem{}Региональная культурная политика: методология, институты, практики : Ценностно-нормативный подход [Текст]: монография / И.~И.~Горлова, Т.~В.~Коваленко, А.~В.~Крюков и др.; отв. ред. А.~Л.~Зорин; Юж. ф-л Рос. науч.-иссл. ин-та культурного и природ. наследия им.~Д.~С.~Лихачёва.~--- М.~: Ин-т Наследия, 2019.~--- 206~с.
\bibitem{}\BibAuthor{Ефимец~М.~А., Мерзлов~Н.~Г., Шаповалов~Д.~В.} Культурные интересы устойчивого города в контексте благосостояния региона: Новоуральский городской округ // Обсерватория культуры.~--- 2019.~--- Т.~16, №~2.~--- С.~142--155.
\bibitem{}\BibAuthor{Иконникова~С.~Н., Леонов~И.~В.} Основные векторы самоотождествлений в формировании культурной идентичности // Вестник Санкт-Петербургского государственного института культуры.~--- 2019.~--- №~2~(39).~--- С.~6--10.
\bibitem{}\BibAuthor{Козловцева~Н.~А.} Региональные практики консолидации русского мира на основе русского языка // Образование и культура: потенциал взаимодействия и ресурсы НКО в социокультурном развитии регионов России. Теория и практика социокультурного развития: Сборник материалов III~Культурного форума регионов России. Москва-Волгоград-Новосибирск-Рязань-Сыктывкар, (февраль-сентябрь 2017~г.) / Составители и общая редакция: О.~Н.~Астафьева и О.~В.~Коротеева.~--- М.~: ИП~Лядов~К.~В., 2017.~--- Вып.~2.~--- С.~48--55.
\bibitem{}\BibAuthor{Кольев~А.~Н.} Нация и государство. Теория консервативной реконструкции.~--- М.~: Логос, 2005.~--- 800~с.
\bibitem{}Концепт <<русская культура>> и современные практики культурного наследования / Капитолина Антоновна Кокшенева.~--- М.~: Институт Наследия, 2019.~--- 472~с.
\bibitem{}\BibAuthor{Ройфе~А.~Б.} Культура регионов России в контексте культурологии М.~С.~Кагана // Образование и культура: потенциал взаимодействия и ресурсы НКО в социокультурном развитии регионов России. Теория и практика социокультурного развития: Сборник материалов III~Культурного форума регионов России. Москва-Волгоград-Новосибирск-Рязань-Сыктывкар, (февраль-сентябрь 2017~г.). / Составители и общая редакция: О.~Н.~Астафьева и О.~В.~Коротеева.~--- М.~: ИП~Лядов~К.~В., 2017.~--- Вып.~2.~--- С.~38--43.
\bibitem{}\BibAuthor{Сенюшкина~Т.~А.} Кризис идентичности и цивилизационное наследие Православной культуры // Россия как государство-цивилизация: высшие цели и альтернативы развития: Коллективная монография по материалам Юбилейных международных Панаринских чтений, посвященных 75-летию со дня рождения А.~С.~Панарина / Отв. ред.: В.~Н.~Расторгуев; науч. ред.: А.~В.~Никандров / Рос. науч.-исслед. ин-т культурного и природ. наследия им.~Д.~С.~Лихачёва (Ин-т Наследия); Мос. гос. ун-т им.~М.~В.~Ломоносова, Филос. ф-т.~--- М.~: Институт Наследия. 2016.~--- С.~173--182.
\bibitem{}\BibAuthor{Швецова~А.~В.} Национальная идентичность как социокультурный феномен // Обсерватория культуры. 2017.~--- Т.~14, №~6.~--- С.~653--661.
\bibitem{}\BibAuthor{Шюц~А.} Чужак: социально-психологический очерк // Шюц~А. Избранное: Мир, светящийся смыслом / Пер. с нем. и англ. В.~Г.~Николаев, С.~В.~Ромашко, Н.~М.~Смирнова.~--- М.~: РОСПЭН, 2004.~--- С.~533--549.
\bibitem{}\BibAuthor{Хюбнер~К.} Нация.~--- М.~: Канон, 2001.~--- 396~с.
\bibitem{}\BibAuthor{Эриксон~Э.~Г.} Детство и общество.~--- М.~: ИТД <<Летний сад>>, 2000.~--- 416~с.
\bibitem{}\BibAuthor{Breton~R.} Ethnic Relations in Canada: Institutional Dynamics.~--- Quabec:~McGill-Queen's University Press, 2005.~--- 424~p.

\end{thebibliography}
