\procTitle{Стоянка эпохи верхнего палеолита в Волго-Окском междуречье}
\procAuthor{Исмагилова~Е.\,Э.}
\procEmail{yekaterina.ismagilova.97@bk.ru}
\procOrganization{ЕИ КФУ} \procCity{Елабуга}

\makeProcTitle
\index{i@Исмагилова~Е.\,Э.}

% УНИКАЛЬНОСТЬ ТЕКСТА ~45%, много прямых заимствований. см. скриншот.
% Реферативная работа, отстутствует научная составляющая автора.

Палеолит Волго-Окского междуречья является крaйне интересной, но недостаточно исследованной темой. Памятники верхнего палеолита располагаются в различных геолого-геоморфологических условиях: в долинах рек и на междуречьях, в равнинных и горных местностях. Многие из них содержат культурные слои с остатками жилых строений, с многочисленными предметами из кремня, костей млекопитающих и т.\,п [5 С.\,56].

В эпоху палеолита климат Земли, её флора и фауна достаточно сильно отличались от современных климатических условий. Данный этап, межледниковье, характеризовался довольно тёплым климатом, периодически аналогичным современному, отсутствием ледникового покрова в рамках всей Русской равнины. Благоприятные условия выделяются в средневалдайском межледниковье этот этап насчитывает три периода разделённых на более холодные фазы. С 30-го по 22-е тысячелетие, это заключительный период, именно для него был характерен мягкий климат, и он же является самым продолжительным. Начало позднего валдая или осташковское время, 24--20 тыс.\,л.\,н., характеризовалось плавным похолоданием это происходило из-за наступления ледника, который, около 20--18 тыс.\,л.\,н., смог достичь максимального распространения. Этот период является самым холодным в течение всего вюрма. Окончание вюрма, позднеледниковье, 15--13, 5--12 тыс.\,л.\,н.,~--- время, когда климат становится мягче из-за отступления ледника, происходившего не планомерно, а своеобразными пульсациями, то есть период потепления сменялся с периодом похолодания [5 С.\,56]. По мере того как менялся климат, приближался или же наоборот отступал ледник, можно сделать вывод, что древние люди вели полукочевой образ жизни.

В зависимости от изменения климатических условий, группа животных в разных регионах периодически менялась. В период последнего оледенения (20--10 тыс.\,л.\,н.), на территории России оно достигло до верхнего течения Оки и среднего течения Волги, данное оледенение называется окским. В период Окского оледенения некоторые теплолюбивые животные стали вымирать или уходить в более тёплые места, а на смену им, ближе к леднику, пришли такие животные как мускусный бык и северный олень. После на территории Европейской России льды днепровского оледенения, которые разделились на две части, стали доходить до района Днепровских порогов и возможно до района современного Волго-Донского канала. Из-за холодного климата на данную территорию пришли такие животные как: мамонты, шерстистые носороги, дикие лошади, бизоны и т.\,д. Это может быть связано с самым большим похолоданием того периода и обусловленным этим обширным продвижением окололедниковых рельефов [7]. Основная причина исчезновения и снижения популяции различных видов животных заключается в значительном изменении климата и ландшафтов, а не промысловая деятельность человека.

В валдайский этап сплошной массив льда был несравненно меньше по размеру. К нему вплотную примыкала зона своеобразной приледниковой растительности, состоявшей из горно-тундровых, лесных и степных видов. Южнее находилась зона лесостепи, а за ней~--- обширные степные пространства. Это было время наиболее широкого распространения <<мамонтовой фауны>>~--- мамонта, шерстистого носорога, северного оленя, песца, обского лемминга, сайги, байбака [3 С.\,9].

Не только животные и растения <<подстраивались>> под суровые климатические условия, также и древний человек приспосабливался к ним. Они сооружали свои жилища и землянки у водоёмов, в основном у рек, и как только глобальные климатические условия менялись, человек уходил на другое место. Из-за обмельчания, разлива или же перемены прежнего направления реки, эти <<дома>> могли быть уничтожены. В свою очередь не только природные стихии устраняли данную <<преграду>> ещё и животные эмигрируя в другую местность, возможно, уничтожили многие поселения того времени, это и могло являться причиной того, что данные стоянки не все <<дошли>> до нашего времени.

За время раскопок учёные нашли огромное количество находок и отдельных предметов. На обнаруженных стоянках археологи находят изделия из мамонтовой кости, погребения древних людей, много орудий труда, предметов быта, ремёсел, украшений, оружия того времени. Найдены жезлы, дротики и копья из бивней мамонта, кремниевые наконечники, диски из мамонтовой кости с прорезями. Найдены украшения на верхней и нижней одежде, браслеты (под коленями и выше стопы), а также цельные кольца на пальцах. Найдено ожерелье, при изготовлении которого поверхность бусин обработана так, чтобы соседние бусины располагались перпендикулярно друг другу [4 С.\,154; 6].

Редкими произведениями первобытного искусства являются найденные фигурки животных~--- мамонта, лошади-сайги. Так же Зарайский бизон был найден в 2001 г. на дне ямы, примыкающей к очагам жилищно-хозяйственного комплекса стоянки. Его возраст~--- 22--23 тыс.\,л. [1]. И это самое настоящее произведение древнего искусства, работа выполнена на высоком уровне, можно рассмотреть мельчайшие детали, например, глаза, рот, челка и т.\,д. Данная фигурка была сделана из бивня мамонта и использовалась «магических» целях, например, охотники проводили ритуал перед охотой. Примечательно то, что у фигуры бизона отсутствовали два копыта, в шею был помещен кусочек кости, возможно, это могло означать копье или другое оружие того времени, так же с одной стороны он был выкрашен охрой, то есть краской, что могло означать кровь.

Существуют небольшие захоронения мамонтов по всей стране, но самое знаменитое кладбище шерстистых млекопитающих находится на территории Сибири в месте под названием Волчья Грива. При раскопках в 1957 г. впервые были найдены кости мамонта, бизона и лошади, а дальнейшие исследования позволили установить, что 14--11 тыс.\,л.\,н. в этой местности обитали последние сибирские мамонты. В те времена Волчья Грива представляла собой длинный и узкий полуостров среди болот и озер, который заканчивался крутым обрывом. В наши дни~--- это палеонтологический памятник природы областного значения. Уже обнаружено более 600 останков мамонтов. Такой высокой численности остатков доисторических животных ни на Волчьей Гриве, ни в каком-нибудь другом мамонтовом местонахождении России не встречается. Помимо мамонтов, было найдено несколько костей, принадлежавших бизону, лошади, хищникам (вероятно, лисе или песцу) и грызунам.

На данный момент, на территории Волго-Окского междуречья, известны лишь некоторые поселения эпохи Верхнего Палеолита: Сунгирь, Зарайск и т.\,д.

1) Стоянка <<Сунгирь>> открыта в 1955 г. и именно с этого времени начались ее планомерные раскопки. Они велись комплексной экспедицией, возглавляющейся О.\,Н.\,Бадером [4 С.\,201]. Благодаря тому, что смогли найти учёные экспедиции во время раскопок, свыше 50 тыс. отдельных предметов, позволили восстановить жизнь древнего человека приблизительно с исчерпывающей полнотой. Сегодня учёные располагают данными о том, что это многослойный археологический памятник, отражающий, по крайней мере, восемь тысячелетий (от 20 до 28 тыс.\,л.\,н.), в течение которых на Сунгире останавливались первобытные охотники. Это одно из самых северных верхнепалеолитических поселений на Русской равнине. Возраст стоянки около 29--25 тыс.\,л.

2) Комплекс археологических памятников~--- группа стоянок эпохи верхнего палеолита, расположенных в историческом центре города Зарайск Московской области, на высоких мысах правого берега реки Осётр. На данный момент выделяются четыре отдельных памятника, которые были обозначены как Зарайск A, B, C и D. Из них основной и наиболее хорошо изученный~--- многослойный Зарайск\,А. Весь комплекс стоянок датируется временем 23--16 тыс.\,л. Среди находок Зарайской стоянки~--- фигурки женщин~--- Венер и бизона из бивня мамонта, костяные изделия с врезным орнаментом, ожерелье из зубов песца, подвески из зубов волка и песца и др. В основе хозяйства~--- охота, 90\,\% костных остатков принадлежат мамонту, на втором месте песец, есть росомаха, волк, бизон, северный олень, заяц, серебристая чайка, тетерев, серый гусь [2 С.\,689]. Исследование данной стоянки даёт представление о приледниковой экосистеме современной Московской области.

Палеолитические стоянки Волго-Окского междуречья способствовали развитию отечественной археологии. Благодаря находкам на территориях этих памятников учёные смогли приоткрыть небольшую завесу жизни первобытных людей. Но несмотря на многочисленные раскопки и находки, эти стоянки представляются собой лишь небольшой крупицей чего-то удивительного, скрытого от нас и дожидаясь своего часа.

\begin{thebibliography}{99}
%1
\bibitem{}\BibAuthor{Амирханов\,Х.\,А., Лев\,С.\,Ю.} Статуэтка бизона с Зарайской стоянки: археологический и знаково-символический аспекты изучения // Российская археология. 2003.~--- №\,1~--- С.\,14--28.
%[2]
\bibitem{}\BibAuthor{Амирханов\,Х.\,А., Ахметгалеева\,Н.\,Б., Бужилова\,А.\,П., Бурова\,Н.\,Д., Лев\,С.\,Ю., Мащенко\,Е.\,Н.} Исследования палеолита в Зарайске 1999--2005 / под ред. Х.\,А.\,Амирханова.~--- М., 2009.~--- 758\,с.
%[3]
\bibitem{}\BibAuthor{Аникович\,М.\,В.} Становление верхнего палеолита Евразии: единство или многообразие путей? // Актуальные вопросы Евразийского палеолитоведения.~--- Новосибирск\,: Изд-во ИАЭТ СО РАН. 2005.~--- 60 с.
%[4]
\bibitem{}\BibAuthor{Бадер\,О.\,Н.} Сунгирь: Верхнепалеолитическая стоянка.~--- М.\,: Наука, 1978.~--- 271 с.
%[5]
\bibitem{}\BibAuthor{Борисковский\,П.\,И.} Древнейшее прошлое человечества~--- М., 1957.~--- 240 с.
%[6]
\bibitem{}\BibAuthor{Дебец\,Г.\,Ф.} Палеоантропологические находки в Костенках // Советская этнография.~--- 1955.~--- № 1.~--- С.\,43--53.
%[7]
\bibitem{}\BibAuthor{Деревянко\,А.\,П., Куделин\,А.\,Б., Тишков\,В.\,А.}  Адаптация народов и культур к изменениям природной среды, социальным и техногенным трансформациям.~--- М.\,: РОССПЭН, 2010.~--- 544 с.

\end{thebibliography}
\thispagestyle{empty}
