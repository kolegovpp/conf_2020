
\procTitle{Особенности сезонного освоения запасов донных рыб северной части Охотского моря в~2019~г.}
\procTitleNewLine{Особенности сезонного освоения запасов донных рыб\\северной части Охотского моря в~2019~г.}

\procAuthorI{Елатинцева~Ю.\,А., Семёнов~Ю.\,К.}
\procEmailI{elatinceva.1992@mail.ru, sapmagniro@mail.ru}
\procOrganizationI{МагаданНИРО}
\procCityI{Магадан}

\procAuthorII{Смирнов~А.\,А.}
\procEmailII{andrsmir@mail.ru}
\procOrganizationII{ВНИРО}
\procCityII{Москва}
\procOrganizationIV{СВГУ}
\procCityIV{Магадан}

\makeProcTitleIVNewLine
\index{e@Елатинцева~Ю.\,А.}
\index{s@Семёнов~Ю.\,К.}
\index{s@Смирнов~А.\,А.}



В последние годы в~Северо-Охотоморской подзоне (СОМ) Охотского моря промысел донных рыб, который в~значительной степени ведётся ярусами и~донными сетями, активно развивается [3, 6].

В ходе многовидового промысла вылавливаются такие основные объекты, как чёрный и~белокорый палтусы (\textit{Reinhardtius hippoglossoides} и~\textit{Hip\-po\-glos\-sus stenolepis}), треска (\textit{Gadus macrocephalus}), макрурусы (промыслом осваивается, в~основном, малоглазый макрурус (\textit{Albatrossia pectoralis})) [5], в~меньшей степени~--- скаты, у которых в~уловах (от 70 до 95\,\%) доминирует щитоносный скат (\textit{Bathyraja parmifera}) [2], длиннопёрый шипощёк (\textit{Sebastolobus macrochir}). Запасы ликодов, среди которых в~Охотском море преобладает ликод Солдатова (\textit{Lycodes soldatovi}), существенны, но специализированный промысел не~ведётся. Этот вид осваивается как прилов при ярусном промысле чёрного палтуса и~трески, но при этом часто выбрасывается, а не идёт в~продукцию [1].

По информации Магаданского филиала ФГБНУ <<ВНИРО>> (<<МагаданНИРО>>), подготовленной на~основе судовых суточных донесений (ССД), а также данных, поступивших от~научных наблюдателей, находившихся на~промысловых судах в~Охотском море, далее мы проанализировали освоение запасов донных рыб в~СОМ в~2019~г. По месяцам вылов распределялся неравномерно.

В январе работало до 7 судов-ярусоловов, которые затратили на~промысел 76 судосуток и~за~месяц выловили 5,4\,\% годового объёма вылова всех видов промысловых донных рыб. Скатов было выловлено 14,6\,\% от~их годового объёма вылова, белокорого палтуса~--- 99,7\,\%, трески~--- 53,8\,\%, чёрного палтуса~--- 1,1\,\%, длиннопёрого шипощёка~--- 2,3\,\%, ликодов~--- 35\,\%.

В феврале работали до 4 судов-ярусоловов и~1 сетелов. Они затратили на~промысел 58 судосуток и~за~месяц выловили 1,7\,\% годового объёма вылова всех видов. Чёрного палтуса было выловлено 2,7\,\% от~его годового объёма вылова, скатов~--- 1,1\,\%, макрурусов~--- 0,4\,\%, белокорого палтуса~--- 0,1\,\%, трески~--- 5,1\,\%, длиннопёрого шипощёка~--- 2,8\,\%.

В марте работали до 5 судов-ярусоловов и~до 4 сетеловов. Они затратили на~промысел 158 судосуток и~за~месяц выловили 5,2\,\% годового объёма вылова всех видов. Чёрного палтуса было выловлено 9,4\,\% от~его годового объёма вылова, скатов~--- 3,4\,\%, длиннопёрого шипощёка~--- 2,3\,\%, макрурусов~--- 0,7\,\%.

В апреле работали до 9 судов-ярусоловов и~до 6 сетеловов. Они затратили на~промысел 266 судосуток и~за~месяц выловили 8,5\,\% годового объёма вылова всех видов Чёрного палтуса было выловлено 13,5\,\% от~его годового объёма вылова, ликодов~--- 62,5\,\%, скатов~--- 11,5\,\%, макрурусов~--- 1,2\,\%, длиннопёрого шипощёка~--- 8\,\%.

В мае работали до 7 судов-ярусоловов и~до 6 сетеловов. Они затратили на~промысел 297~судосуток и~за~месяц выловили 11,5\,\% годового объёма вылова всех видов. Чёрного палтуса было выловлено 11\,\% от~его годового объёма вылова, ликодов~--- 2,5\,\%, скатов~--- 29,6\,\%, макрурусов~--- 6\,\%, длиннопёрого шипощёка~--- 18,8\,\%.

В июне работали до 9 судов-ярусоловов и~до 6 сетеловов. Они затратили на~промысел 283~судосуток и~за~месяц выловили 15,7\,\% годового объёма вылова всех видов. Чёрного палтуса было выловлено 19,3\,\% от~его годового объёма вылова, скатов~--- 22,9\,\%, макрурусов~--- 9,3\,\%, длиннопёрого шипощёка~--- 17,6\,\%.

В июле работали до 12 судов-ярусоловов и~до 7 сетеловов. Они затратили на~промысел 388 судосуток и~за~месяц выловили 19,8\,\% годового объёма вылова всех видов. Чёрного палтуса было выловлено 18,9\,\% от~его годового объёма вылова, скатов~--- 7,8\,\%, макрурусов~--- 27,9\,\%, трески~--- 4,6\,\%, длиннопёрого шипощёка~--- 14,8\,\%.

В августе работали до 12 судов-ярусоловов и~до 6 сетеловов. Они затратили на~промысел 367 судосуток и~за~месяц выловили 19,2\,\% годового объёма вылова всех видов. Чёрного палтуса было выловлено 11,8\,\% от~его годового объёма вылова, скатов~--- 3,5\,\%, макрурусов~--- 37,8\,\%, трески~--- 5,5\,\%, длиннопёрого шипощёка~--- 17\,\%.

В сентябре работали до 7 судов-ярусоловов и~до 5 сетеловов. Они затратили на~промысел 231\,судосуток и~за~месяц выловили 7,8\,\% годового объёма вылова всех видов. Чёрного палтуса было выловлено 9,9\,\% от~его годового объёма вылова, скатов~--- 4,3\,\%, макрурусов~--- 6,4\,\%, трески~--- 12,9\,\%, длиннопёрого шипощёка~--- 10,8\,\%, белокорого палтуса~--- 0,2\,\%.

Далее, в октябре\,---\,декабре, работу флота осложняли штормы, поэтому количество судосуток на промысле и вылов по месяцам значительно уменьшились.

В октябре работали до 6 судов-ярусоловов и, эпизодически, 1 сетелов. Они затратили на промысел 100 судосуток и~за~месяц выловили 1,6\,\% годового объема вылова всех видов. Чёрного палтуса было выловлено 2,3\,\% от~его годового объема вылова, скатов~--- 1,3\,\%, трески~--- 15,7\,\%, длиннопёрого шипощёка~--- 5,7\,\%.

В ноябре работали до 3 судов-ярусоловов. Они затратили на~промысел 32 судосуток и~за~месяц выловили 3\,\% годового объёма вылова всех видов. Чёрного палтуса было выловлено 0,1\,\% от~его годового объёма вылова, макрурусов~--- 8,4\,\%, скатов~--- 0,1\,\%, трески~--- 2,4\,\%.

В декабре эпизодически работали 1--2 судна-ярусолова. Они затратили на~промысел 6~судосуток и~за~месяц выловили 0,7\,\% годового объёма вылова всех видов. Весь улов состоял из макрурусов, которых было выловлено 1,9\,\% от~их годового объёма вылова.

Подводя итоги, можно отметить следующее. Всего в~2019~г. в~промысле донных рыб принимали участие до 12 судов-ярусоловов (их минимальное количество было в~декабре, максимальное~--- в~июле-августе) и~до 7 сетеловов (минимум~--- в~феврале и~октябре, когда работало 1 судно, максимум~--- в~июле). В январе и~ноябре-декабре сетеловы в~промысле участия не принимали). Флот приступил к работе уже в~январе, несмотря на~сложные условия (штормы, ледовые поля в~районе работ). С января по сентябрь число судов увеличивалось, в~июле их было максимальное количество (до 19), после чего наблюдалось снижение их количества на~промысле.

Максимум месячного вылова всех видов промысловых донных рыб был достигнут в~июле, минимум~--- в~декабре.

Вылов по месяцам различных объектов был неравномерным. Максимальное количество чёрного палтуса было выловлено в~июне, минимальное~--- в~ноябре. Белокорого палтуса фактически ловили только в~январе, когда было освоено 99,7\,\% его годового вылова. Вылов макрурусов был наибольшим в~августе и~наименьшим~--- в~феврале. Освоение скатов было максимальным в~мае и~минимальным~--- в~феврале. Максимум вылова трески отмечен в~январе, минимум~--- в~ноябре. Длиннопёрый шипощёк лучше всего ловился в~мае, хуже~--- в~январе. Максимальный прилов ликодов наблюдался в~апреле, минимальный~--- в~мае.
\clearpage
В целом за год освоение по видам от~рекомендуемых объёмов составило,~\%: чёрный палтус~--- 62,3, белокорый палтус~--- 283, макрурусы~--- 83,2, скаты~--- 59,1, длиннопёрый шипощёк~--- 11,7, ликоды~--- 0,4.

Необходимо отметить, что с~2018~г. в~Охотском море чёрный и~белокорый палтусы были объединены в~одну группу <<палтусы>> и~их промысел начал проходить в~составе суммарного объёма (Приказ Минсельхоза~РФ №~405 от~14~августа 2017~г). В связи с~этим вылов белокорого палтуса в~СОМ за~счёт ведения его специализированного промысла, который в~ССД указывали некоторые пользователи, в~2018--2019~гг. существенно вырос. Поэтому в~2019~г. освоение белокорого палтуса и~составило 283\,\% от~рекомендуемого [3].

Таким образом, в~2019~г. наиболее полно, хотя и~не полностью, были освоены рекомендуемые объёмы макрурусов. В~следующие годы вполне возможно увеличить долю освоения чёрного палтуса, трески, скатов, ликодов, длиннопёрого шипощёка. Для этого необходимо при промысле чёрного палтуса и~трески обязательно выделять квоты и~на вылов объектов прилова: скатов, ликодов, длиннопёрого шипощёка, из которых вполне возможно выпускать продукцию, которая пользуется спросом [4].

\begin{thebibliography}{99}

\bibitem{}
\BibAuthor{Бадаев~О.~З.} Приловы и выбросы на ярусном промысле рыб Дальневосточного рыбохозяйственного бассейна // Вопросы рыболовства.~--- 2018.~--- Т.~19, №~1.~--- С.~58--72.
\bibitem{}
\BibAuthor{Прикоки~О.~В.} Промысел, биология и перспективы промыслового использования массовых видов скатов в северной части Охотского моря // Рыбное хозяйство.~--- 2015.~--- №~4.~--- С.~75--80.
\bibitem{}
\BibAuthor{Семёнов~Ю.~К., Смирнов~А.~А., Елатинцева~Ю.~А., Ткаченко~А.~А.} Особенности промысла донных рыб в 2019~г. в северной части Охотского моря // Там же.~--- 2020.~--- №~2.~--- С.~43--50.
\bibitem{}
\BibAuthor{Смирнов~А.~А., Семёнов~Ю.~К.} Перспективы развития многовидового промысла донных рыб в Охотском море // Материалы Всероссийской научной конференции <<Устойчивое использование биологических ресурсов морей России: проблемы и перспективы>>.~--- Сочи, 2012.~--- С.~36--37.
\bibitem{}
\BibAuthor{Тупоногов~В.~И., Новиков~Н.~П.} Макрурусы~--- важный резерв глубоководного промысла в~дальневосточных морях // Рыбное хозяйство.~--- 2016.~--- №~6.~--- С.~54--60.
\bibitem{}
\BibAuthor{Юсупов~Р.~Р. и др.} Структура годового улова, состояние и перспективы освоения запасов донных рыб в Северо-Охотоморском промысловом районе и зал.~Шелихова Охотского моря~/ Р.~Р.~Юсупов, Ю.~К.~Семёнов, Л.~П.~Николенко, А.~И.~Каика, М.~В.~Ракитина, А.~С.~Сергеев, А.~Ю.~Немченко, Ю.~В.~Сидяков // Материалы докл. II~междунар. науч.-техн. конф. <<Актуальные проблемы освоения биологических ресурсов Мирового океана>>.~--- Владивосток~: Дальрыбвтуз, 2012.~--- Ч.~1~--- С.~369--374.
\end{thebibliography}
