\procTitle{Рудные минералы Au-Ag месторождения Ирбычан: влияние фундамента на состав эпитермальной минерализации}

\procAuthor{Прийменко~В.\,В., Фомина~М.\,И., Глухов~А.\,Н., Михалицына~Т.\,И.}
\procEmail{priymenkovladimir@gmail.com, mif-74@yandex.ru, gluhov76@list.ru, tim\_66@mail.ru}
\procOrganization{СВКНИИ ДВО РАН} \procCity{Магадан}


\index{p@Прийменко~В.\,В.}
\index{g@Глухов~А.\,Н.}
\index{f@Фомина~М.\,И.}
\index{m@Михалицына~Т.\,И.}

\makeProcTitle

\textbf{Введение}

Месторождение Ирбычан расположено в~северной части Эвенского рудного района Охотско–Чукотского вулканогенного пояса (ОЧВП). Месторождение открыто в~1975~г., но слабо охарактеризовано в~литературе. В~частности, остаются неясными спектр рудных минералов и~факторы рудоконтроля.

\textbf{Геолого-структурная позиция}

Месторождение приурочено к Хивгичанской вулканической просадке, осложняющей Ирбычанскую вулкано-тектоническую депрессию (ВТД) в~зо\-не влияния Доктомычанского глубинного разлома. В~его пределах выделены три рудные зоны: Восточная, Центральная и~Северная. Рудовмещающие породы~--- игнимбриты риолитов и~дацитов, прорванные дайками андезитов, дацитов, интерсивно аргиллизированные.

\textbf{Рудная минералогия}

Определение минералов выполнено на сканирующем электронном микроскопе Jeol JSM-6510LA c энергодисперсионным спектрометром (С.-Пб., 2019. Аналитик О.~Л.~Галанкина) и на микроанализаторе Camebax (Магадан, 2020. Аналитик Е.~М.~Горячева).

Минеральный состав руд не отличается большим разнообразием. Золото-серебряная минерализация представлена электрумом, кюстелитом, акантитом, фрейбергитом, тетраэдритом, науманнитом.

В рудах установлен пирит двух генераций. Пирит~I образует ксеноморфные включения в~кварце, реже массивные агрегаты в~ассоциации с арсенопиритом; срастается с блеклой рудой, халькопиритом, сфалеритом и~акантитом; в~пиите~I наблюдаются редкие каплевидные включения пирротина. По периферии пирит~I частично или полностью замещен марказитом. Состав пирита~I стехиометричный. Пирит~II представляет собой скопления мелкокристаллических идиоморфных разностей в~кварце, формирующих, как единичные включения, так и~сфероидные формы. В~призальбандовых частях кварцевых прожилков он выполняет просечки мощностью до 1~мм. В~цементе кварцевых брекчий пирит~II приурочен к периферии обломков пород, где развивается по микротрещинам и~равномерно распределен в~них. Характерная примесь As до 1~\%. Марказит полиморфно развивается по пириту обеих генераций и~заполняет пустоты в~нем. Арсенопирит представлен идиоморфными короткопризматическими кристаллами. По соотношению As/S выделяются две его разновидности: арсенопирит~I (0,37--0,53), ассоциирующий с пиритом, и~арсенопирит~II (1,97), образующий срастания со сфалеритом. Халькопирит встречается в~виде редкой эмульсии в~сфалерите, и~в~срастании с сульфидами (пиритом, арсенопиритом, марказитом и~сфалеритом) и~блеклой рудой. Сфалерит образует срастания с пиритом~I, блеклой рудой и~халькопиритом; содержит примеси Fe (1,93--5,22~\%), Cd (0,05--0,08~\%) и~Mn (0,39~\%). Галенит встречается в~срастании с акантитом, в~нем установлена примесь Se (1,17--1,45~\%). Блеклая руда отлагается в~виде ксеноморфных обособлений, цементирующих пирит-арсенопирит-марказитовый агрегат. Срастается с халькопиритом, пиритом~I и~акантитом. По химическому составу различают фрейбергит, Fe-фрейбергит и Fe-Zn тетраэдрит. Науманнит образует тонкие минеральные смеси с акантитом. Акантит встречается в~свободном состоянии в~кварце и~в~срастании с сульфидами (пиритом, арсенопиритом, халькопиритом, сфалеритом) и~блеклыми рудами. Содержит мелкую вкрапленность электрума и~кюстелита. По данным микрозондового анализа акантит подразделяется на стехиометричный и~селенсодержащий (микропримесь Se от 0,87 до 11,51~\%). Кроме того, в~рудах установлены Ag-содержащие минеральные смеси акантита, науманнита и~блеклых руд. Электрум (самородное золото) образует срастания с акантитом и~включения в~нем. Размер его выделений от 1,5 до 7 мкм. Пробность 402--426~\permil. Кюстелит (самородное серебро) представлено небольшими округлыми включениями в~акантите размером до 15~мкм.

Выделены три минеральных парагенезиса:
\begin{enumerate}[noitemsep]\vspace{-8pt}
  \item дорудный (пирит~I~+ арсенопирит~I);
  \item рудный (блеклая руда~+ халькопирит~+ сфалерит~+ пирротин~+\\галенит~+ акантит~+ электрум~+ кюстелит);
  \item пострудный (пирит~II~+ марказит~+ арсенопирит~II).
\end{enumerate}
 \vspace{-8pt}Нашими данными не подтверждено существование золото-аргентитового парагенезиса, ранее выделенного Р.~Г.~Кравцовой [3].

 Нестехиометричность составов серебросодержащих минеральных смесей дает
 возможность предполагать одностадийность рудного процесса в резко
 градиентных условиях [6].

\textbf{Сопоставление c другими месторождениями Эвенского рудного района}

Цоколем Ирбычанской ВТД является консолидированная кора Омолонского кратонного террейна [1, 2, 4, 7]~--- в~отличие от Туромчинской ВТД, вмещающей месторождения Сопка Кварцевая, Дальнее, Невенрекан и~заложенной на образованиях Гижигинской складчатой зоны [1, 4, 5, 7]. Однако различия в обобщенных данных минерального состава руд
месторождений Сопка-Кварцевая, Дальнее, Ороч и Ирбычан [6, 8] указывают на то, что влияние
фундамента на него было крайне незначительным. Это также подтверждается
изотопно-геохимическими данными Р.Г. Кравцовой с соавторами [4].

\textit{Авторы выражают благодарность В.~В.~Акинину,  Н.~А.~Горячеву, Е.~М.~Горячевой, Е.\,Е.~Коловой, Н.~Е.~Савве (СВКНИИ ДВО РАН), С.~Ф.~Петрову, А.~П.~Бороздину (ООО~<<ЛИМС>>), О.~Л.~Галанкиной (ИГГД РАН) за содействие при выполнении данной работы.}


\begin{thebibliography}{99}

\bibitem{}
\BibAuthor{Горячев~Н.~А., Егоров~В.~Н., Савва~Н.~Е., Кузнецов~В.~М., Фомина~М.~И., Рожков~П.~Ю.} Геология и~металлогения фанерозойских комплексов юга Омолонского массива.~--- Владивосток~: Дальнаука, 2017.~--- 312~с.
\bibitem{}
\BibAuthor{Животнев~А.~Я., Литовченко~З.~И.} Структурная позиция Ирбычанского рудопроявления // Материалы по геологии и~полезным ископаемым Северо–Востока СССР.~--- 1977.~---  Кн.~1, №~23~--- С.~162--167.
\bibitem{}
\BibAuthor{Кравцова~Р.~Г.} Геохимия и~условия формирования золото-серебряных рудообразующих систем Северного Приохотья.~--- Новосибирск~: Академическое изд-во <<Геос>>, 2010.~--- 292~с.
\bibitem{}
\BibAuthor{Кравцова~Р.~Г., Дриль~С.~И., Алмаз~Я.~А., Татарников~С.~А., Владимирова~Т.~А.} Первые данные по Rb-Sr возрасту и~изотопному составу золото-серебряных руд месторождения Дальнего (Эвенский рудный район, Северо-Восток России) // Доклады Академии Наук.~--- 2009.~--- Т.~428, №~2.~--- С.~240--243.
\bibitem{}
\BibAuthor{Костырко~Н.~А., Пляшкевич~Л.~Н., Болдырев~М.~В.} Строение и~вещественный состав рудных зон Эвенского рудного поля // Материалы по геологии и~полезным ископаемым Северо–Востока СССР.~--- 1974.~--- №~21.~--- С.~87--94.
\bibitem{}
\BibAuthor{Савва~Н.~Е.} Минералогия серебра Северо-Востока России.~--- М.~: Изд-во Триумф, 2018.~--- 544~с.
\\\bibitem{}
\BibAuthor{Терехов~М.~И.} Стратиграфия и~тектоника южной части Омолонского массива.~--- М.~: Наука, 1979.~--- 114~с.

\textbf{Архивные источники}

\bibitem{}Технико-экономическое обоснование постоянных разведочных кондиций для подсчета запасов золота и серебра месторождений: Сопка-Кварцевая, 2006\,г, 2010\,г, Ороч, 2010\,г, Ирбычан, 2017\,г.~--- Магадан\,: АО <<Полиметалл>>, 2006.

\end{thebibliography}
\thispagestyle{empty}
