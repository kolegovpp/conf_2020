\procTitle{Геохимические индикаторы фациальных обстановок нижне- и~верхнепермских терригенных отложений Омолонского массива}
\procAuthor{Брынько~И.\,В.}
\procEmail{ibrynko@mail.ru}
\procOrganization{СВКНИИ ДВО РАН} \procCity{Магадан}

\makeProcTitleRazdel
\index{b@Брынько~И.\,В.}

Изучение примесных элементов позволяет получить важную информацию относительно фациальных обстановок седиментационных бассейнов в~геологическом прошлом.

В настоящей работе сделана попытка интерпретации данных по малым и редкоземельным элементам из пермских отложений Омолонского массива по двум разрезам (руч.~Водопадный и р.~Русская-Омолонская); их возраст принят согласно [10]. Данные получены методом ICP-MS во ВСЕГЕИ (г.~Санкт-Петербург) и Институте тектоники и геофизики ДВО РАН (г. Хабаровск).

Пермские отложения изучаемой территории представлены джигдалинской, омолонской, гижигинской и хивачской свитами; породы комплексно охарактеризованы в~работе [3]. Джигдалинская свита сложена тёмно-серыми туффитами, туфоалевролитами, алевролитами. Омолонская свита состоит преимущественно из колымиевых известняков и поэтому нами не изучалась. Гижигинская свита представлена серо-зелёными туффитами, тёмно-серыми диамиктитами и серыми туфоалевролитами. Хивачская свита сложена зеленовато-серыми туфоалевролитами, туфопесчаниками, колымиевыми известняками.

В качестве показателей окислительно-восстановительных обстановок для терригенных образований мы использовали достаточно известные геохимические индексы Мо/Мn [4], V/Cr [4], U/Th [11], V/(V+Ni) [12]. Судя по полученным значениям отношения Мо/Мn, породы образовывались в~бескислородных условиях, за исключением нижнегижигинских отложений (значение индекса от 0,001 до 0,007). На отсутствие кислородных обстановок указывают и индексы V/Cr, U/Th, V/(V+Ni), эти результаты подтверждаются отсутствием
бентосных организмов.

Судя по отношению Sr/Ba, в~начале гижигинского и конце хивачского времени наблюдалось некоторое опреснение бассейна [5], а значения Zr/Cu [6] свидетельствуют о том, что все пермские отложения формировались в~типично морских условиях.

В качестве показателей климата мы используем значения $\Sigma$Ce/$\Sigma$Y [1], согласно этим данным, отложения джигдалинской свиты формировались в~условиях аридного климата; гижигинской~--- семиаридного~--- семигумидного; хивачской~--- гумидного. Другим критерием оценки палеоклимата и~степени выветривания пород является индекс химического выветривания CIA [Al$_{2}$O$_{3}$/Al$_{2}$O$_{3}$ + CaO + Na$_{2}$O + K$_{2}$O]~$\times$~100 [14]. Значения этого коэффициента для джигдалинской свиты колеблется в~пределах от 40,8 до 75,6 при среднем значении 64,5, для гижигинской свиты~--- от 48,8 до 64,7 при среднем значении 58,4; для хивачской свиты~--- от 41,5 до 63,9 при среднем значении 56,4. \enlargethispage{\baselineskip}Разграничением для этого коэффициента является значение 70, т.\,е. при значении меньше указанного породы формировались в~условиях умеренного климата, а при более~--- в~условиях теплого и влажного.


Для определения питающей провинции нами были выбраны следующие диаграммы: Nb/Y~--- Zr/TiO$_2$ [16], La/Sc~--- Th/Co [9] и Th/Sc~--- Eu/Eu* [7]. Так, распределение фигуративных точек на диаграммах Nb/Y~--- Zr/TiO$_2$ [16] и La/Sc~--- Th/Co [9] показало, что при образовании всех пермских отложений существенное значение имел размыв пород кислого и среднего состава, при этом хивачские отложения сформировались за счёт размыва более основных пород, чем джигдалинские и гижигинские. Сходные результаты даёт диаграмма Th/Sc~--- Eu/Eu* [7]. Здесь фигуративные точки пород джигдалинской и гижигинской свит преимущественно тяготеют к полю источников средне-кислого состава, а фигуративные точки пород хивачской свиты~--- к~полю источников основного состава.

Для интерпретации палеогеодинамических обстановок нами были выбраны наиболее популярные диаграммы: La/Th~--- Hf [7], K$_2$O\,+\,Na$_2$O~--- SiO$_2$ [15], SiO$_2$/Al$_2$O$_3$~--- K$_2$O/Na$_2$O [13]. Фигуративные точки пород джигдалинской и гижигинской свит на диаграмме La/Th~--- Hf [7] попадают в поле и приурочены к нему островных вулканических дуг, а фигуративные точки хивачской свиты~--- к полю активных континентальных окраин. На диаграмме, предложенной Б.~Роузом и Р.~Коршом K$_2$O\,+\,Na$_2$O~--- SiO$_2$ [15], фигуративные точки джигдалинской свиты расположены в~полях пассивной континентальной окраины и активной окраины; точки гижигинской свиты распределены в~полях активной окраины и океанической островной дуги; а точки хивачской свиты группируются в~поле океанической островной дуги. На диаграмме SiO$_2$/Al$_2$O$_3$~--- K$_2$O/Na$_2$O [13] фигуративные точки джигдалинской свиты попадают в~поле пассивной континентальной окраины, а гижигинской и хивачской свит находятся в~полях островодужных обстановок, тяготея к преддуговым и задуговым бассейнам.

Проведённые исследования позволяют предполагать, что пермские отложения Омолонского массива формировались в~относительно мелководных условиях [2] умеренно тёплого климата, в~бескислородной среде в придонной части
бассейна. Было несколько источников сноса обломочного материала в~осадочный бассейн. Так, для отложений джигдалинской свиты основным источником была кедонская серия кислых и средних вулканитов, к которой примешивался пирокластический материал с Охотско-Тайгоносской вулканической дуги. Отложения гижигинской свиты имели следующие источники сноса: докембрийские метаморфические отложения, кедонскую серию вулканитов, пирокластический и вулканокластический материал Охотско-Тайгоносской вулканической дуги. В позднехивачское время в~бассейн поступали лишь продукты размыва дуги, без пирокластического материала, что, по-видимому, свидетельствует о затухании активности Охотско-Тайгоносской вулканической дуги [8].

\textit{Исследование выполнено при финансовой поддержке РФФИ в рамках научного проекта № 20-05-00604.}


\begin{thebibliography}{99}
%1
\bibitem{}\BibAuthor{Балашов~Ю.~А.} Геохимия редкоземельных элементов.~--- М.~: Наука, 1976.~--- 268 с.
\bibitem{}\BibAuthor{Бяков~А.~С.} Зональная стратиграфия, событийная корреляция, палеобиогеография перми Северо-Востока Азии (по двустворчатым моллюскам).~--- Магадан~: СВКНИИ ДВО РАН, 2010.~---262 с.
\bibitem{}\BibAuthor{Кашик~Д.~С., Ганелин~В.~Г., Караваева~Н.~И. и др.} Опорный разрез перми Омолонского массива.~--- Л.~: Наука, 1990.~--- 200 с.
\bibitem{}\BibAuthor{Холодов~В.~Н., Недумов~Р.~И.} О геохимических критериях появления сероводородного заражения в водах древних водоёмов // Изв. АН СССР. Сер. геол.~--- 1991.~--- №~12.~--- С. 74--82.
\bibitem{}\BibAuthor{Юдович~Э.~Я., Кетрис~М.~П.} Геохимические индикаторы литогенеза.~--- Сыктывкар~: Геопринт, 2011.~--- 742 с.
\bibitem{}\BibAuthor{Яночкина~З.~А.} Малые элементы~--- индикаторы условий седиментации // Литология и полезные ископаемые.~--- 1964.~--- №~2.~--- С. 127--131.
\bibitem{}\BibAuthor{Bhatia~M.~R., Crook~K.~A.~W.} Trace element characteristics of grauwackes and tectonic settings discrimination of sedimentary basins // Contrib. Mineral Petrol.~--- 1986.~--- Vol.~92.~--- P.~181--193.
\bibitem{}\BibAuthor{Biakov~A.~S., Shi~G.~R.} Palaeobiology and palaeogeographical implications of Permian marine bivalve faunas in Northeast Asia (Kolyma-Omolon and Verkhoyansk-Okhotsk regions, northeastern Russia) // Palaeogeography, Palaeoclimatology, Palaeoecology.~--- 2010.~--- Vol.~298.~--- P.~42--53.
\bibitem{}\BibAuthor{Cullers~R.~L.} Implications of elemental concentrations for provenance, redox conditions, and metamorphic studies of shales and limestones near Pueblo, CO, USA // Chem. Geol.~--- 2002.~--- Vol.~191.~--- P.~305--327.
\bibitem{}\BibAuthor{Ganelin~V.~G., Biakov~A.~S.} The Permian biostratigraphy of the Kolyma-Omolon region, Northeast Asia // Journ. of Asian Earth Sciences.~--- 2006.~--- Vol.~26, No.~3--4.~--- P.~225--234.
\bibitem{}\BibAuthor{Hatch~J.~R., Leventhal~J.~S.} Relationship between inferred redox potential of the depositional environment and geochemistry of the Upper Pennsylvanian (Missourian) Stark Shale Member of the Dermis Limestone, Wabaunsee County, Kansas, USA // Chem. Geol.~--- 1992.~--- Vol.~99.~--- P.~65--82.
\bibitem{}\BibAuthor{Jones~В., Manning~D.~A.~C.} Comparison of geochemical indices used for the interpretation of palaeoredox conditions in ancient mudstones // Ibid.~--- 1994.~--- Vol.~111.~--- P. 111--129.
\bibitem{}\BibAuthor{Maynard~J.~B., Valloni~R., Yu~H.~S.} Composition of modern deep-sea sands, from arc-related basins~// Trench~--- Forearc Geology. Sedimentation and tectonics of modern and ancient plate margins. Oxford, London, Edinburg, Melbourne~: Blackwell Scientific Publications, 1982.~--- P.~551--561.
\bibitem{}\BibAuthor{Nesbitt~H.~W., Young~G.~M.} Early Proterozoic climates and plate motions inferred from major element chemistry of lutites // Natire.~--- 1982.~--- Vol.~299.~--- P.~715--717.
\bibitem{}\BibAuthor{Roser~B.~P., Korsch~R.~J.} Determination of tectonic setting of sandstone-mudstone suites using SiO$_2$ content and K$_2$O/Na$_2$O ratio // Journ. Geology.~--- 1986.~--- Vol.~94, No.~5.~--- P.~635--650.
\bibitem{}\BibAuthor{Winchester~J.~A., Floyd~P.~A.} Geochemical discrimination of different magma series and their differentiation products using immobile elements // Chem. Geol.~--- 1977.~--- Vol.~20.~--- P.~325--343.

\end{thebibliography}
\thispagestyle{empty}
