\begin{figure}[H]
  \begin{center}
    \includegraphics[width=0.75\textwidth]{authors/mandzhieva-fig-2.png}
  \end{center}
  \caption{Разрез по месторождениям Стрельцовское и Антей [5].
Условные обозначения: 1~--- фельзиты; 2~--- трахидациты; 3~--- туфы (\textit{а}) и туфолавы (\textit{б}) трахидацитов; 4~--- базальты; 5~--- конгломераты (\textit{а}) и структурный элювий гранитов (\textit{б}); 6~--- гранитоиды; 7~--- крутопадающий разлом (\textit{а}) и пологие срывы (\textit{б}); 8~--- рудные зоны и отдельные тела. Слева~--- шкала высот над уровнем моря, справа~--- положение в разрезе и номера горизонтов горных выработок}
  \label{fig:mandzhieva-fig2}
\end{figure}
