
\procTitle{Термохимический синтез волластонита из гипсового техногенного сырья}
\procTitleNewLine{Термохимический синтез волластонита\\из гипсового техногенного сырья}
\procAuthorI{Ярусова~С.\,Б., Гордиенко~П.\,С.}
\procEmailI{yarusova\_10@mail.ru}
\procOrganizationI{Институт химии ДВО РАН}
\procCityI{Владивосток}

\procAuthorII{Данилова~С.\,Н., Охлопкова~А.\,А.}
\procEmailII{dsn.sakhayana@mail.ru}
\procOrganizationII{СВФУ}
\procCityII{Якутск}

\makeProcTitleIINewLine

\index{g@Гордиенко~П.\,С.}
\index{d@Данилова~С.\,Н.}
\index{o@Охлопкова~А.\,А.}
\index{z@Ярусова~С.\,Б}

Широкое применение волластонита Ca$_6$Si$_6$O$_{18}$ в различных отраслях промышленности обусловлено рядом ценных технологических свойств данного минерала. Волластонит используется в качестве различных добавок в~материалы в целях увеличения их прочности, жаростойкости, химической стойкости и износостойкости, улучшения диэлектрических и~электрических характеристик, сокращения длительности технологических процессов при их изготовлении, снижения температуры обработки [1, 5].

Ранее [4, 6] в~Институте химии ДВО РАН была показана возможность получения волластонита из отходов производства борной кислоты (борогипса). Исследованы процессы получения гидросиликатов кальция и~волластонита в автоклавных условиях, а также в~условиях ультразвуковой и микроволновой обработки борогипса [2, 3, 6].

\begin{table}[H]
\caption*{\textbf{Кристаллические фазы волластонита в продуктах щелочной обработки борогипса}}
\label{tab:yrysov-tab}
\begin{tabular}{m{2cm}m{4cm}m{6cm}}
   \toprule
Щелочь                     & Температура и время обжига осадка на второй стадии термической обработки   & Кристаллическая модификация волластонита и параметры кристаллической ячейки волластонита \\
\midrule
\multirow{3}{*}{\raisebox{3ex}[2.2cm][0cm]{NaOH}} & 900\,\dgc, 1~ч. & Триклинная,\newline а=10,0400; b=11,05400; с=7,30500;\newline $\upalpha$=99,530; $\upbeta$=100,560; $\upgamma$=83,440 \\
\cmidrule(r){2-3}
                      & 950\,\dgc, 1~ч. & Триклинная,\newline а=10,0400; b=11,05400; с=7,30500;\newline $\upalpha$=99,530; $\upbeta$=100,560; $\upgamma$=83,440 \\
                      \cmidrule(r){2-3}
                      & 950\,\dgc, 1,5~ч. & Триклинная,\newline а=7,92580; b=7,32020; с=7,06530;\newline $\upalpha$=90.055; $\upbeta$=95.217; $\upgamma$=103.426 \\
                      \midrule
\multirow{3}{*}{\raisebox{3ex}[2.2cm][0cm]{KOH}}  & 900\,\dgc, 1~ч. & Триклинная,\newline а=7,89600; b=7,28500; с=7,08400;\newline $\upalpha$=90.000; $\upbeta$=95.270; $\upgamma$=103.370\\
\cmidrule(r){2-3}
                      & 950\,\dgc, 1~ч. & Триклинная,\newline а=7,92580; b=7,32020; с=7,06530;\newline $\upalpha$=90.055; $\upbeta$=95.217; $\upgamma$=103.426 \\\cmidrule(r){2-3}
                      & до 1000\,\dgc & Триклинная,\newline а=7,92580; b=7,32020; с=7,06530;\newline $\upalpha$=90.055; $\upbeta$=95.217; $\upgamma$=103.426 \\ \bottomrule\\
\end{tabular}
\end{table}


Целью данной работы является исследование закономерностей формирования волластонита из борогипса (Горно-химическая компания <<Бор>>, Приморский край, г.~Дальнегорск) путём термохимического синтеза.

Борогипс\,\, характеризуется\,\, следующим\,\, содержанием\,\, основных\,\, компонентов,\,\, мас.\,\%: SiO$_2$~--- 26--28; CaO~--- 26--28; SO$_4^{-2}$~--- 38--40; Fe$_2$O$_3$~--- 1,8--2,0; Al$_2$O$_3$~--- 0,6--0,8; B$_2$O$_3$~--- 0,7--1,2; MnO~--- 0,2; MgO~--- 0,1--0,2.

Исходные компоненты (борогипс и гидроксид калия или натрия) смешивали в стехиометрическом соотношении, помещали в корундовый тигель и добавляли дистиллированную воду в соотношении твёрдой и жидкой фаз, равном 1:2.\enlargethispage{\baselineskip}

Термохимический синтез волластонита осуществляли в два этапа:
\begin{enumerate}[noitemsep]\vspace{-8pt}
  \item нагрев реакционной смеси с периодическим перемешиванием в интервале температур 200--300\,\dgc в течение 1--2~ч; промывка полученного осадка дистиллированной водой от~щелочи и растворимых солей с последующей сушкой при температуре 80--95\,\dgc;
    \item обжиг осадка при температуре 900--1000\,\dgc в течение 1,0--1,5~ч.
\end{enumerate}
 \vspace{-8pt}

Согласно данным рентгенофазового анализа, щелочная обработка борогипса гидроксидом калия или гидроксидом натрия в условиях нагрева при температурах 200--300\,\dgc с последующим обжигом в интервале температур 900--1000\,\dgc приводит к формированию волластонита триклинной модификации с параметрами кристаллической ячейки, приведёнными в таблице.


%\vspace{-20pt}
\begin{thebibliography}{99}
\bibitem{}\BibAuthor{Гладун~В.~Д., Акатьева~Л.~В., Холькин~А.~И.} Синтетические силикаты кальция.~--- М.~: ИРИСБУК, 2011.~--- 232 с.
\bibitem{}\BibAuthor{Гордиенко~П.~С. и др.} Влияние микроволновой обработки на кинетику формирования и морфологию гидросиликатов кальция / авт. П.~С.~Гордиенко, В.~В.~Баграмян, С.~Б.~Ярусова, А.~А.~Саркисян, Г.~Ф.~Крысенко, Н.~В.~Полякова, Ю.~В.~Сушков // Журнал прикладной химии.~--- 2012.~--- Т.~85.~--- Вып.~10.~--- С.~1582--1586.
\bibitem{}\BibAuthor{Гордиенко~П.~С. и др.} Влияние ультразвуковой обработки на кинетику формирования гидросиликата кальция из борсодержащих техногенных отходов / авт. П.~С.~Гордиенко, С.~Б.~Ярусова, А.~П.~Супонина, Ю.~В.~Сушков, В.~А.~Степанова // Химическая технология.~--- 2014.~--- Т.~15, №~10.~--- С.~577--581.
\bibitem{}Способ комплексной переработки борогипса [Текст]~: пат. 2601608 Рос. Федерация.~--- №~2015141651/05~; заявл. 30. 09. 2015~; опубл. 10.11.16, Бюл. №~31.~--- 8~c.
\bibitem{}Способ получения волластонита [Текст]~: пат. 2595682 Рос. Федерация.~--- №~2015141614/05~; заявл. 30. 09.2015~; опубл. 27.08.16, Бюл. №~24.~--- 9~c.
\bibitem{}\BibAuthor{Тюльнин~В.~А., Ткач~В.~Р., Эйрих~В.~И., Стародубцев~Н.~П.} Волластонит: уникальное минеральное сырье многоцелевого назначения.~--- М.~: Руда и металлы, 2003.~--- 144~с.

\end{thebibliography}
