\bigskip
\procTitle{Деградация растительных сообществ, почв под действием антропогенного фактора в условиях Магаданской области}

\procAuthor{Красноперова~В.}
\procEmail{petycay@mail.ru}
\procSchool{МКОУ СОШ п.~Ола, 9А класс}
\procCity{пос.\,Ола}
\procSuperviser{к.\,с.-х.\,н. Фандеева~Я.\,Д.}
\procSuperviserOrganization{МКОУ СОШ п.~Ола}


\makeProcTitleSchool

\index{k@Красноперова~В.}

\textbf{Цель}: выявить антропогенное влияние на почву и растительные сообщества в условиях Магаданской области.

\textbf{Задачи}:
\begin{enumerate}[noitemsep]\vspace{-8pt}
\item проанализировать литературные источники по данной теме;
\item определить видовой состав растений в окрестностях пос.~Ола;
\item установить основные антропогенные факторы, оказывающие влияние на почву и растительные сообщества;
\item изучить способы восстановления растительности.	\enlargethispage{\baselineskip}
\end{enumerate}\vspace{-8pt}

\textbf{Объект}: сообщества травянистых растений.

\textbf{Предмет}: изменение видового состава травянистых растений и почв под действием антропогенного фактора.

Использована простейшая методика (А.~С.~Боголюбов, А.~Б.~Панков), позволяющая стандартизировать процедуру описания сообществ травянистых растений~--- проведены в~двух растительных сообществах, подверженных присутствию человека~--- на Нюклинской косе и~<<массовой>> поляне в окрестностях пос.~Ола, излюбленном месте для отдыха.

Видовая насыщенность травянистых растений на Нюклинской косе составляет 15 видов на~1\,м$^2$. Меньше всего видов представлено на туристической площадке~--- 9 видов на~1\,м$^2$.

Между площадками 1 и 3 наблюдаются серьёзные отличия в видовом разнообразии растений, связанных, прежде всего, со способностью некоторых растений приспосабливаться к~действию антропогенного фактора, в частности вытаптыванию. В основном эти растения имеют невысокий рост, сильную и разветвлённую корневую систему.

Проективное покрытие травянистого яруса это подтверждает: на площадке с максимальным действием человека оно составляет всего 45\,\%, что ниже на 35\,\% площадки с~естественным произрастанием растений.

Из всех травянистых растений, произрастающих в этих условиях, более устойчивыми к~вытаптыванию оказалась осока камчатская, которая имеет сильно развитый куст. Установлены виды растений, которые преобладают на участке с действием антропогенного характера: тысячелистник обыкновенный, подорожник камчатский, одуванчик рогоносный. Наиболее уязвимые растения: горечавка холодная, ива чукотская, герань волосистоцветковая, борщевик шерстистый, погремок малый, горец земноводный, пухонос альпийский.

Видовая насыщенность травянистых растений на Нюклинской косе составляет 14 видов на~1\,м$^2$. Меньше всего видов представлено на третьей площадке~--- 10 видов на~1\,м$^2$. Виды растений, которые преобладают на участке с действием антропогенного характера: одуванчик рогоносный, герань волосистоцветковая, тысячелистник обыкновенный. Наиболее уязвимые растения: гравилат крупнолистный, бухарник шерстистый, ирис щетинистый, цинна широколистная.

Одним из основных антропогенных факторов являются лесные пожары, наносящие огромный экономический ущерб. Нарушения приводят к необратимым последствиям, в результате чего экосистема деградирует. Основным способом лесовосстановления в Магаданской области является естественное восстановление лесов, которое обеспечивается проведением мер содействия естественному возобновлению, а также естественное заращивание вырубок и гарей. Основные лесообразующие породамы на территории Магаданской области: из хвойных – лиственница, из мягколиственных~--- ива древовидная, тополь и берёза.

В ходе нашего исследования мы пришли к следующим выводам:

\begin{enumerate}[noitemsep]\vspace{-8pt}
\item рекреационная деятельность оказывает существенное негативное влияние на растительные сообщества. Установлена закономерность снижения многообразия растительности от активной деятельности человека в местах её произрастания;
\item выявлены наиболее уязвимые по отношению к фактору вытаптывания виды растений;
\item самым негативным антропогенным фактором на территории Магаданской области являются лесные пожары по вине человека, приносящие огромный урон растительным сообществам, животному миру и почвенному покрову;
\item определена роль экологических волонтёров, способных помочь лесничествам в восстановлении утраченных лесов.
\end{enumerate}\vspace{-8pt}

Полученные результаты исследования имеют практическое применение, возможно их использование в целях снижения антропогенной нагрузки на туристические места для постепенного восстановления растительного и почвенного покрова, а также привлечения внимания к этой проблеме населения Магаданской области.\enlargethispage{\baselineskip}
