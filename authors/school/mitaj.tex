\bigskip
\procTitle{<<Я знаю~--- саду цвесть\dots>>}

\procAuthor{Митяй~А.}
\procEmail{magmakoll@mail.ru
}
\procSchool{МАОУ <<СО(РК)Ш №\,2>>, 2Б класс}
\procCity{Магадан}
\procSuperviser{Малюкова~М.\,В.}
\procSuperviserOrganization{МАОУ <<СО(РК)Ш №\,2>>}


\makeProcTitleSchool

\index{m@Митяй~А.}

\textbf{Цель работы}: изучение особенностей выращивания яблони в непростых климатических условиях северного края; привлечь внимание общества к восстановлению садово-паркового искусства в г.Магадане.

\textbf{Гипотеза}: если мы овладеем определёнными знаниями об особенностях культивирования яблони, то сможем выращивать и получать растения не характерные для нашего края и разнообразить флору нашего города.

\textbf{Задачи}:

\begin{itemize}[noitemsep]\vspace{-8pt}
\item изучить литературу о первых опытах выращивания маленьких саженцев яблони в~Магадане;
\item изучить литературу об особенностях выращивания яблони и правильному уходу за~ней;
\item провести беседу со специалистами по данной теме;
\item провести исследование  выращивания яблони в домашних условиях;
\item привлечь внимание окружающих к восстановлению флорического питомника в г.Ма\-га\-да\-не.
\end{itemize}\vspace{-8pt}

Я родился и живу в Магадане. Летом мы с семьёй ездим в отпуск на материк или отдыхаем у дедушке в пос.~Стекольный. Наша колымская природа не похожа на природу центральных районов нашей страны. Там растут фруктовые деревья, разные ягоды и овощи. Дедушка тоже выращивает душистую  клубнику, сладкую малину и смородину. Я~с~удовольствием помогаю ему ухаживать за растениями.

Знакомясь на уроках <<Окружающего мира>> с природными явлениями, изучая растительный и животный мир, я задумался о различии природы в нашей большой стране. Мама рассказала мне, что недалеко от нашей школы в 30-е годы прошлого века Александр Хмелинин создал Охотско-Колымский краеведческий музей и на пустом месте возле него разбили флорический питомник, где вырастили яблони, вишни, сирень и другие необычные для нашего края растения. Именно тут вырастили первую колымскую малину и клубнику. И этот музей находился совсем рядом с нашей школой, и я неоднократно проходил около него по пути домой.

Но в 2015~г. старый музей сгорел. При пожаре пострадал и уникальный дендрарий. Только некоторые растения удалось сохранить.
В 2016~г. из прессы стало известно, что правительство Магадана планирует восстановить плодово-ягодное искусство.

Мамин рассказ о ботаническом саде заинтересовал меня. Я решил, что, если в XX~веке колымчане смогли вырастить уникальный сад, то сейчас можно повторить этот опыт.

Если мне удастся вырастить хотя бы одну яблоню, то значит в будущем наш родной Магадан сможет радовать колымчан  яблоневым цветом.

Я думаю, что многие ребята, узнав эту удивительную историю, с удовольствием примут участие в восстановлении пострадавшего комплекса и попытаются вырастить какие-нибудь из утерянных растений.
