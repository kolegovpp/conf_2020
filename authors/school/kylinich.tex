\bigskip
\procTitle{Бионика. Сердце~--- пламенный мотор}

\procAuthor{Кулинич~Н.}
\procEmail{olga.gerashenko.68@mail.ru}
\procSchool{МАОУ <<Лицей (эколого-биологический)>>, 10А класс}
\procCity{Магадан}
\procSuperviser{Воловодюк~Е.\,В.}
\procSuperviserOrganization{МАОУ <<Лицей (эколого-биологический)>>}


\makeProcTitleSchool

\index{k@Кулинич~Н.}


\textbf{Цель работы}: сравнить строение сердца с двигателем внутреннего сгорания.

\textbf{Задачи работы}: узнать, что такое бионика, её историю и применение в современном мире.

\textbf{Гипотеза}: сердце~--- пламенный мотор! Просто строчка из песни? Или объяснимый с~точки зрения бионики принцип работы?

\textbf{Актуальность}: человек, изобретая что-то новое, всегда пользуется уже созданным природой.

\textit{В теоретической части} говорится о бионике. Даётся понятие бионики, где она зародилась. Где её можно применить в современном  мире.

\textit{В практической части} сравнивается первый автомобиль с внутренними органами человека и сердце человека с двигателем внутреннего сгорания.

\textbf{Вывод}. Из моего исследования я могу сделать вывод, что работа сердца и работа двигателя действительно построены на одних принципах. Несомненно, что и различий много, и это говорит о труде изобретателей. Значит ли, что сердце~--- пламенный мотор? Теперь, на основе своих сравнений, я могу уверенно сказать <<Да!>>.

Бионика стремительно вошла во все отрасли нашей жизни. Сейчас она имеет несколько направлений:
\begin{enumerate}[noitemsep]\vspace{-8pt}
\item архитектурно-строительная бионика;
\item исследование органов чувств организмов для создания датчиков;
\item разработка материалов, копирующих природные;
\item изучение принципов ориентации и навигации животных для использования в технике;
\item робототехника и искусственные органы;
\item нейробионика~--- создание искусственного интеллекта.
\end{enumerate}\vspace{-8pt}

Как мы видим, без этой науки не возможен прогресс, и помогает в этом природа, создавшая уникальные материалы, системы, органы. А нам стоит лишь всё изучить и с помощью бионики сделать окружающий мир чуточку лучше.
