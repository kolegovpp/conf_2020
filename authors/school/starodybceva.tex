\procTitle{Правильное питание~--- залог здоровья и~долголетия}

\procAuthor{Стародубцева~В.}
\procEmail{svetlana.kuznetsova.2012@mail.ru
}
\procSchool{МАОУ <<Гимназия (английская)>>, 2Б класс}
\procCity{Магадан}
\procSuperviser{Алтынбаева~Э.\,Р.}
\procSuperviserOrganization{МАОУ <<Гимназия (английская)>>}


\makeProcTitleSchool

\index{s@Стародубцева~В.}

10А

\textbf{Цель исследования}: выяснить, полезна ли рыба для детского организма.

\textbf{Задачи исследования}:
\begin{enumerate}[noitemsep]\vspace{-8pt}
\item выяснить, полезна рыба или нет? Какая рыба полезна для детей?
\item в какой рыбе больше полезных веществ~--- в~красной или белой?
\item как часто необходимо употреблять рыбу?
\item как правильно готовить рыбу?
\end{enumerate}\vspace{-8pt}

\textbf{Актуальность темы}: Здоровый образ жизни~--- тема старая и~одновременно новая, которая будет всегда интересовать и~взрослых, и~детей. Во все времена человечество стремится быть здоровым и~жить долго.

\textbf{Гипотеза}. Рыба не является очень полезной для детей. Чтобы разобраться в~этом вопросе, я решила провести собственное исследование. Но предварительно я решила выяснить, что знают мои одноклассники и~другие ученики начальных классов о пользе рыбы.

Для изучения вопроса о пользе и~употреблении разных сортов рыбы я провела анкетирование гимназистов 1-х классов МАОУ~<<Гимназия (английская)>> в~количестве 68 человек, 49~сотрудников ООО <<Кинросс Дальний Восток>> и~15~сотрудников кейтеринговой фирмы ООО~<<АКС>>. Всего были опрошены 132 человека.

В результате исследований, опросов, изучения научно-познавательной и~справочной литературы я выяснила, что рыба считается диетическим продуктом и~обладает значительными полезными свойствами. По сравнению с~мясом рыба намного быстрее переваривается (как минимум в~2 раза быстрее). Польза и~вред рыбы для организма, калорийность меняются в~зависимости от технологической и~кулинарной обработки.

Так я узнала, что если есть рыбу каждый день, то это положительно влияет на память~--- а значит, рыбу просто необходимо есть детям, им это понадобится для учёбы; концентрацию~---  с~возрастом люди становятся более рассеянными, употребляя рыбу, им удастся снизить данную тенденцию; стрессы и~депрессию~--- как говорят специалисты, люди, которые чаще едят рыбу, становятся менее подвержены стрессам, благодаря выработке гормона радости, который способен поддерживать наше настроение; сон~--- помогает при бессоннице; работу сердца~--- при употреблении рыбы снижается риск возникновения инфарктов и~инсультов, конечно, это не панацея, но, по заявлениям специалистов, кто чаще ест рыбу, тот меньше страдает от проблем с~сердцем; обмен веществ~--- питательные вещества, омега-3 кислоты активизируют наш организм и~поддерживают его в~тонусе; иммунную систему~--- улучшается работа иммунной системы, благодаря чему мы становимся менее подверженными риску заболевания гриппом и~простудой; авитаминоз~--- в~зимнее время многие страдают авитаминозом из-за отсутствия натуральных витаминов.

Согласно исследованиям, продолжительность жизни зависит на 30\,\% от генетической предрасположенности и~на 70\,\% от образа жизни, т.\,е. сна, физической деятельности, профессии и~питания. С помощью правильно подобранного рациона еды реально продлить свою жизнь на несколько десятилетий, а точнее~--- убрать причины, которые её сокращают. По~проведённым в~2014\,г. исследованиям оказалось, что половина из 40 людей в~мире, которым исполнилось 110~лет, живут в~Японии. Чем же питаются японцы-долгожители? Благодаря питанию, основанному на рыбном меню, у них самый низкий показатель по смертности от инсультов и~сердечных приступов, они практически не употребляют мясо, исключение составляют нежирная курица и~утка. 

Проведя свои исследования, я пришла к следующим \textbf{выводам}:
\begin{enumerate}[noitemsep]\vspace{-6pt}
\item рыба является полезным продуктом питания для человека~--- польза и~вред рыбы во~многом зависят от её вида или сорта и~способа приготовления. Но в~общем можно сказать, что это ценный диетический продукт, который необходим человеку для~поддержания полноценного здоровья и~долголетия;
\item полезные вещества имеются как в~белой, так и~в~красной рыбе,  причём развеян миф о том, что в~красной рыбе полезных веществ больше, а это не так;
\item употреблять рыбу необходимо 2--3 раза в~неделю, а если брать пример с~японцев~--- то~рыбу можно и~нужно употреблять каждый день;
\item лучшие способы приготовления рыбы~--- это запекание, отваривание и~приготовление на пару.
\end{enumerate}\vspace{-6pt}

Питание, основанное на рыбном меню, очень актуально для детей, проживающих в~условиях Крайнего Севера, так как нехватка витаминов и~тёплых солнечных дней негативно сказывается на их здоровье.

Правильное питание поможет сохранить здоровье детей на долгие годы!
