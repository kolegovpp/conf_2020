\procTitle{Фонтан своими руками}

\procAuthor{Аношко~Д.}
\procEmail{svetlana.kuznetsova.2012@mail.ru
}
\procSchool{МАОУ <<Гимназия (английская)>>, 2Б класс}
\procCity{Магадан}
\procSuperviser{Алтынбаева~Э.\,Р.}
\procSuperviserOrganization{МАОУ <<Гимназия (английская)>>}


\makeProcTitleSchool

\index{a@Аношко~Д.}

\textbf{Цель исследования}: разработка и изготовление фонтана в домашних условиях.

\textbf{Задачи исследования:} 1. рассмотреть устройство фонтанов; 2. разработать свой вариант фонтана; 3. изготовить фонтан для интерьера собственной квартиры, составить технологическую карту сборки фонтана.

\textbf{Реализация проекта}. Определив дизайн будущего домашнего фонтана, мы начали с выбора необходимых материалов и инструментов. Описав действия при создании фонтана у нас получилась технологическая карта его изготовления.

Для начала мы выбрали заготовку для площадки фонтана и пластиковый сосуд, который будет являться водоёмом. С помощью горячего клея мы закрепили водоём.

Чтобы наш фонтан получился привлекательным и интересным, для оформления трубки мы использовали детали конструктора LEGO в виде пальмы. Так же воспользовались горячим клеем для закрепления трубки и насоса-помпы.

Дно водоёма мы украсили искусственными водорослями и ракушками. И конечно же декорировали поверхность площадки.
Фонтан готов! Последний этап~--- это залив воды и подключение его к сети.

В ходе работы над проектом мы узнали об устройстве фонтана, о способах его изготовления. Мы остались довольны итоговым результатом своей работы~--- создан неповторимый фонтан! Нам было приятно увидеть в нём элемент интерьера, гармонично вписавшийся в нашу квартиру и приносящий пользу всем членам нашей семьи, являясь мощным увлажнителем воздуха.

Мы убедились, что фонтан можно создать самостоятельно. Составленную нами технологическую карту сборки фонтана можно использовать на уроках технологии, в результате чего полученный фонтан можно установить в классной комнате, где он будет также выполнять не только декоративную функцию, но и функцию ионизатора.
