\procTitle{<<Есть ли режим у~птиц?>>}

\procAuthor{Сафонова~К.}
\procEmail{svetlana.kuznetsova.2012@mail.ru
}
\procSchool{МАОУ <<Гимназия (английская)>>, 3Б класс}
\procCity{Магадан}
\procSuperviser{Вронская~О.\,Ю.}
\procSuperviserOrganization{МАОУ <<Гимназия (английская)>>}


\makeProcTitleSchool

\index{s@Сафонова~К.}


\textbf{Цель исследования}: подкармливая птиц этой зимой, я решила выяснить для себя, есть ли у~них режим.

\textbf{Задачи исследования}:
\begin{enumerate}[noitemsep]\vspace{-8pt}
\item провести наблюдение за птицами, прилетающими к кормушкам;
\item вести дневник прилёта птиц;
\item опознать птиц, прилетающих к кормушкам;
\item сделать выводы.
\end{enumerate}\vspace{-8pt}

\textbf{Объект исследования}: птицы.

\textbf{Предмет исследования}: наблюдение за птицами с~целью определить наличие либо отсутствие у~них режима дня.

\textbf{Актуальность темы}: зимой птицам очень тяжело, потому что наступают сильные холода и~им надо хорошо питаться, чтобы выжить. Помочь птицам в~этом может и~должен человек.

У каждого человека есть режим, у~нас с~вами тоже. Каждый день примерно в~одно и~то~же время мы просыпаемся, умываемся, завтракаем, идём в~школу. После школы возвращаемся домой, обедаем, делаем уроки, занимаемся домашними делами. Вечером ужинаем, умываемся и~ложимся спать. И так каждый день. Это и~есть наш режим дня. Мне стало интересно, а есть ли режим дня у~птиц? Для этого я решила понаблюдать за птицами.

\textbf{Гипотеза исследования}: у~птиц тоже есть режим дня.

\textit{Первая часть проекта}~--- это наблюдение за тем, какие птицы прилетают к кормушкам.

\textit{Вторая часть проекта}~--- изучение литературы с~целью узнать, как появились воробьи на Колыме, что это за птицы, а также наблюдение за тем, в~какое время прилетают птицы к кормушкам (продолжительность моего наблюдения~--- 14 дней).

Изучив литературу, я выяснила, что изначально воробьёв на Колыме не было, они появились тогда, когда отмечали тридцатилетие города и~Магадану решили сделать подарок: наловили воробьёв (несколько сотен) в~Приморском крае и~отправили птиц пароходом в~клетках в~Магадан.

После этого я провела наблюдение за птицами, которые прилетали к кормушке. Я заносила время прилёта птиц в~таблицу.

\textbf{Сделаны выводы}:

У воробьёв, как и~у нас с~вами, есть режим. В течение 14 дней я наблюдала за птицами, снимала видео и~фото, подтверждающие, что каждый день воробьи прилетали к кормушкам, чтобы поесть, примерно в~одно и~то же время~--- с~08:00 до 10:00 утром, а также с~17:00 до 18:00 вечером. В ходе наблюдения я выявила интересный факт. Когда на улице становилось холоднее, то есть температура опускалась ниже 28 градусов, воробьи прилетали не 2~раза в~день, а~3~раза. Из этого я сделала вывод, что в~сильные морозы птицам требуется больше пищи, чтобы не замёрзнуть.

В результате своего наблюдения я подтвердила гипотезу: <<У птиц существует режим дня>>.
