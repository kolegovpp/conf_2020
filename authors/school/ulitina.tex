\procTitle{H$_2$O~--- что в имени тебе моём?}

\procAuthor{Улитина~С.}
\procEmail{anneta@north-east.ru
}
\procSchool{МАОУ <<Гимназия (английская)>>, 4А класс}
\procCity{Магадан}
\procSuperviser{Липинская~М.\,А.}
\procSuperviserOrganization{МАОУ <<Гимназия (английская)>>}


\makeProcTitleSchool

\index{u@Улитина~С.}

\textbf{Целью проекта} явилось определение микроэлементного состава воды как основного источника химических элементов в организме человека.

Для решения поставленной цели были выделены следующие \textbf{задачи исследования}:
\begin{enumerate}[noitemsep]\vspace{-8pt}
\item провести анкетирование среди учащихся, чтобы узнать, какую воду они предпочитают употреблять, считают ли фильтрованную более полезной, и обработать результаты;
\item проанализировать элементный состав воды из различных источников (родниковая вода, водопроводная вода, очищенная с помощью фильтра и детская вода марки <<Фруто-няня>>) на содержание в них жизненно важных для человека микроэлементов и отсутствие вредных;
\item изучить состав водопроводной воды до и после очистки её через фильтр;
\item проанализировать полученные результаты и сделать выводы;
\item дать рекомендации.
\end{enumerate}\vspace{-8pt}

\textbf{Гипотеза исследования}: магаданскую водопроводную воду необходимо пропускать через фильтр для её очистки.

\textbf{Методы исследования}:

\textit{Анкетирование}. В рамках исследования было проведено анкетирование среди учащихся. Всего приняли участие 54 человека.

\textit{Исследование воды} на определение концентраций вредных элементов и полезных минералов и сравнение их с предельно допустимыми концентрациями с использованием атомно-эмиссионного спектрометра с микроволновой плазмой. В результате чего были получены результаты анализа в виде таблицы с содержанием элементов в пробах воды и сделаны выводы об их содержании.

\textbf{Заключение}. Проведенные мною исследования показали, что микроэлементный состав водопроводной воды г. Магадана по жизненно важным элементам (кальций, магний, калий, натрий) находился ниже нормативного диапазона для данных элементов с содержанием железа, соответствующим нормативным показателям. Исходя из этого, мы можем сделать заключение, что наша вода слабо минерализована жизненно необходимыми микроэлементами. При этом во всех пробах водопроводной воды тяжёлых вредных элементов обнаружено не было. Проба воды торговой марки <<Фруто-няня>> соответствовала нормам, за исключением содержания в ней калия, концентрация которого находилась ниже нормативного диапазона.

При использовании фильтра в водопроводной воде отмечалось снижение жизненно важных микроэлементов и в целом ухудшение качества воды. Исходя из этого, можно сделать вывод, что использование данных фильтров в г.\,Магадан не актуально, так как их действие снижает минерализацию воды и приближает состав воды к дистиллированной, что~противоречит гипотезе моего исследования.

\textbf{Выдвинутая мною гипотеза опровергнута}.

На основании моих данных сформированы рекомендации. Жителям города рекомендуется использовать бутилированную воду, обогащённую микроэлементами или вводить микроэлементные добавки в рацион питания.
