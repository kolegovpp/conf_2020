\bigskip
\procTitle{Исследование качества глазированных творожных сырков}

\procAuthor{Бидоленко~Р.}
\procEmail{svetlana.kuznetsova.2012@mail.ru
}
\procSchool{МАОУ <<Гимназия (английская)>>, 2Б класс}
\procCity{Магадан}
\procSuperviser{Алтынбаева~Э.\,Р.}
\procSuperviserOrganization{МАОУ <<Гимназия (английская)>>}


\makeProcTitleSchool

\index{b@Бидоленко~Р.}

\textbf{Цель исследования}: исследовать состав глазированных творожных сырков и научиться правильно выбирать данный продукт.

\textbf{Задачи исследования}:
\begin{itemize}[noitemsep]\vspace{-8pt}
\item изучить историю возникновения глазированных сырков;
\item выяснить состав и полезные свойства сырков;
\item провести исследование качества сырков;
\item выявить различия между сырками разных производителей;
\item рекомендовать способы выбора наиболее качественных видов сырков.
\end{itemize}\vspace{-8pt}

\textbf{Объект исследования}: глазированные творожные сырки различных фирм-произ\-во\-ди\-те\-лей, поступающие в торговые сети г.~Магадана.

\textbf{Предмет исследования}: соответствие глазированных творожных сырков потребительским качествам, а также различие сырков в зависимости от их производителя.

\textbf{Способы исследования}:
\begin{enumerate}[noitemsep]\vspace{-8pt}
\item изучение надлежащей литературы;
\item проведение опытов;
\item сравнение и наблюдение;
\item анализ, обработка и систематизация результатов;
\item опрос и составление рейтингов;
\item вывод.
\end{enumerate}\vspace{-8pt}

\textbf{Актуальность темы}

На прилавках магазинов появилось огромное количество сырков, которые любят дети. Мне захотелось разобраться~--- всегда ли мы покупаем и употребляем качественные сырки, полезны ли они для здоровья, можно ли определить их качество.

Сырок~---  это творожный десерт, который изготавливается из творожной массы и покрывается глазурью. Однако на красивых обёртках указан очень большой список компонентов. Помимо творога, сахара и масла, в составе также имеются растительный жир, ванилин, ароматизаторы, какао-порошок и многие другие не самые полезные вещества. Именно в~твороге содержатся необходимые для организма человека витамины и различные вещества (кальций, магний, фосфор). Шоколадная глазурь~--- это тоже не только вкусный компонент, но и продукт, который стимулирует работу мозга и улучшает настроение.

Как же избежать обмана, что нужно знать при покупке сырков? Необходимо знать некоторые методы определения его качества. Я  хочу привести  некоторые рекомендации, которые помогут вам купить качественный продукт.

Для эксперимента я взял сырки известных производителей, которые имеются на прилавках магазинов г.~Магадана: <<Простоквашино>>, <<Советские традиции>>, <<Б.\,Ю.\,Александров>>, <<Сваля>>, <<Ностальгия>>, <<Ростагроэкспорт>>, <<Свитлогорье>>, а также сырок местного производителя~--- <<Лучик>> (гормолзавод <<Магаданский>>).

Я проанализировал состав этих сырков по следующим параметрам: читаемость этикетки, общее число компонентов, калорийность, наличие добавок, срок хранения.

Многие недобросовестные производители добавляют в творог добавки и загустители¸ в~том числе и крахмал. Наличие крахмала можно установить при помощи раствора йода. В~результате реакции появляется чёрно-фиолетовое окрашивание. Для этого я к каждому образцу добавил спиртовой раствор йода. В результате эксперимента установлено, что в~глазированном творожном сырке <<Простоквашино>> обнаружен крахмал, который не указан на упаковке, так как цвет творога стал чёрно-фиолетовым. В остальных сырках я~не~заметил изменения окраски йода.

Для получения информации об употреблении сырков я провёл опрос среди учеников первых классов нашей гимназии.

Анализируя результаты опроса, я понял, что многие употребляют в пищу сырки, так как считают их полезным продуктом. Большинство употребляют сырки как вкусный десерт либо в качестве перекуса, чтобы утолить голод. Половина знает, как правильно выбрать сырки в магазине. Однако не многим известно, из чего он состоит.

Исследованные нами образцы показали, что они соответствуют стандартам качества. Однако наиболее качественными и натуральными продуктами являются сырки, в составе которых наименьшее количество добавок и у которых срок хранения составляет не более 15 суток.

Мои исследования подтвердили, что на прилавках магазинов г.~Магадана наиболее качественным и натуральным продуктом является глазированный творожный сырок <<Лучик>> (гормолзавод <<Магаданский>>). Эти же выводы были сделаны моим братом Бидоленко Богданом в ходе исследования молока и йогуртов. Он установил, что наиболее качественными и полезными продуктами являются продукты местных производителей.

Проведённые мною исследования позволили разработать следующие рекомендации:
\begin{itemize}[noitemsep]\vspace{-8pt}
\item сырок должен быть изготовлен из натуральных молочных компонентов, информация на упаковке должна быть легко читаемой;
\item упаковка сырка должна быть целой, без подтеканий и следов краски;
\item срок хранения охлаждённых сырков не должен быть больше 15 суток;
\end{itemize}\vspace{-8pt}

Правильно выбранный творожный сырок не только улучшит состояние нашего здоровья, но и поднимет настроение.
