\procTitle{Анализ жидких моющих средств для посуды}

\procAuthor{Иванова~Е.}
\procEmail{elizaveta.2735@yandex.ru}
\procSchool{МАОУ <<Лицей (эколого-биологический)>>, 10А класс}
\procCity{Магадан}
\procSuperviser{Нечаева~Г.\,И.}
\procSuperviserOrganization{МАОУ <<Лицей (эколого-биологический)>>}


\makeProcTitleSchool

\index{i@Иванова~Е.}


С развитием промышленности связано развитие и социальной сферы. Человек пытается облегчить свою жизнь при наименьших затратах. В каждой семье для мытья посуды используют специальные средства. Авторы рекламных роликов рассказывают только о достоинствах своего товара. А что происходит с этими средствами в организме человека и~окружающей среде?

Регулярное воздействие средств бытовой химии на окружающую среду объясняет \textbf{актуальность} моей темы. Вопрос о влиянии моющих средств на биологические объекты остаётся открытым. В этом~--- новизна темы. \textbf{Цель исследования}: анализ химического состава и определение наиболее экологичного средства. Исходя из цели, поставлены следующие \textbf{задачи}:

\begin{enumerate}[noitemsep]\vspace{-8pt}
\item анализ литературы [1--11];
\item выявление наиболее используемых марок средств на примере учеников МАОУ <<Лицей (эколого-биологический)>>;
\item раскрытие связи между компонентами, входящими в состав средства, и их влиянием на биологические объекты.
\end{enumerate}\vspace{-8pt}

В соответствии с задачами были использованы как теоретические, так и эмпирические \textbf{методы исследования}:
\begin{enumerate}[noitemsep]\vspace{-8pt}
\item теоретические: анализ, обобщение, классификация;
\item эмпирические: наблюдение, описание, сравнение, анкетирование, эксперимент.
\end{enumerate}\vspace{-8pt}

\textbf{Объектом исследования} является влияние жидких моющих средств для мытья посуды на биологические объекты.

\textbf{Предметом исследования} являются жидкие средства для мытья посуды разных торговых марок.

\textbf{Гипотеза}: производители жидких моющих средств для посуды применяют современные технологии, менее вредные для здоровья человека и окружающей среды.

\textbf{Методика и материалы}: изучение теоретической литературы, проведение анкетирования среди учащихся МАОУ <<Лицей (эколого-биологический)>> по данной проблематике.

\textbf{Практическое применение}: использование материалов и выводов в  научных работах и пособий по данной теме, выявлению глубоких исследований при разработке наиболее экологичного моющего средства, а также при проведении факультативных и элективных курсов по химии.

\textbf{Перспектива исследования}: создание экологичного средства ручной работы на основе детального изучения химического состава и их влияния на живые объекты.

По результатам моего исследования сделаны следующие выводы:
\begin{enumerate}[noitemsep]\vspace{-8pt}
\item наиболее популярным жидким средством для мытья посуды является <<Fairy>>;
\item производители моющих средств ради своей выгоды применяют дополнительные добавки. Но на этикетках упаковок не указывают маркировку, способ применения, не~пишут полную информацию о составе;
\item наиболее дорогим и эффективным средством является <<Mamori>>;\\
\item все исследуемые средства:
\begin{itemize}[noitemsep]
  \item[а)] хорошо растворимы в воде, дают обильную пену, имеют нейтральную среду раствора;
  \item[б)] способствуют усилению коррозии железных предметов;
  \item[в)] негативно влияют на рост и развитие живых организмов.
\end{itemize}
\end{enumerate}\vspace{-8pt}

Гипотеза, что производители жидких моющих средств для посуды применяют современные технологии, менее вредные для здоровья человека и окружающей среды, не~нашла своего подтверждения.

\begin{thebibliography}{99}
\bibitem{}\BibAuthor{Амбрамзон~А.~А., Бочаров~В.~В. и др.} Поверхностно-активные вещества. Синтез, анализ, свойства, применение: учеб. пособие.~--- Л.~: Химия, 1988.~--- 200~с.
\bibitem{}Аналитические рекции хлорид- и фосфат-ионов // Мегаобучалка.~--- 2015.~--- URL: https://megaobuchalka.ru/9/6779.html (дата обращения: 5.12.2019).
\bibitem{}\BibAuthor{Верзейм~Д., Окслейд~К., Ватерхаус~Дж.} Химия. Школьный иллюстрированый справочник.~--- М.\,: Росмэн, 1995.~---127~с.
\bibitem{}Воздействие: хлорид // Большая Энциклопедия Нефти и Газа.~--- 2019.~--- URL: https://www.ngpedia.ru/id533155p1.html (дата обращения: 29.11.2019).
\bibitem{}\BibAuthor{Глинка~Н.~Л.} Общая химия.~--- Л.~: Химия, 1981.~--- 750~с.
\bibitem{}\BibAuthor{Кошель~П.~А.} Большая школьная энциклопедия. Точные науки.~--- М.~: ОЛМА-Пресс, 2002.~--- 416~с.
\bibitem{}Фосфаты и их влияние на человека и окружающую сферу // Водоподготовка. Водоочистка. Монтаж оборудования для подготовки и очистки воды.~--- 2013.~--- URL: http://gostvoda.ru/fosfaty-i-ih-vliyanie-na-cheloveka (дата обращения: 29.11.2019).
\bibitem{}\BibAuthor{Харлампович~Г.~Д. и др.} Многоликая химия.~--- М.~: Просвещение, 1992.~--- 150~с.
\bibitem{}\BibAuthor{Чалмерс~Л.} Химические средства в быту и промышленности.~--- Л.~: Химия, 1969.~--- 528~с.
\bibitem{}\BibAuthor{Шварц~А., Перри~Дж.} Поверхностно-активные вещества: их химия и технические применения.~--- Л.~: Химия, 1981.~--- 304~с.
\bibitem{}\BibAuthor{Шпаусус~З.} Путешествие в мир химии.~--- М.~: Просвещение, 1967.~--- 432~с.
\end{thebibliography}
