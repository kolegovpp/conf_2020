\bigskip
\procTitle{<<Дышим полной грудью, кислород~--- мой друг!>>}

\procAuthor{Немцов~И.}
\procEmail{svetlana.kuznetsova.2012@mail.ru
}
\procSchool{МАОУ <<Гимназия (английская)>>, 5В класс}
\procCity{Магадан}
\procSuperviser{Алтынбаева~Э.\,Р.}
\procSuperviserOrganization{МАОУ <<Гимназия (английская)>>}


\makeProcTitleSchool

\index{n@Немцов~И.}

\textbf{Цель исследования}: изучение различных свойств такого важного газа, как кислород. Кислород~--- необходимый компонент для обеспечения человека энергией жизни.

\textbf{Задачи исследования}:
\begin{enumerate}[noitemsep]\vspace{-8pt}
\item Изучить процесс получения кислорода в~домашних условиях, а также показать, как кислород способствует процессу горения.
\item Определить содержание кислорода в~крови у моих одноклассников и~сопоставить полученные результаты с~режимом двигательной активности.
\end{enumerate}\vspace{-8pt}

\textbf{Объект исследования}: кислород.

\textbf{Предмет исследования}: получение кислорода в~домашних условиях; измерение содержания кислорода в~крови у моих одноклассников с~помощью прибора~--- пульсоксиметра.

\textbf{Актуальность темы}: кислород~--- один из самых распространённых элементов на~Земле. Это газообразное вещество, без запаха и~цвета, поэтому не ощутимо никакими органами чувств. В виде простого вещества этот элемент является вторым по количеству и~первым по значению для жизни составной частью атмосферы.

\textbf{Гипотеза исследования}: концентрация кислорода в~крови увеличивается с~увеличением режима двигательной активности и~временем нахождения на~свежем воздухе.

\textit{Первая часть проекта}~--- это получение кислорода в~домашних условиях из перекиси водорода и~марганцовки, которые я взял аптечке. В колбу я добавил перекись водорода и~марганцовки. При этом выделяется кислород. Опуская тлеющую палочку в~колбу, я увидел усиление огня. Без кислорода огонь гаснет. Отсюда я сделал вывод, что кислород вызывает горение и~может способствовать пожарам.

\textit{Вторая часть проекта}~--- измерение содержания кислорода в~крови у моих одноклассников с~помощью прибора~--- пульсоксиметра. Основные достоинства, благодаря которому пульсоксиметр так распространён~--- возможность постоянно отслеживать уровень кислорода в~крови, получать результаты быстро, не тратя время на~лабораторную диагностику крови, записывать и~сохранять данные на~компьютере. Этот метод контроля кислорода в~крови гарантирует точность исследований!

У всех моих одноклассников этот параметр в~норме, с~тенденцией увеличения у тех, кто больше времени занимается спортом и~больше времени проводит на~свежем воздухе.

Изучив литературу, я выяснил, что природные условия Крайнего Севера признаны экстремальными (низкие температуры, резкие перепады погоды, загрязнение окружающей среды, геомагнитная обстановка, недостаток витаминов и~важных для жизнедеятельности организма микроэлементов) в~результате воздействия климатических, геофизических и~космических факторов, которые могут приводить к появлению стрессовых состояний организма человека. Все это вызывает в~организме человека напряжение, в~результате чего ряд органов и~систем функционируют на~пределе своих возможностей. Стресс сопровождается развитием гипоксических состояний. В результате обследования студентов, которое включало определение насыщение крови кислородом у молодых людей, проживающих в~условиях Севера, были выявлены основные факторы риска гипоксии:
\begin{itemize}[noitemsep]\vspace{-8pt}
  \item курение, в~том числе пассивное курение;
  \item наследственная предрасположенность к заболеваниям дыхательной системы;
  \item хронические заболевания организма;
  \item низкая двигательная активность;
  \item небольшой северный стаж (менее 15 лет).
\end{itemize}\vspace{-8pt}

Были сделаны следующие \textbf{выводы}:
\begin{enumerate}[noitemsep]\vspace{-8pt}
\item Кислород можно получить в~домашних условиях, и~он способствует горению.
\item Содержание кислорода в~крови у моих одноклассников находится в~пределах нормы и~это говорит о том, что все ребята занимаются спортом и~достаточно времени проводят на~свежем воздухе. Что подтверждает гипотезу исследования.
\end{enumerate}\vspace{-8pt}

Для того, чтобы не испытывать недостаток кислорода и~насытить свой организм кислородом, необходимо вести активный образ жизни и~находить время для прогулок и~занятий спортом.

Среди методов насыщения крови кислородом существуют такие простые и~вполне доступные каждому: физические упражнения, дыхательная гимнастика, прогулки на~свежем воздухе. Причём во время физической активности насыщаться будет и~головной мозг, а это способствует улучшению памяти, работоспособности и~сообразительности. При таком режиме жизни не только организм будет насыщаться кислородом, но и~улучшится настроение и~общее самочувствие человека. Желаю всем отличного здоровья!
