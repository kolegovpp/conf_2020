\bigskip
\procTitle{Время с точки зрения физики}

\procAuthor{Маковецкий~Д.}
\procEmail{www.gudeeva.ru@mail.ru}
\procSchool{МБОУ СОШ п.~Омсукчан, 10 класс}
\procCity{пос.\,Омсукчан}
\procSuperviser{Шадаева~О.~В., Гудеева~С.\,А.}
\procSuperviserOrganization{МБОУ СОШ п.~Омсукчан}


\makeProcTitleSchool

\index{m@Маковецкий~Д.}

В ходе исследовательской деятельности по физике мы рассмотрели имеющиеся теории о времени. Выяснили, что время~--- это та самая единица измерения, которая отсчитывает, сколько ты уже прожил. Человек не в силах вернуть вчерашний день или год, но он может не тратить его понапрасну.

Изучили принципы работы времени и определили, что окружающий нас мир существует во времени. Время необратимо. Оно не материально. Это самый дефицитный и самый ценный ресурс, которым каждый располагает, как хочет.

\textbf{Цель исследовательской} деятельности: узнать и разобраться, что такое время с точки зрения физики.

\textbf{Задачи}:
\begin{itemize}[noitemsep]\vspace{-6pt}
\item рассмотреть роль времени в процессах, которые происходят вокруг и внутри нас;
\item изучить время с точки зрения пространственно-временной среды;
\item проанализировать сущность и теории, посвящённые времени, с точки зрения физики.
\end{itemize}\vspace{-6pt}

\textbf{Методы исследования}: теоретический метод (анализ, измерение); наблюдение; мысленный эксперимент.

\textbf{Практическое использование результатов}: знания, полученные при изучении времени, значительно улучшают понимание устройства мира. Результаты, полученные в данной работе, могут быть использованы для создания новых физически обоснованных моделей времени. Кроме того, отдельные результаты могут быть использованы для тестирования и~модификации, описывающих физические задачи.
