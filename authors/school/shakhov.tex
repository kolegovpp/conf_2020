\procTitle{Невероятный картофель}

\procAuthor{Шахов~П.}
\procEmail{olga.gerashenko.68@mail.ru}
\procSchool{МАОУ <<Лицей (эколого-биологический)>>, 3А класс}
\procCity{Магадан}
\procSuperviser{Антонова~И.\,Ю.}
\procSuperviserOrganization{МАОУ <<Лицей (эколого-биологический)>>}


\makeProcTitleSchool

\index{x@Шахов~п.}

\textbf{Цель исследования}: собрать сведения о картофеле, его пользе и вреде, разнообразных способах использования. Результаты работы представить одноклассникам и их родителям. Пополнить методическую копилку начальной школы для использования на уроках окружающего мира.

\textbf{Актуальность работы} в популяризации нетрадиционных способов применения картофеля. Гипотеза: картофель универсальный продукт.

В теоретической части работы говорится к какому семейству принадлежит картофель, из каких веществ состоит.	Кто завёз картофель в Россию. Картофель по популярности четвёртая продовольственная культура в мире. Он рос в Перу ещё в 8 тысячелетии до нашей эры. Индейцы называли его <<чуньо>>, а клубни <<паппо>>. Испанские конкистадоры завезли картофель в Европу.  Землепашцы поняли, что картофель выращивать гораздо легче, чем пшеницу или овёс. Первое описание картофеля принадлежит испанцу Педро Чиеза де~Леоне.  В Европе у картофеля появилось научное название~--- Солянум туберозум эскулентум (паслён клубненосный).

Были проведены домашние эксперименты.\\
\textbf{ОПЫТ №\,1} Наличие крахмала. \textbf{Вывод}: Картофель состоит из воды и крахмала.\\
\textbf{ОПЫТ №\,2} Картофель и йод. \textbf{Вывод}: Картофель состоит из большого количества крахмала.\\
\textbf{ОПЫТ №\,3} Картофель и перекись водорода. \textbf{Вывод}: В сыром картофеле находится белок каталаза.\enlargethispage{2\baselineskip}\\
\textbf{ОПЫТ №\,4} Зелёная картошка. \textbf{Вывод}: Растения имеют выраженный зелёный цвет.\\
\textbf{ОПЫТ №\,5} Картофельные часы.	\textbf{Вывод}: Картофель~--- химическая батарея, природный  аккумулятор.

Мы привыкли, что картофель~--- один из основных продуктов на кухне. Мало кто знает, что этот овощ можно использовать не только по прямому назначению: для чистки обуви, как пятновыводитель, для лечения простуды и кашля. Его используют в домашней косметологии, при чистке окон, чтобы спасти пересоленную еду. Сок картофеля борется с~бородавками. Отваром чистят серебро. Картофель~--- первая помощь при ожогах. Подробнее с этими способами можно ознакомиться в научной работе и памятке <<Использование картофеля в быту>>

 При работе над темой исследования выяснилось, что картофель состоит из воды и крахмала. В сыром картофеле находится белок каталаза. Под воздействием солнечного света в~клубнях картофеля вырабатывается хлорофилл. Картофель можно использовать как природный  аккумулятор. Средства на основе картофеля полезны для организма: заживляют раны. Есть множество способов использования его в быту. Картофель называется <<вторым хлебом>>. Из него можно приготовить более 2000 блюд. Это универсальный продукт. Действительно~--- невероятный картофель.
