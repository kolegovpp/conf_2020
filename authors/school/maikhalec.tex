\procTitle{Вода и её свойства}

\procAuthor{Михайлец~К.}
\procEmail{olga.gerashenko.68@mail.ru}
\procSchool{МАОУ <<Лицей (эколого-биологический)>>, 10А класс}
\procCity{Магадан}
\procSuperviser{Асанова~Н.\,В.}
\procSuperviserOrganization{МАОУ <<Лицей (эколого-биологический)>>}


\makeProcTitleSchool

\index{m@Михайлец~К.}


\textbf{Введение}

Человек, часто не замечает окружающие его предметы из-за обыденности, а зачастую именно они скрывают много интересного и загадочного. Есть в природе явления, которые восхищают людей, заставляют удивляться – одно из них вода. Учёные обнаружили у неё свойства, которых у неё не должно быть, но которыми она обладает!

\textbf{Гипотеза}

Некоторые жидкости не ладят друг с другом. Возьмём масло и воду в качестве примера, мы можете смешать их вместе и трясти так сильно, как только можем, но они никогда не станут друзьями \dots или станут?

\textbf{Цель проект}: с помощью опыта выявить удивительные свойства воды при взаимодействии с маслом.

\textbf{Задачи}: доказать свойства и изучить их.

В основной части работы говорится о том, что такое вода. О химических и физических свойствах воды.

Вода на Земле может существовать в трёх основных состояниях~--- жидком, газообразном и твёрдом. К основным физическим свойствам воды относят: цвет, запах, вкус, прозрачность, температуру, плотность, сжимаемость, вязкость, радиоактивность и электропроводность.

\textbf{Исследования}

Мне понадобилось для эксперимента: вода, три стакана, растительное масло, красители, шипучие таблетки.

Самое главное, что нам нужно знать о растительном масле и воде~--- это то, что они не смешиваются. Теперь, когда мы об этом знаем, давайте проведём некоторую эксперимент с ними. Берём 3 стакана и наливаем сначала масло, потом воду.	И здесь, мы видим как раз факт, что они не смешиваются. Далее я взяла красители 3-х цветов: красный, синий и зелёный и с помощью шприца начала добавлять окрашенную воду в стаканы. А здесь чётко видно, что окрашенные капли тоже не смешиваются с маслом, а проходят через него,  попадая в воду, растворяются и окрашивают только ее.

Для красоты я использовала ещё и шипучие таблетки, которые превратили воду и масло в стакане в бурлящую лаву.

\textbf{Выводы}:

На опыте я доказала, что вода тяжелее масла и поэтому,  она опускается на дно стакана, а масло легче и всплывает наверх. Даже при явлении с лавой, со временем  снова все встает на свои места.
