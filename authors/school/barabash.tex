\bigskip
\procTitle{Расчёт объема эрозионного выноса материалов в ручье Весёлый при катастрофическом паводке 2014\,г. на основе космических снимков}

\procAuthor{Барабаш~И.}
\procEmail{enggym@mail.ru}
\procSchool{МАОУ <<Гимназия (английская)>>, 10Б класс}
\procCity{Магадан}
\procSuperviser{Колегов~П.\,П.}
\procSuperviserOrganization{СВКНИИ ДВО РАН}


\makeProcTitleSchool

\index{b@Барабаш~И.}


\textbf{Введение}

Катастрофический паводок~--- это опасное редкое метеорологическое явление возникшее в результате интенсивного таяния снега или обильных дождей в короткий промежуток времени, повлёкшее подтопление и разрушение инженерных сооружений, а также гибель населения и животных. Так в середине лета 2014\,г. в г.\,Магадане за несколько суток выпало более 175\,мм осадков, что составило трехмесячную норму для данного периода времени. Данный факт является первым за всю историю (с 1936\,г.) метеорологических наблюдений в городе. В результаты возник мощный (катастрофический) дождевой паводок, который подтопил долинную часть р.\,Магаданки и повлёк разрушение городской инфраструктуры.

В устьевых местах рр. Магаданка, Дукча, руч. Холодный, Весёлый, Кедровый ключ и~др. водотоков возник или сильно видоизменился аллювиальный конус выноса и подводные дельты [1]. Также на крутых и скальных берегах вдоль побережья примагаданья появилось более 30 обвалов и оползней, средней длиной 50--100\,м [1]. Одним из характерных для изучения объектов паводка является руч.\,Весёлый.

\textbf{Цель работы}~--- оценить объём вынесенных пород и площадь затронутой поверхности долины в результате катастрофического паводка 2014\,г. на примере руч.\,Весёлый (бухта Старая весёлая, Охотское море).

\textbf{Характеристика руч.\,Весёлый}

Район исследования расположен в центральной части пол-ва Старицкого (59\dg29’30”\,с.\,ш., 150\dg54’30”\,в.\,д.). Территория имеет среднегорный рельеф с абсолютными высотными отметками 705\,м, водораздельные пространства имеют высоты в среднем 550--650\,м. Высота перевала между бухтами Старая весёлая и Светлая~--- 291 м. Расчленённость рельефа составляет 450--550\,м.

Основным водотоком является руч.\,Весёлый. Его морфометрические параметры следующие, м: длина~--- 7374, ширина в среднем течении 3--4, в верховьях~--- 1--2, в низовье~--- 8--10, среднее значение 5. Падение реки составляет 565~м, уклон ручья~--- 13. Площадь водоносного бассейна 25~км$^2$.

\textbf{Рельеф долины}

Строение долины имеет классический вид, и состоит из: русла, поймы, надпойменной (2,5~м) и 4,5~м террасы. Ширина долины в среднем и нижнем течении варьирует от 50 до~250~м, при среднем значении 80~м. Русло имеет ширину 4--6~м, слабо извилистое, с~небольшим количеством проток. Состав отложений представлен валуно-галечным материалом, в~местах излучин реки, а также на пойме, наблюдаются песчано-иловые фации. Пойменная часть до паводка 2014~г. имела ширину от 10 до 20~м, в широких частях до 40~м; после паводка пойма увеличила свою ширину на 20--50~м.

Надпойменная терраса высотой 2,5~м сложена слабо сортированными валуно-галечными отложениями с песчаным заполнителем.

Терраса 4,5~м уровня расположена в южном борте ручья в его низовье. Отложения террасы схожие с вышеописанными за исключением наличие песчано-иловых линз (старичных?), а также более хорошей их сортировки.

Устьевая часть слабо развита, она представлена небольшим конусом выноса~--- подводной дельтой (до паводка). Также в приустьевой часть руч.~Весёлый происходит воздействие волн во время максимальных приливов и отливов, а также штормов.

Анализ изменений рельефа руч. Веселый

Методика исследования включала в себя дешифрирование космических снимков высокого разрешения (до 2 м/пиксел) отснятых до и после катастрофического паводка, а также полевые наблюдения для замеров высоты террас и описание отложений. Картографические материалы были скачаны в программе SAS.Планета [2] из открытых источников Google и~Bing. Для оконтуривания и подсчёта объёмов выноса пород использовалась геоинформационная система QGIS [3].

Алгоритм работы по подсчёту объёмов выноса:
\begin{enumerate}[noitemsep]\vspace{-8pt}
  \item Загрузить геопривязанные космические снимки сервисов Google и Bing в программу QGIS;
  \item Обрисовать русло ручья, для расчёта его длины. Измерение ширины в разных участках ручья;
  \item Оконтурить бассейн ручья, для расчёта водосборной площади;
  \item Оконтурить полигоны русла ручья до и после паводка.
  \item Рассчитать разницу в площади полигонов;
  \item Рассчитать объем вынесенного материала.
\end{enumerate}\vspace{-8pt}

Для выполнения расчёта объёма выноса использовались следующие параметры:\\
длина ручья~--- 7374 м;\\
площадь русла до паводка~--- 103\,559~м$^2$;\\
площадь русла после паводка~--- 149\,916~м$^2$;\\
средняя высота террасы (высота полигона)~--- 3,25~м;\\
объем выноса~--- 150\,660 м$^3$.

Для оценки результатов мы рассчитали объёмы аккумуляции (формирования) новой дельты (и её подводной части):\\
площадь дельты~--- 51\,478~м$^2$;\\
средняя мощность накопленных осадков~--- 1 м;\\
объем осадков~--- 51\,478 м$^3$.

\textbf{Выводы}

В результате паводка 2014~г. долина руч.~Весёлого подверглась изменениям, что в свою очередь повлияло на геоморфологическую и инженерную обстановку. Так была сэродирована терраса 4,5~м уровня с увеличением пойменной части реки до 50~м. Это вызвало перемытие подъездных путей к мосту, а также разрушение более мелкого моста.

Площадь, затронутая при катастрофическом паводке составила 46\,357~м$^2$. Объём сэродированного материала руч. Весёлым составил 150\,660~м$^3$.


\begin{thebibliography}{99}

\bibitem{}\BibAuthor{Важенин~Б.~П.} Экологические и техносферные последствия экстремальных ливней 2014 г. в~Магадане // Технологии техносферной безопасности.~--- 2015.~--- №~2~(60).~--- С.~263--276.
\bibitem{}SAS.Планета // SASGIS~--- Веб-картография и навигация.~--- 2008.~--- URL: http://www.sasgis.org/sasplaneta (дата обращения: 23.11.2019).
\bibitem{}QGIS. Свободная географическая информационная система с открытым кодом.~--- 2010.~--- URL: https://www.qgis.org/ru/site/index.html (дата обращения: 23.11.2019).
    \end{thebibliography}
