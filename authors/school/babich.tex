\procTitle{Фронтовой путь моего прадедушки}

\procAuthor{Бабич~Д.}
\procEmail{svetlana.kuznetsova.2012@mail.ru
}
\procSchool{МАОУ <<Гимназия (английская)>>, 3Б класс}
\procCity{Магадан}
\procSuperviser{Вронская~О.\,Ю.}
\procSuperviserOrganization{МАОУ <<Гимназия (английская)>>}


\makeProcTitleSchool

\index{b@Бабич~Д.}


\textbf{Цель исследования}: исследовать фронтовой путь прадеда в годы войны и на примере его биографии показать мужество, героизм, любовь к Родине, стойкость советского солдата.

\textbf{Задачи исследования:} 1. узнать в каких войсках служил мой прадедушка во время войны; 2. разыскать информацию об участии прадеда в боевых действиях; 3. отразить на карте боевой путь части, в которой он служил.

\textbf{Объект исследования}: фронтовой путь прадеда.

\textbf{Предмет исследования}: карта боёв, участником которых был мой прадед, исторические справки, наградные листы, фотографии.

\textbf{Актуальность темы}: 2020 год в нашей стране объявлен Годом памяти и славы. Вся страна отмечает 75-летнюю годовщину со дня окончания Великой Отечественной войны. Мужество, героизм и любовь к Родине тех, кто защищал нашу Родину никогда не должны быть забыты. Я решил изучить фронтовой путь моего прадеда, чтобы на примере героизма моего родного человека показать, какой дорогой ценой досталась нам Победа. И чтобы не стиралась память поколений, многие смогли бы восстановить фронтовые пути своих родственников.

\textbf{Гипотеза исследования}: используя сохранившиеся исторические данные, наградные листы, сведения с фронтов, фотографии, можно восстановить фронтовой путь участника Великой Отечественной войны.

Из беседы с дедушкой я узнал, что в 1941 году моему прадеду было всего 19 лет. После 6 месячных курсов в 1942 году он прибыл на фронт. Служить он стал в полку реактивных минометов <<Катюша>>. Мой прадед незаметно пробирался на передовую, вел наблюдение, выявлял огневые точки противника, фиксировал координаты и передавал их командованию.

Чтобы изучить боевой путь войсковой части, где служил мой прадед мы с дедом на сайте Министерства обороны изучили рассекреченные журналы боевых действий и установили, что полк входил в состав Армий прорыва, то есть использовался для прорыва фронта или сильно укреплённых районов.

Боевое крещение полк принял возле города Белёв Тульской области.

Весь 1943 год полк принимал непосредственное участие в освобождении от врага Орловской, Тверской областей, городов Великие Луки, Псков, Великий Новгород, Ленинградской области.

В феврале 1944 года полк переброшен в Подмосковье на переформирование в связи с потерями личного состава и боевой техники.

Однако вскоре его вновь направлен для участия в освобождении Крыма и города Севастополь.

С июля 1944 года полк с боями освобождал нынешнюю территорию Белоруссии, Литвы, и Польши.

Далее полк принимал участие в штурме города-крепости Кёнигсберг, а после его взятия полк был переброшен под Берлин в район города Гюрау.

В этом районе 26 апреля 1945 года они провели последний залп из <<Катюш>> в Великой Отечественной войне.

8 мая 1945 года личный состав полка узнал, что Берлин взят, Германия объявила капитуляцию, война окончена!

За годы войны мой прадедушка был награждён 5 правительственными наградами. Медалью <<За отвагу>>, орденом <<Красной звезды>>, дважды орденом <<Отечественной войны 2~степени>>, медалью <<За взятие Кёнигсберга>>.

Из наградных документов я узнал, что старший лейтенант Увижев Рамазан Саламгериевич являлся во время войны являлся командиром взвода разведки, обеспечивал бесперебойную работу связи. Благодаря сведениям, добытым моим прадедушкой, были спасены тысячи жизней при форсировании советскими войсками реки Одер.

В ходе исследования я достиг тех целей, которые были поставлены, узнал много нового о Великой Отечественной войне, а также родном мне человеке. Мой прадедушка, начавший войну 19 летним юношей, закончил её боевым офицером. Его наградные листы свидетельствуют о том, что мой прадед был смелым и отважным воином.

Мы все в неоплатном долгу перед теми, кто остался на полях сражений, перед теми, кто вернулся, обеспечив нам мирную, спокойную жизнь на Земле. Наш долг~--- помнить о тех суровых днях и героях войны.
