\bigskip
\procTitle{<<Мойте окна, запомните это: окна~--- источник жизни и света>>}

\procAuthor{Фёдорова~А.}
\procEmail{magmakoll@mail.ru}
\procSchool{МАОУ <<СО(РК)Ш №\,2>>, 2Б класс}
\procCity{Магадан}
\procSuperviser{Малюкова~М.\,В.}
\procSuperviserOrganization{МАОУ <<СО(РК)Ш №\,2>>}


\makeProcTitleSchool

\index{f@Фёдорова~А.}

\textbf{Цель работы}: найти альтернативные безопасные средства для мытья окон сделанные своими руками.

\textbf{Гипотеза}: если мы найдем эффективное средство для мытья окон, то можно его безопасно для здоровья использовать дома и обходиться без применения средств бытовой химии.

\textbf{Задачи}:

\begin{itemize}[noitemsep]\vspace{-8pt}
\item изучить литературу по теме исследования;
\item изучить ассортимент средств для мытья стекол в магазинах нашего города;
\item сравнить химический состав средств для мытья окон представленных в магазинах и~изучить их возможное влияние на организм человека;
\item собрать информацию об альтернативных способах мытья окон, позволяющих избежать применение бытовой химии;
\item самостоятельно изготовить средство для мытья окон;
\item разработать рекомендации по использованию народных средств для мытья окон и~первой помощи при отравлении бытовой химией.
\end{itemize}\vspace{-8pt}

\textbf{Методы исследования}:

\begin{enumerate}[noitemsep]\vspace{-8pt}
\item Информационно – познавательный;
\item Эксперимент;
\item Наблюдение;
\item Сравнение;
\item Анализ.
\end{enumerate}\vspace{-8pt}

\textbf{Предмет исследования}:

Влияние веществ, входящих в состав средств бытовой химии для мытья окон, на здоровье человека, альтернативные средства для уборки в быту.

В этом году весь мир борется с новым неизученным заболеванием, уносящим жизни миллионов людей.

Учёные всего мира ищут лекарства и вакцины от <<Covid-19>>, а врачи не уходят с работы, спасая заболевших людей. Каждый человек должен выполнять рекомендации врачей: соблюдать дистанцию и  масочный режим в общественных местах, пользоваться дезинфицирующими средствами и чаще мыть руки. Важно регулярно проводить влажную уборку помещений, чтобы убивать осевшие на поверхностях микробы и бактерии.

Когда я хочу помочь маме в уборке квартиры, она говорит, что магазинные моющие средства содержат вредные для здоровья вещества, но можно дома самим приготовить растворы, которые будут безвредны и просты в использовании. Мама предложила мне посмотреть мультфильмы, в которых очень понятно  рассказано о вредных моющих средствах и о изготовлении безвредных средств в домашних условиях.

Мне захотелось самой приготовить средство для мытья стёкол. На стекле лучше видна грязь, которая плохо пропускает свет и лучи солнца. Результаты чистки стекла  видны сразу после уборки. Поэтому я решила начать эксперимент с изготовления моющих средств для окон.
