\bigskip
\procTitle{Как определить качество мёда в~домашних условиях}

\procAuthor{Радовская~А.}
\procEmail{svetlana.kuznetsova.2012@mail.ru
}
\procSchool{МАОУ <<Гимназия (английская)>>, 2Б класс}
\procCity{Магадан}
\procSuperviser{Алтынбаева~Э.\,Р.}
\procSuperviserOrganization{МАОУ <<Гимназия (английская)>>}


\makeProcTitleSchool

\index{r@Радовская~А.}

\textbf{Цель исследования}: найти и~проверить доступные методы определения качества мёда в~домашних условиях.

\textbf{Задачи исследования}:
\begin{enumerate}[noitemsep]\vspace{-8pt}
\item  Найти на просторах интернета, литературе наиболее популярные методы определения качества мёда, подходящие для использования дома;
\item  Купить в~магазинах образцы мёда разных производителей и~видов;
\item  Провести эксперименты со всеми образцами, сверить результаты и~сделать выводы об~эффективности.
\end{enumerate}\vspace{-8pt}
В ходе проведения нашего исследования, мы выдвинули \textbf{гипотезы}:
\begin{enumerate}[noitemsep]\vspace{-8pt}
\item  Качество мёда сложно определить по внешнему виду, запаху, внешнему виду, вкусу;
\item  Точное качество можно определить, используя специальные методы.
\end{enumerate}\vspace{-8pt}
\textbf{Объект исследования}: разные виды мёда.

\textbf{Предмет исследования}: доступные методы определения качество мёда, оценка их~эффективности.


\textbf{Способы исследования}:
\begin{enumerate}[noitemsep]\vspace{-8pt}
\item  Изучение литературы.
\item  Проведение экспериментов.
\item  Наблюдение.
\item  Анализ, обработка и~систематизация результатов.
\item  Вывод.
\end{enumerate}\vspace{-8pt}

\textbf{Актуальность темы}. Пчела~--- уникальное явление природы, она дает человеку целебный, питательный мёд, пчелиный яд, воск, полезные прополис, маточное молочко и~цветочную пыльцу, пергу. Самый популярный и~любимый нами продукт пчёл, конечно, мёд.

Каждый покупатель предпочитает купить хороший мёд, без добавления воды и~с ярко выраженными вкусом и~ароматом. Если вы не профессиональных бортник, то определить "на глазок" очень сложно.

Чтобы исключить ошибку нужно провести сложный химический анализ, являющийся надёжным способом для распознавания натурального продукта от подделки. Но такой способ недоступен нам в~обычной жизни.

Все это придаёт практическую важность нашей работы: научиться простым методам определения качества мёда, доступным в~повседневной жизни.

В своей работе я попыталась выбрать и~использовать наиболее простые методы определения качества мёда, которые можно использовать в~домашних условиях: с~помощью йода, салфетки, чёрствого хлеба, уксуса. Провела эксперименты, в~ходе которых исследовала образцы мёда, купленные в~разных торговых магазинах города Магадана, проанализировала и~полученные результаты, сделала выводы об их эффективности и~простоте применения, сравнила результаты с~моими впечатлениями о вкусе после употребления. Подготовила практический совет для тех, кто хочет купить действительно хороший и~полезных продукт.
