\procTitle{К лихенофлоре острова Спафарьева}
\procAuthor{Желудева~Е.\,В.}
\procEmail{elena.zheludeva.88@mail.ru}
\procOrganization{ИБПС ДВО РАН} \procCity{Магадан}

\makeProcTitleRazdel
\index{j@Желудева~Е.\,В.}

Изучение биоразнообразия и оценка его богатства~--- одна из важнейших задач современности. Без знания видового состава лишайников, играющих значительную роль в сложении растительного покрова, невозможно решать проблемы охраны природы и рационального использования растительных ресурсов.

Лишайники относятся к одной из наиболее широко распространённых групп организмов на территории Магаданской области. Они господствуют в напочвенном покрове большинства растительных сообществ. Несмотря на это, лихенофлора Магаданской области изучена весьма слабо и неравномерно. Проводившиеся ранее исследования касались, главным образом, её континентальной горной части (Тенькинский р-н) (Андреев,1978; Журбенко, 2003; Королев, Толпышева, 1980; Котлов, 1995;  Локинская, 1970) и~охватывали, в основном, напочвенные и эпилитные виды. По южной части области имеются лишь отрывочные сведения о лишайниках (Окснер, Блюм, 1971; Пярн, Пааль, 1991; Davydov, Zhurbenko, 2008). В~настоящее время начато планомерное изучение лихенофлоры Северо-Восточного Приохотья и~остров Магаданской области (Желудева, 2012, 2015, 2017, 2019; Макрый, Желудева, 2012).

Остров Спафарьева представляет собой два горных массива, соединенных низким песчано-галечным перешейком шириной от 300 до 600~м и длиной около 1~км. Площадь северного массива составляет около 22~км$^2$, южного~--- 10~км$^2$. Рельеф острова низкогорный: эрозионно-денудационные склоны сочетаются с узкими трогами и долинами ручьёв. Поверхность плато полого наклонена и осложнена останцами и нагорными террасами. Наивысшая точка южного массива~--- г.~Мыс Кактина, 313,5~м н.~у.~м., северного~--- г.~Командора Беринга, 572~м н.~у.~м.

Климат на острове (по данным метеостанций) характеризуется как приморский субарктический, с избыточным увлажнением, холодным летом и~снежной зимой. На о.~Спафарьева были найдены лишь несколько единичных экземпляров лиственницы кустовидной и стланиковой форм. Каменноберезняки на о.~Спафарьева (северный массив) произрастают по приморским склонам северо-северо-западной экспозиции, защищённым горами от ветров южных направлений. На южной части острова каменная берёза так же, как и лиственница, представлена лишь несколькими экземплярами стланиковой или кустовидной формы. Крупные кустарники на о.~Спафарьева так же, как и каменная берёза, предпочитают склоны, защищенные от юго-западных ветров, но распространены они несколько шире и образуют во многих случаях полукольцо, окружающее каменноберезовые рощи с трёх сторон (сверху, справа и слева). Особенностью южного массива о.~Спафарьева являются распространенные на пологих склонах и плато кустарниково-кустарничковые сообщества (0,1--0,5~м высотой) из стелющейся \textit{Duschekia fruticosa} (Rupr.) Pouzar, вересковых кустарничков и других видов, характерных для кустарничковых тундр, а также \textit{Boschniakia rossica} Chamisso et Sehlechtendal. Тундровая растительность занимает вершинные плато на о.~Спафарьева (южн.), верхние части склонов, водоразделы и~гребни гор, переходит на пологие склоны и днища долин. Для тундровой растительности характерно содоминирование кустарничков в различных соотношениях с лишайниками, мхами, осоками и травами. Лишайниково-кустарничковые (включая щебнистые и каменистые) тундры характерны для верхних частей склонов, щебнистых и каменистых водоразделов. Каменистые и щебнистые фрагменты занимают 20--70\,\% поверхности (Хорева, 2003).


\begin{table}[h!]
\caption*{\textbf{Таксономическая структура лихенофлоры класса \textit{Lecanoromycetes} о.~Спафарьева}}
\label{tab:zhelydeva}
\begin{tabular}{lllcc}
   \toprule
Подкласс          & Порядок                  & Семейство                 &  \parbox[c][4em][c]{0.1\textwidth}{ \centering Кол-во родов} & \parbox[c][4em][c]{0.1\textwidth}{ \centering Кол-во видов} \\
 \midrule
Lecanoromycetidae & Lecanorales              & \textit{Cladoniaceae}     & 1                & 19               \\
                  &                          & \textit{Parmeliaceae}    & 11                & 23               \\
                  &                          & \textit{Ramalinaceae}     & 1                & 3                \\
                  &                          & \textit{Stereocaulaceae}  & 1                & 2                \\
                  &                          & \textit{Sphaerophoraceae} & 1                & 2                \\
                  & Peltigerales             & \textit{Nephromataceae}   & 1                & 3                \\
                  &                          & \textit{Peltigeraceae}    & 1                & 8                \\
                  &                          & \textit{Lobariaceae}      & 1                & 1                \\
                  & Rhizocarpales            & \textit{Rhizocarpaceae}   & 1                & 1                \\
                  & Teloschistales           & \textit{Teloschistales}   & 1                & 1                \\
 \midrule
Ostropomycetidae  & Pertusariales            & \textit{Porpidiaceae}     & 1                & 1                \\
                  &                          & \textit{Icmadophilaceae}  & 2                 & 2                \\
                  &                          & \textit{Ochrolechiaceae}  & 1                & 2                \\
 \midrule
Incertae sedis    & Umbilicariales           & \textit{Ophioparmaceae}   & 1                & 1                \\
                  &                          & \textit{Umbilicariaceae}  & 2                & 5                \\
 \midrule
Всего: 2          & 6                        & 15                        & 27               & 74\\
\bottomrule

\end{tabular}
\end{table}


Работа написана по материалам, собранным сотрудниками ИБПС ДВО РАН Сазановой~Н.~А. и Зеленской~Л.~А. на территории острова в 2018~г. Обработка материалов осуществлялась в лаборатории ботаники Института биологических проблем Севера ДВО РАН (ИБПС ДВО РАН, г.~Магадан). Собранный материал (более 100 образцов) определен с использованием стандартных анатомо-морфологических методов, а также цветных реакций, принятых при изучении лишайников, с использованием Определителя лишайников СССР (1971, 1975, 1978), Определителя лишайников России (1996, 1998), а также  монографии и публикации по отдельным таксономическим группам (Домбровская, 1996; Заварзин, 2001; Чабаненко, 2001). Номенклатура и систематическое положение большей части видов выверены по каталогу <<Список лихенофлоры России>> (Урбанавичюс, 2010). Гербарные образцы хранятся в Гербарии ИБПС ДВО РАН (MAG, г.~Магадан).

В результате проведённого исследования выявлено 74~вида лишайников (см. таблицу), относящихся к 27~родам, 15~семействам, 6~порядкам класса Lecanoromycetes. Согласно проведённому анализу, в ряду субстратных групп доминируют эпигеиды – 44 вида. Их разнообразие значительно выше, чем у эпилитов~--- 15~видов и эпифитов~-- 13~видов. Эпиксилы представлены 2~видами и существенно уступают доминирующим группам.

Среди биоморфологических форм преобладают листоватые лишайники~--- 38~видов, кустистые~--- 29~видов, накипные~--- 4~вида и шиловидные~--- 3~вида. На острове Спафарьева отмечен редкий вид \textit{Peltigera scabrosella} Holt.-Hartw. Он обитает на почве среди других лишайников под покровом кедрового стланика на склонах северо-западной экспозиции. Вид, включённый в красную книгу Магаданской области (Желудева, 2019).

\begin{thebibliography}{99}
\bibitem{}\BibAuthor{Андреев М.~П.} Лишайники стационара <<Абориген>> (Тенькинский район, Магаданской области) // Бот. журн.~--- 1978.~--- Т.~63, №~11.~--- С.~1626--1632.

\bibitem{}\BibAuthor{Домбровская А.~В.} Род \textit{Stereocaulon} на территории бывшего СССР. СПб~: Мир и семья-95, 1996.~--- 270~с.

\bibitem{}\BibAuthor{Желудева Е.~В.} Новинки лихенофлоры Магаданской области // Turczaninowia.~--- 2017.~--- Т.~20, №~2.~--- С.~64--74.

\bibitem{}\BibAuthor{Желудева Е.~В.} Лишайники // Красная книга Магаданской области. Редкие и находящиеся под угрозой исчезновения виды животных, растений и грибов. Магадан~: Охотник, 2019.~--- С.~301--312

\bibitem{}\BibAuthor{Желудева Е.~В.} Новые виды лишайников Магаданской области из Северо-Восточного Приохотья // Turczaninowia.~--- 2015.~--- Т.~18. №~4.~--- С.~5--15

\bibitem{}\BibAuthor{Желудева Е.~В.} Первые данные о лишайниках Ямского участка заповедника <<Магаданский>> // Вестник СВНЦ ДВО РАН.~--- 2012.~--- № 3.~--- С.~28--31.

\bibitem{}\BibAuthor{Журбенко М.~П.} Новые и редкие виды лишайников (Lichenes) из республики Саха-Якутия и Магаданской области // Бот. журн.~--- 2003.~--- Т.~88. №~1.~--- С.~111--117.

\bibitem{}\BibAuthor{Заварзин А.~А.} К характеристике рода \textit{Peltigera} на территории России (предварительный список и ключ для определения таксонов) // Тр. Первой Рос. лихенолог. школы (Апатиты, 06–12.08.2000).~--- Петрозаводск~: Карельский науч. центр РАН. 2001.~--- С.~46-65.

\bibitem{}\BibAuthor{Королев Ю.~Б., Толпышева Т.~Ю.} Очерк флоры лишайников стационара <<Контакт>> (Верхнеколымское нагорье) // Новости систематики низших растений.~--- Л.~: Наука, 1980.~--- С.~137--149.

\bibitem{}\BibAuthor{Котлов Ю.~В.} Материалы к лихенофлоре Верхнеколымского нагорья // Новости систематики низших растений.~--- Л.~: Наука. 1995.~--- С.~66--72.

\bibitem{}\BibAuthor{Локинская М.~А.} Наиболее распространенные виды лишайников на Северо-Востоке СССР // Водоросли и грибы Сибири и Дальнего Востока.~--- Новосибирск~: Наука. 1970.~--- С.~233--245.

\bibitem{}\BibAuthor{Макрый Т.В., Желудева Е.~В.} Новые и редкие лишайники для Магаданской области // Turczaninowia.~--- 2012.~--- Т.~15. №~3.~--- С.~40--44.

\bibitem{}\BibAuthor{Окснер А.~Н., Блюм О.~Б.} К флоре лишайников Советского Дальнего Востока. I Сем. \textit{Peltigeraceae} // Новости систематики низших растений.~--- Л.~: Наука. 1971.~--- С.~249--263.

\bibitem{}Определитель лишайников России.~--- СПб.~: Наука, 1996.~--- Вып.~6.~--- 203~с.; 1998.~--- Вып.~7.~--- 166~с.

\bibitem{}Определитель лишайников СССР.~--- Л.~: Наука, 1971.~--- Вып.~1.~--- 412~с.; 1975.~--- Вып.~3.~--- 275~с.; 1978.~--- Вып.~5.~--- 305~с.

\bibitem{}\BibAuthor{Пярн А., Пааль Я.} Список лишайников // Исследования экосистем полуострова Кони (Магаданский заповедник) / А.~Лейто, Р.~Мянд, Т.~Оя, Я.~Паль, Т.~Тальви.~--- Таллин~: Изд-во Ак. Эстонии, 1991.~--- С.~16--18.

\bibitem{}\BibAuthor{Урбанавичюс Г.~П.} Список лихенофлоры России.~--- СПб.~: Наука, 2010.~--- 194~с.

\bibitem{}\BibAuthor{Хорева М.~Г.} Флора островов Северной Охотии.~--- Магадан~: ИБПС ДВО РАН, 2003.~--- 173~с.

\bibitem{}\BibAuthor{Чабаненко С.~И.} Обзор рода \textit{Hypogymnia} Российского Дальнего Востока // Тр. Первой Рос. лихенолог. школы (Апатиты, 06--12.08.2000).~--- Петрозаводск~: Карельский науч. центр РАН, 2001.~--- С.~265--276.

\bibitem{}\BibAuthor{Davydov E.~A., Zhurbenko M.~P.} Contribution to \textit{Umbilicariaceae} (lichenized Ascomycota) studies in Russia. l. Mainly arctic species // Herzogia.~--- 2008.~--- Vol.~21.~--- Р.~157-166.

\end{thebibliography}
\thispagestyle{empty}
