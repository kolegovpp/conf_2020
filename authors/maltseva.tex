\procTitle{Этнографическая коллекция Музея естественной истории СВКНИИ ДВО РАН}
\procAuthor{Мальцева~Н.\,В.}
\procEmail{banshchikova1986@mail.ru}
\procOrganization{СВКНИИ ДВО РАН} \procCity{Магадан}

\makeProcTitle
\index{m@Мальцева~Н.\,В.}

Может ли сейчас, в~постиндустриальную и~глобальную эпоху, быть востребована на~ценностном, технологическом, прикладном уровнях традиционная культура коренных малочисленных народов Севера (КМНС)? Какие шаги необходимо предпринять, чтобы музейный предмет этнографической коллекции не превращался в~мёртвые диковины уходящих культур, эпох и~людей? Попробуем разобраться в~этих вопросах на~примере этнографической коллекции археолого-этнографического зала Музея естественной истории. Она достаточно небольшая, и~её основное пополнение осуществляется за счет научных командировок сотрудниками лаборатории истории института.

Экспонируемый этнографический предметный ряд характеризует различные сферы жизни местных этнических групп. Наиболее полно представлены оленеводческие культуры (сборы к.~и.~н. Л.~Н.~Хаховской и~к.~и.~н. С.~Б.~Слободина), которые ещё существуют в~Магаданской области.

Несмотря на~свою малочисленность, коллекция предметов оленеводческих культур разнообразна. Она представлена предметами быта, элементами одежды, орудиями, детскими игрушками, промысловыми и~ритуальными предметами.

Самым крупным экспонатом, характеризующим эвенский вьючно-вер\-хо\-вой способ передвижения и~сразу же привлекающим внимание посетителя, является эвенская колыбель <<\textit{бэбэ}>> (б/н, длина 71~см, ширина 42~см, высота 52~см, из~сборов С.~Б.~Слободина, Среднеканский район, 2000-е~гг.).

Колыбель изготавливали из~древесины берёзы и~даурской лиственницы. Для придания нужной формы деталям <<\textit{бэбэ}>> их вымачивали несколько дней в~воде. Когда они становились гибкими, выгибали и~закрепляли нужную форму ровдужными ремнями, а потом просушивали. Детские колыбели сшивали ровдужными ремешками, помещали в~ровдужный чехол [3, С.~119--120]. Внутрь <<\textit{бэбэ}>> ничего не стелили. Она использовалась эвенами для перевозки маленьких детей до 5~лет [1, С.~103].

Ещё одним предметом, характеризующим эвенское оленеводство, является подпруга <<\textit{урульты}>>, кожаная эвенская (СВКНИИ МЕИ\,КП-13, длина подпруги 123~см, ширина подпруги 8~см). Она представляет собой сложносоставной предмет, состоящий из~костяной [рог лося] пряжки (длина 6,5~см, ширина 4,3~см, длина пуговицы с~крепёжным ремешком 15,3~см), кожаного ремня и~длинного тонкого кожаного ремешка (длина ремня 208,5~см), подвижно соединённых между собой. Подпруга была изготовлена в~середине XX~в. и~использовалась оленеводами колхоза <<\textit{Победа}>>, практиковавшими вьючно-верховой способ передвижения на~оленях (сбор Л.~Н.~Хаховской). Передана в~дар Домной Родионовной Трифоновой в~1996~г., село Тахтоямск, Ольский район.

Колокольчик из~бронзы на~ездового оленя (б/н, диаметр 7~см, высота 7,5~см) применяют оленеводческие этносы тундровой зоны. Эвены широко используют колокольчики при содержании домашних оленей, поскольку их оленеводческое хозяйство является таёжным и~отличается полувольным содержанием животных без постоянного окарауливания. Таёжный олень менее подвижен по сравнению с~тундровым, поэтому эвены могут оставлять свои стада на~вольном выпасе на~сутки и~даже более. Главными помощниками тунгусского оленевода при выпасе стада являются превосходное знание местности и~повадок своих животных, обязательное использование колокольчиков. Значение колокольчика для сохранности поголовья и~лучшего управления стадом оценивается достаточно высоко. Во-первых, звучание оленьих колокольчиков позволяет пастухам уже издалека определить местонахождение стада и~облегчает поиск отбившихся животных. Во-вторых, ботала отпугивают волков и~других хищников и, кроме того, препятствуют примыканию домашних оленей к~их диким собратьям. В-третьих, колокольчики помогают потерявшимся животным найти своё стадо.

Строгого учёта поголовья в~культуре оленеводов не существовало. Владельцы стад знали примерную численность своих оленей. Новорожденных телят отмечали зарубками на~специальной деревянной палочке: самочек и~самцов по этим зарубкам считали отдельно [3, С.~92]. В витрине представлен эвенский счётчик для учёта новорожденных телят <<\textit{илкэн}>> (б/н, длина 31~см, ширина 1~см, высота 0,8~см). Получен в~дар от Христины Ивановны Баковой (с.~Гижига, Северо-Эвенский район).

Транспортное средство оленеводов палеоазиатской группы представлено нартами. Нарты использовались чукчами и~коряками во время перекочевок. Они были грузовыми и~ездовыми. В зале представлены грузовые нарты 7-копыльные (б/н, длина 228~см, ширина 57,5~см, высота 25,3~см, из~сбора С.~Б.~Слободина, Среднеканский район, 2000-е~гг.). Грузовые нарты, в~отличие от ездовых, в~своей конструкции не имели спинки. Детали нарт изготавливались из~дерева (берёза или даурская лиственница), и~собирались как конструктор, вставляясь друг в~друга, а крепежами служили кожаные ремни из~шкур морского зверя [лахтака]. Гвозди для крепления деревянных деталей не использовались, поскольку в~условиях постоянной влажности они начинали ржаветь и~гнить, и, следовательно, портили дерево.

Стада оленей у чукчей и~коряков были многочисленны. Олени отличались слабой приручённостью. Для приманивания ездовых оленей оленевод использовал кожаный сосуд <<\textit{гычек'инын}>> (СВКНИИ МЕИ КП-11, длина с~петлей 19~см, длина (по контуру) 48,6~см, диаметр от 6,5~до 14,6~см, высота 8~см, из~сбора П.~П.~Павлова). Пастух наполнял сосуд мочой и~издавал особый горловой звук-призыв для оленя. Чукотский сосуд для жидкости с~петлей изготовлен из~кожи животного (морзверя или толстой кожи старого оленя) и~сшит жильными нитями стежком <<\textit{через край}>>, которые при намокании разбухали и~закупоривали отверстия от проколов в~коже.

Предметный ряд мастериц представлен инструментарием для обработки шкур оленя, коробками, наперстками, сумками, предметами одежды (рукавицы, перчатки, накладки на~тыльную сторону ладони) и~др.

Эвенские инструменты для обработки шкур оленя в~витрине представлены скобелем <<ө>> (СВКНИИ КП-32 Э-21) для соскабливания мездры со~шкуры животного. К деревянной рукояти (28,5~см) скобеля <<ө>> всадным способом прикреплён железный круглый наконечник диаметром 4,7~см с~отверстием и~остро отточенным лезвием.

Корякский скребок для выделки кожи называется <<\textit{авыт}>> (СВКНИИ КП-17, длина 62~см, длина по выгнутой части рукояти 63,5~см, ширина 5,3~см, ширина с~вкладышем 8,4~см, высота 3,3~см), изготовлен в~середине XX~в. Представляет собой копылообразную (или в~виде коромысла) рукоять из~дерева с~каменным вкладышем (наконечником), который располагался посередине рукояти. Скребок получен в~дар от Ирины Гергольтаговны Кэчгельхут в~2002~г. (из сбора Л.~Н.~Хаховской).

Также в~витрине расположен халцедоновый вкладыш (СВКНИИ КП-36, длина 5~см, ширина 4,2~см) с~заостренным обработанном рабочем краем для скребка <<\textit{авыт}>>. Получен в~дар Ирины Гергольтаговны Кэчгельхут в~2002~г. (из сбора Л.~Н.~Хаховской). Обработан в~середине XX~в.
\clearpage
Шкурки мелких промысловых животных~--- белки, горностая~--- просушивали на~правилах клиновидной формы. Доска для сушки белок, горностая, бытовала у эвенов (СВКНИИ КП-34 Э-25, длина 42,4~см, ширина 3,7~см, высота 0,4~см) получена в~дар от Христины Ивановны Баковой (с. Гижига, Северо-Эвенский район). Деревянные правила изготовлены из~реек лиственных пород (берёза), на~которых не допускалось появление~смолы. Шкурку с~белки снимали трубкой с~разрезом по огузку (меху или кожи с~задней части животного), обрабатывали и~натягивали на~правила.

Коробка для хранения предметов женского рукоделия <<\textit{коробдя}>> (СВКНИИ МЕИ КП-19/1, длина 34,8~см, ширина 13,2~см, высота 3,5~см, МЕИ СВКНИИ КП-19/2, длина крышки 32,5~см, ширина 0,5~см, высота 0,5~см, из~сбора Л. Н. Хаховской). Она изготовлена из~дерева в~виде пенала для хранения церковных восковых свечей. Верхняя крышка пенала сдвигается. Предмет передан в~дар Христиной Ивановной Баковой в~2001~г. (с.~Гижига, Северо-Эвенский район, время изготовления~--- середина XX~в.).

Мастерицами для пошива одежды использовались напёрстки. В витрине представлены два напёрстка <<\textit{велывел}>> корякских, бочкообразной формы, без макушки. Рабочей поверхностью служили боковые стороны напёрстка. Изготовлены во второй половине XX в., один из~рога оленя (б/н, высота 2,4~см, диаметр 1,8~см), второй~--- латунь (материалом послужил срез сантехнической трубы, высота 2~см, диаметр 1,8~см). Получены в~дар от Ирины Гергольтаговны Кэчгельхут в~2002~г. (из сбора Л.~Н.~Хаховской).

В витрине представлены работы эвенских мастериц~--- ровдужные перчатки (МЕИ СВКНИИ КП-22/1 Э-22/1 длина 26~см, ширина 15,5~см, МЭИ СВКНИИ КП-22/2 Э-22/2, длина 24,5~см, ширина 11,5~см), летние варежки (СВКНИИ КП-33/1, длина 24,5~см, ширина 16,5~см, СВКНИИ КП-33/2, длина 24,5~см, ширина 16~см) и~эвенская сумочка для хранения огнива и~кремня <<\textit{хилтэк}>> (СВКНИИ МЕИ КП-4, диаметр 11,5~см, общая длина с~подвесками и~<<\textit{ремешком}>> из~низки бусин 22~см), расшитые бисером. Коряки не носили перчатки, только меховые рукавицы в~зимнее время. Переданы в~дар от Христины Ивановны Баковой (с. Гижига, изготовлены во второй половине XX в.). Из сбора Л.~Н.~Хаховской.

В дар от Христины Ивановны Баковой в~2002~г. в~музей поступили две эвенские сумочки~--- футляр для хранения ложек (МЕИ СВКНИИ КП-23 Э-14, длина 28,5~см, ширина 12,7~см, высота 2~см) и~сумка для рукоделия <<\textit{авсы}>> (СВКНИИ КП-41, Э-28, длина 24,5~см, ширина 20,5~см, высота 2,5~см). Сумки изготовлены одной мастерицей во второй половине XX в. Футляр для ложек выполнен из~шкуры оленя белого и~серого цветов, спереди украшен цветочным орнаментом, который не характерен для эвенов. Скорее всего, он был воспринят ими у камчадалов. Сумка для рукоделия <<\textit{авсы}>> изготовлена из~шкуры оленя серого цвета, спереди имеет состыкованный шахматный орнамент, воспринятый у коряков. По верхнему краю сумка оторочена красной тканью. К каждой сумке для подвешивания на~ровдужный ремешок прикреплён латунный крючок, изготовленный из~одной цельной детали.

Традиция жевать табак у оленеводческих народов практикуется достаточно давно. Поскольку ребёнок в~возрасте 5 лет был вовлечён в~хозяйственную деятельность семьи, то~и~практика употребления табака становилась ему доступной с~малолетства. Так, человек всю свою жизнь употреблял табак, и~даже после~смерти, коробки для хранения табака входили в~него погребальный инвентарь. В витрине археолого-этнографического зала представлены две коробки для хранения табака~--- из~рога барана и~бересты.

Коробка для хранения табака <<\textit{ялепючгын}>> корякская (СВКНИИ КП-35 Э-26, длина 9,5~см, ширина 3,3~см, высота 6,3~см, длина по крышке 11,2~см). Основу коробки составляет рог барана. Мастер отпиливал рог высотой 6--7~см, вынимал внутренности, прочищал, просушивал. Дно коробки изготовлено из~деревянной дощечки, сверху прикреплена металлическая накладка-крышка на~петлях. Предмет имеет архаичный вид. Время изготовления~--- середина XX в. В табакерке хранили жевательный табак. После~смерти владельца она обшивалась и~входила в~похоронный инвентарь. Предмет получен в~дар от Ирины Гергольтаговны Кэчгельхут в~2002~г. (из сбора Л.~Н.~Хаховской).

Коробка для хранения табака из~бересты [возможно, якутская или корякская] (СВКНИИ МЕИ КП-3/1, длина 7,7~см, ширина 4~см, высота 12,6~см, СВКНИИ МЕИ КП-3/2, длина крышки 7,5~см, ширина 2,7~см, высота крышки с~ручкой 5~см). Искусство выделки бересты и~характерный зубчатый замок по середине коробки на~всю длину указывает на~мастерство якутов. Коробка изготовлена в~пос. Сеймчан в~середине XX в. Использовался эвенами.

Ответственным делом взрослого мужчины-оленевода считалось изготовление арканов~--- <<\textit{маутов}>>. В музее хранятся два аркана (фрагменты), б/н. Один изготовлен из~кожи оленя (общая длина 69,6~см, ширина 1,5~см), на~конце прикреплена пуговица из~металла (длина 3,8~см, ширина 2,6~см). Для этой цели использовали весенние линные шкуры взрослых самцов-оленей, а также лосей и~диких оленей. Сырую шкуру укладывали, растянув на~земле, мездрой вверх, подкладывали под нее доску и~вырезали по спирали тонкие и~длинные ремни шириной не более 5~мм, двигая острым мужским ножом вокруг шкуры, начиная с~краёв, и~далее~--- к~центру, середине. Срезанные полоски ремней увлажняли водой и~вытягивали руками. После этого их сушили, туго натянув по стволам деревьев, а затем снимали, выскабливали с~них шерсть скребком <<\textit{кочай}>> (эвенск.). Высушенные ремни разминали руками, мяли, сплетали из~них аркан~--- маут в~три или четыре ряда, необходимой длины, после чего мазали подогретым оленьим салом и~дымили, сложив бунтом. Получался очень крепкий маут, который не воспринимал воды и~сырости [3, С.~120]. Второй изготовлен из~кожи морзверя [тюлень, лахтак] (длина 461,2~см, ширина (диаметр) 1,1~см). Он сшит из~кожаного ремня и~вывернут наизнанку швом внутрь изделия, через которое протянута скрученная из~трёх нитей синтетическая верёвка.

В витрине представлены и~детские игрушки. Игрушка-головоломка эвенская (СВКНИИ МЕИ КП-16, длина дощечки 10,8~см, ширина 2,6~см, высота 0,5~см, длина подвесных элементов 18~см)~--- модель, изготовлена Игорем Алексеевичем Давыдовым в~1990-х~гг.,~г. Магадан. Традиционная эвенская игра <<\textit{Угадай}>> в~виде дощечки с~верёвочками, подвижной петлей и~двумя бусинами, рассчитанная на~детей от трёх до десяти лет. Задача игрока в~том, чтобы перевести обе бусины сквозь подвижную петлю на~другую дугу нити.

Игрушка марионетка эвенская (СВКНИИ МЕИ\,КП-20 Э-18, длина 32~см, длина куклы с~поднятыми руками 18,4~см, длина туловища 9,5~см, длина руки 7,4~см, длина ноги 8,5~см). Эвенская игрушка марионетка изготовлена Василием Филипповичем Александровым из~даурской лиственницы, как наиболее распространённого деревянного материала, в~конце 1990-х~гг., пос. Ола. Мастер был переселенцем из~пос. Озерное (Срднеканский район, совход <<\textit{Челбанья}>>). Кукла с~подвижными руками и~ногами, выдержанными пропорциями тела и~лица. На голове куклы изображено (вырезано) лицо с~глазами, ушами, ртом, носом. Марионетка сделана по образцу игрушек фабричного производства, привязана за руки кручёной верёвочкой к~каркасной основе в~виде турника. Игрушка приводится в~движение, то~есть осуществляет кувырок через голову~--- при сведении друг к~другу нижних частей палочек каркасной основы~--- турника (из сбора Л.~Н.~Хаховской).

В витрине также представлен предмет из~драгоценного металла. Главным поставщиком ювелирных изделий~--- серебряных цепочек с~христианскими крестами, серег, браслетов и~поясов~--- до начала XX в. являлись якуты [3, С.~117]. Фрагмент наборного женского эвенского пояса <<\textit{боят}>> (СВКНИИ МЕИ КП-5) изготовлен из~низкопробного серебра. Музейный предмет представлен фрагментом из~фигурной пластины (длина 7,7~см, ширина 4,4~см), расположенной возле пряжки, и~замшевой основы (длина 9,5~см, ширина 6~см) с~двумя металлическими прямоугольными пластинами (длина каждой пластины 5~см, ширина 4,6~см), прикреплёнными к~основе ровдужными завязками-ремешками и~заклёпочным стержнем. Пластины украшены набивным геометрическим рисунком (сбор \enlargethispage{2\baselineskip}С.~Б.~Слободина).
\clearpage
Ритуальные предметы коренных малочисленных народов Севера на~выставке представлены двумя огнивными досками, шаманской шапочкой и~колотушкой для бубна.

Несмотря на~то, что предметы относятся к~разным этносам~--- чукчам, корякам, эвенам, они представляют собой комплекс, характеризующий аспекты духовной культуры коренных малочисленных народов Севера.

Прибор для возжигания огня чукотский (СВКНИИ МЕИ КП-7, длина 40,5~см, ширина~--- 7,6~см, диаметр углублений от1,5 до 2~см, 16 круглых ямок-углублений; из~сборов Н.~Н.~Дикова, 1963~г.). <<\textit{Судя по надписи на~экспонате, он найден в~1963~г. вблизи селения Нутепельмен, на~восточном берегу лаг. Пынгопильгын, побережье Чукотского моря. Прибор изготовлен из~тополя, от времени древесина растрескалась и~приобрела светло-серый цвет. Экспонат в~виде бруска имеет антропоморфный облик, так как его верхняя часть оформлена в~виде овальной <<\textit{головы}>>, которая переходит в~перемычку-шею и~<<\textit{туловище}>> удлиненной, зауженной к~нижнему концу формы. Руки и~ноги никак не оформлены, черты лица не обозначены. На основной части прибора имеется 16 круглых ямок-углублений, расположенных в~три продольных ряда. В среднем ряду 8 ямок, в~левом~--- 5, правом~--- 3}>> [4, С.~280].

Прибор для возжигания огня чукотский (СВКНИИ МЕИ КП-1, длина 40~см, ширина 7,6~см, диаметр углублений от 2,2 до 2,7~см, 7 круглых ямок-углублений; из~сборов П.~П.~Павлова, 1972~г.). <<\textit{Найден в~1972~г. на~Чукотке, на~месте покинутого стойбища в~междуречье Большого и~Малого Пыкарваама (притоки р. Белая, Анадырский район). Экспонат изготовлен из~древесины берёзы в~виде удлинённого бруска, на~котором двумя боковыми выемками у верхнего конца оформлены плечики, шея и~овальная голова. На огнивной доске обозначены черты лица: две точечные ямки~--- глаза и~прямой короткий желобок~--- рот. На туловище прибора расположены два продольных ряда углублений: слева~--- 4, справа~--- 3}>> [4, С.~280].

С помощью огнивной доски и~лучкового сверла, который вращали в~ямке-углублении, куда клали сухой мох или вываренный и~высушенный древесный гриб, добывали ритуальный огонь.

Шапочка-капор меховая корякская (шаманская) <<\textit{ЯяйиткОпэн Кэн}>> (СВКНИИ КП-12, длина 24~см, глубина 8,5~см, ширина 20,5~см, мех, кожа; из~сборов П.~П.~Павлова). Основной элемент шаманской одежды. Шапка в~виде капора с~выраженными наушниками, которые выдаются вперёд, со стилизованным намеченным отверстием в~виде розетки на~макушке и~обозначенными кусочками меха местами для ушей. Шапка обшита мехом собаки чёрного цвета.

<<\textit{Капор сшит из~оленьего меха со стриженой шерстью волосом внутрь. Состоит из~двух деталей: широкой прямоугольной полосы, проходящей поперек головы, и~овальной, расположенной на~затылке. Внешняя, мездровая сторона шапочки окрашена настоем ольховой коры в~коричневый цвет. Шапочка опушена по верхнему и~нижнему краям двойными полосками меха: верхней край оторочен с~лицевой стороны чёрным собачьим мехом, с~изнаночной~--- оленьим; нижний край оторочен двойным мехом собаки чёрного цвета. Все детали сшиты жильной нитью стежком <<\textit{через край}>>. Размеры капора: окружность поперек головы 44~см, вдоль головы (без оторочки)~--- 33~см, ширина верхней оторочки 5~см, нижней~--- 1,5. Имеются ремешки-вязки длиной 299~см. На шапочке мало украшений~--- на~боковых сторонах лишь две кисточки из~клочков белого собачьего меха и~на~макушке ложная розетка~--- узкими белыми полосками вышиты три концентрические окружности, две внутренние выполнены из~мандарки, внешняя~--- из~белого оленьего меха. Диаметр розетки 5,5~см. Участок кожи внутри розетки закрашен [сажей или кровью]}>> [5, С.~280].
\clearpage
Колотушка для бубна эвенская~--- <<\textit{гусун}>> (из сборов Л.~Н.~Хаховской, Северо-Эвенский район, с. Гижига, 2001~г.). Получена в~дар от эвенки Христины Ивановны Баковой. <<\textit{Основу колотушки составляет тонкая пластина китового уса, наглухо обшитая мехом и~тканью. Одна сторона рабочей части колотушки (без рукояти) из~меха соболя, другая наборная~--- из~чередующихся узких поперечных полосок коричневого и~белого стриженого меха, лоскутов красной ткани и~трёх бисерных розеток. Ткань и~бисер нашиты на~ровдугу. Рукоять обтянута синей шерстяной тканью, к~её краю прикреплена наборная кисть из~окрашенного в~оранжевый цвет меха морзверя}>>. Длина колотушки 37~см, ширина 3~см, длина кисти 26~см, б/н [4, С.~286--287].

В 2017~г. в~Музей естественной истории СВКНИИ ДВО РАН поступил жирник, найденный в~1975~г. на~одной из~чукотских стоянок арктического побережья. Жирник изготовлен из~вулканита, продольной формы (одна стороны прямая, другая овальная) с~перегородкой, разделённой на~две части. Топливо поступало к~фитилю по трём каналам. Фитиль располагался с~внешней стороны перегородки вдоль края. Высота его~--- 105,~см, наибольшая длина~--- 27,5~см [3, С.~246].

Итак, несмотря на~свою малочисленность предметного ряда, этнографическая коллекция Музея естественной истории СВКНИИ ДВО РАН достаточно подробно иллюстрирует бытовые условия проживания коренных жителей нашего региона, позволяет, выделить основные характерные черты типов хозяйствования и~природопользования этносов.

\begin{thebibliography}{99}
\bibitem{}История и~культура эвенов. Историко-этнографические очерки.~--- СПБ.~: Наука, 1997.~--- 180~с.
\bibitem{}\BibAuthor{Лебединцев~А.~И.} Жирник с~Чукотки: культурно-историческое атрибутирование // Чтения памяти академика К.~В.~Симакова~: материалы докладов Всероссийской научной конференции (22--24 ноября 2017~г.).~--- Магадан~: СВКНИИ ДВО РАН, 2017.~--- С.~246--250.
\bibitem{}\BibAuthor{Попова~У.~Г.} Эвены Магаданской области.~--- М.~: Наука, 1981.~--- 304~с.
\bibitem{}\BibAuthor{Хаховская~Л.~Н., Павлов~П.~П.} Ритуальные предметы в~музейной коллекции СВКНИИ // II~Диковские чтения~: материалы научно-практич. конф., посвящ. 70-летию Дальстроя.~--- Магадан~: СВКНИИ ДВО РАН, 2002.~--- С.~280--288.

\end{thebibliography}
