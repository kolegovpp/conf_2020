\vspace{-16pt}
\procTitle{Результаты кариотип диагностики плаценты и~хориона}

\procAuthorI{Абдурашидова~М.\,Б., Пак~И.\,В.}
\procEmailI{muniraxon.91@mail.ru, pakiv57@mail.ru}
\procOrganizationI{ФГАОУ ВО ТюмГУ}
\procCityI{Тюмень}

\procAuthorII{Винокурова~Е.\,А.}
\procEmailII{vinokurovaelena@mail.ru}
\procOrganizationII{ФГБОУ ВО Тюменский~ГМУ Минздрава~России}
\procCityII{Тюмень}

\makeProcTitleIIRazdel
\index{a@Абдурашидова~М.\,Б}
\index{p@Пак~И.\,В.}
\index{v@Винокурова~Е.\,А.}

\textbf{Введение}. Пренатальная диагностика за последние десятилетия добилась больших успехов в~профилактике наследственных заболеваний и~пороков развития у~плода, что способствовало уменьшению количества мертворождённых детей и~рождений детей с патологиями развития [5,9]. Развитие и~внедрение новых инновационных методов в~пренатальную инвазивную диагностику (ИПД) является одним из важных изменений в~структуре и~в~методике исследования. С помощью инвазивных методов исследования, в~основе которых лежит кариотипирование, можно прогнозировать проявление патологий со 100\,\%-ной точностью [1,7]. При этом снижение летального риска для плода является одним из основных задач в~ИПД [3]. Пренатальная диагностика (ПД) даёт возможность определить хромосомные аномалии и~пороки развития в~ранние сроки беременности, что позволяет сделать выбор беременным женщинам: прервать беременность или пролонгировать беременность [4,10-11]. При пренатальной диагностике неоценима роль кариотипирования. Кариотипирование является цитогенетическим методом диагностики хромосомных аномалий, позволяющим определить отклонения в~числе, а также в~структуре хромосом [2,5]. Нарушения в~кариотипе могут привести к наследственной патологии и~рождению детей с аномалиями развития, а также могут быть причиной бесплодия или возникновения абортусов [6,8]. Кариотипирование исследует количество хромосом, их форму, размеры, а также особенности строения хромосом. Проблема кариотип диагностики всегда была актуальной проблемой не только в~области здравоохранения, затрагивая аспекты генетики.

\textbf{Цель работы}. Изучить применение метода кариотипирования с помощью методов ИПД биопсии плаценты и~хориона у~беременных женщин в~возрасте от~16 до~46~лет.

\textbf{Материалы и~методы исследования}. Исследования были проведены на базе Государственного бюджетного учреждения здравоохранения Тюменской области <<Перинатальный центр>> г.~Тюмени. Объектом исследования являлся биоптат на основе абортного материала, который поступил в~цитогенетическую лабораторию Перинатального центра в~2019--2020~гг. Приготовление препаратов из биологического материала, взятого с помощью биопсии хориона и~плаценты пациента, включает в~себя несколько этапов: получение материала из ворсин хориона и~плаценты; фиксация препаратов; приготовление препарата; окрашивание препарата. Просматривали препарат под микроскопом, фиксируя хромосомные нарушения. Исходя из цели нашего исследования, нами было отобраны результаты изучения данных кариотипирования 544~плодов. Биологические материалы для исследования были получены путём биопсии хориона и~плацентоцентеза, от женщин в~возрасте от~16 до~46~лет (в среднем 32\rpm2,8~лет).  Исследования проводились с учётом этических норм, в~рамках научного исследования совместно с <<Перинатальным центром>> города Тюмени. Статистическую обработку данных проводили с использованием пакета прикладных программ MS~Excel.

\textbf{Результаты и~обсуждение}. Пренатальная диагностика в~настоящее время позволяет определить на любом сроке беременности наличие хромосомных аномалий, без вреда для здоровья матери и~ребёнка. Инвазивный метод диагностики основан на прямом взаимодействии с плодом (проникновением), то есть сбор материала для анализа осуществляется от ребёнка из хориона или же из ворсин. В~зависимости от срока беременности используются разные методы.

В ходе нашего исследования, показаниями к проведению цитогенетической перинатальной диагностики являлись: возраст женщины старше 35~лет; наличие в~семье предыдущего ребёнка с хромосомной аномалией; при прохождении скрининга обнаружение у~одного из родителей присутствие сбалансированной хромосомной перестройки, и/или отклонения от~нормы по биохимическим маркёрам; облучение одного из родителей до~зачатия; перенесение женщиной вирусных инфекции, такие как гепатит, краснуха, токсоплазмоз в~ранних сроках беременности.

Всего в~<<Перинатальном центре>> г.~Тюмени за период исследования, было проведено 629 исследований плода цитогенетическим методом, из них путём кариотипирования биологического материала, полученного методом биопсии хориона~--- 98~шт.; плацентоцентез~--- 446~шт.; кордоцентез~--- 85~шт.; абортивный материал~--- 70~проб. Из них 544~биоматериалов было отобрано для данного исследования.

По результатам первоначального анализа, возраст матерей составил от~16 до~46~лет (в среднем 32\rpm2,8~лет).

В ходе исследования определили, что у~85,5\,\% (n=465) пациентов плод имеет нормальный кариотип (46, ХХ/ХУ). У остальных 14,5\,\% (n=79) было определено наличие хромосомных аномалий (ХА). В~частности, были выявлены:
\begin{itemize}[noitemsep]\vspace{-8pt}
  \item 55~--- числовые нарушения аутосомных хромосом,
\item  5~--- числовые нарушения половых хромосом,
\item  3~--- полиплоидия,
\item  5~--- структурные аномалии,
\item  11~--- плацентарный мозаицизм.
\end{itemize}
\vspace{-8pt}
При цитогенетическом анализе у~69,6\,\% образцов (n=55) из 79 обследованных плодов с ХА были обнаружены числовые нарушения аутосомных хромосом, из них 43 плода (54,4\,\%) оказались с синдромом Дауна.

У трёх плодов (3,8\,\%) наблюдалась трисомия аутосом; у~двух (2,5\,\%) плодов был отмечен синдром Патау, в~частности, у~одной были диагностированы множественные врождённые пороки развития (МВПР); в~7 (8,9\,\%) случаях~--- синдром Эдвардса.

По результатам исследования, у~групп беременных женщин с патологиями кариотипа плода (изменение числа аутосом), n=55 составляют: 36\,\% женщины в~возрасте от 33 до~37~лет (n=20); 26\,\%~--- в~возрасте от 37 до~41~лет (n=14); 20\,\% (n=11)~--- от 41 до~46~лет; 9\,\% (n=5)~--- от 29 до~33~лет; 7\,\% (n=4)~--- от 21 до~25~лет; 2\,\% (n=1)~--- в~возрасте 27~лет.

Среди исследованных 79 плодов у~5 (6,3\,\%) (n=5) выявлены нарушения в~числе половых хромосом (см. таб.~1).


\begin{table}[h!]
\caption{Показания для проведения цитогенетического анализа плодов (изменение числа половых хромосом, n=5)}
\label{tab:Abduraschidova-1}
\begin{changemargin}{-1cm}{0cm}\vspace{-8pt}
\begin{tabular}{cccc}
\toprule
\parbox[c][][c]{0.15\textwidth}{ \centeringВозраст 			женщины} & \parbox[c][][c]{0.2\textwidth}{ \centeringСрок 			беременности, недели} & \parbox[c][][c]{0.35\textwidth}{ \centeringПоказания 			для проведения цитогенетического 			анализа} & \parbox[c][][c]{0.3\textwidth}{ \centeringХромосомная 			аномалия в кариотипе} \\
\midrule
43                 & 16                           & риск 			1:67 (возраст)                                   & 47,XXX                 \\
38                 & 18-19                        & возраст                                                  & 47,ХХХ 			{[}5{]}                  \\
39                 & 16-17                        & возраст                                                  & 47,XXY                             \\
38                 & 16-17                        & возраст                                                  & 45,X 			{[}6{]}           \\
27                 & 12-13                        & ГКН                                                      & 47,ХYY                  \\

\bottomrule

\end{tabular}
\end{changemargin}
\end{table}


По результатам исследования, среди обследованного материала выявлено два случая синдрома Клайнфельтера у~женщин в~возрасте 39~лет и~27~лет, что составило 2,52\,\%; один случай синдрома Шерешевского-Тернера у~женщины в~возрасте 38~лет (1,26\,\%) ; трисомия по Х-хромосоме наблюдалась в~двух случаях у~женщин в~возрасте 38 и~43~лет (2,52\,\%).

Одновременно было определено, что в~трёх случаях (3,9\,\%) (n=3) была выявлена триплоидия (см. таб.~2).


\begin{table}[h!]
\caption{Показания для проведения цитогенетического анализа плодов (триплоидия, n=3))}
\label{tab:Abduraschidova-2}
\begin{changemargin}{-1cm}{0cm}\vspace{-8pt}
\begin{tabular}{cccc}
\toprule
\parbox[c][][c]{0.15\textwidth}{ \centeringВозраст 			женщины} & \parbox[c][][c]{0.2\textwidth}{ \centeringСрок 			беременности, недели} & \parbox[c][][c]{0.35\textwidth}{ \centeringПоказания 			для проведения цитогенетического 			анализа} & \parbox[c][][c]{0.3\textwidth}{ \centeringХромосомная 			аномалия в кариотипе} \\
\midrule
16                 & 13--14                           &   возраст                                 & 69,XXX                 \\
37                 & 12                        & возраст                                                  & 69,ХХХ                  \\
33                 & 12                        & риск 			1:9                                                  & 69,XXY                             \\


\bottomrule

\end{tabular}
\end{changemargin}
\end{table}


При исследовании 79 биопатов от пяти (6,3\,\%) плодов обнаружены структурные аномалии (внутрихромосомные обмены). В~13,9\,\% случаев наблюдается плацентарный мозаицизм.

Как показали наши исследования, в ходе анализе хромосомных нарушений обнаружено, что с наибольшей частотой встречаются числовые нарушения аутосомных хромосом, из них 54,4\,\% плода (n=43) оказались с синдромом Дауна и с наименьшей частотой была выявлена триплоидия у 3,9\,\% плода (n=3). Следовательно, для выявления аномалий развития плода целесообразно использование на ранних этапах беременности пренатального кариотипирования в дородовом скрининге.

\textbf{Вывод}. Цитогенетический анализ биологического материала от пациентов в~возрасте от 16 до~46~лет, выявило наличие в~14,5\,\% (n=79 из 544) биопатах различных хромосомных аномалии, у~остальных 85.5\,\% (n=465) отмечен плод нормального кариотипа (46, ХХ/ХУ). При этом, провоцируется нарушение аутосомных хромосом~--- 69,6\,\% (n=55).

\begin{thebibliography}{99}
\bibitem{}\BibAuthor{Балакишиева~А.~В., Семерненкова~Е.~В., Лукина~Н.~В.} Роль инвазивных методов пренатальной диагностики в~выявлении хромосомных аномалий уauthors/Abduraschidova.texплода // Смоленский медицинский альманах.~--- 2018.~--- №~4.~--- С.~1--3.

\bibitem{}\BibAuthor{Кузнецова~Т.~В.} Пренатальное кариотипирование~--- методы, проблемы и~перспективы // Журнал акушерства и~женских болезней.~--- 2017.~--- Т.~56, №~1.~--- С.~120--128.

\bibitem{}\BibAuthor{Меренова~С.~В. и~др.} Методы и~значение цитогенетического исследования плодного материала как заключительного этапа пренатальной диагностики / С.~В.~Меренова, C.~В.~Белоусова,  Е.~Н.~Назарова, Ж.~Ж.~Валиуллова, Т.~Б.~Ольшевская // Медицина: вызовы сегодняшнего дня~: материалы IIIauthors/Abduraschidova.texМеждунар. науч. конф. (г.~Москва, январь 2016~г.).~--- Москва~: Буки-Веди, 2016.~--- С.~77--79.

\bibitem{}\BibAuthor{Низаева~Н.~В. и~др.} Морфологические особенности мезенхимальных клеток стромы ворсин хориона / Н.~В.~Низаев, Т.~В.~Сухачёва, Г.~В.~Куликова, М.~Н.~Наговицына, Н.~Е.~Кан, О.~Р.~Баев, С.~В.~Павлович, Р.~А.~Серов, А.~И.~Щёголев, Р.~А.~Полавцева // Вестник РАМН.~--- 2017.~--- Т.~72, №~1.~--- С.~76--83.~--- DOI:~10.15690/vramn767.

\bibitem{}\BibAuthor{Никифоровский~Н.~К., Степанькова~Е.~А., Лукина~Н.~В., Покусаева~В.~Н.} Роль раннего пренатального комбинированного скрининга в~диагностике врожденных аномалий развития у~плода в~Смоленской области // Смоленский медицинский альманах.~--- 2018.~--- №~4.~--- С.~19--23.

\bibitem{}\BibAuthor{Томарева~Е.~И., Меладзе~Р.~Д., Евдокимова~Д.~В.} Факторы риска патологического кариотипа плода // Вестник новых медицинских технологий.~--- 2017.~--- №~2.~--- С.~206--211.

\bibitem{}\BibAuthor{Шубина~К.~А., Шумкова~П.~В.} Пренатальная диагностика // Вестник совета молодых учёных и~специалистов Челябинской области.~--- 2016.~--- Т.~1, №~3.~--- C.~54--59.

\bibitem{}\BibAuthor{Chai~H., DiAdamo~A., Grommisch~B., et al.} A Retrospective Analysis of 10-Year Data Assessed the Diagnostic Accuracy and Efficacy of Cytogenomic Abnormalities in Current Prenatal and Pediatric Settings //  Front Genet.~--- 2019.~--- No~10.~--- Р.~1162.~--- DOI:~10.3389/fgene.2019.01162.

\bibitem{}\BibAuthor{Hay~S.~B., Sahoo~T., Travis~M.~K., et al.} ACOG and SMFM guidelines for prenatal diagnosis: Is karyotyping really sufficient? // Prenat Diagn.~--- 2018.~--- Vol.~38, No~3.~--- Р.~184--189.~--- DOI:~10.1002/pd.5212.

\bibitem{}\BibAuthor{Zhang~B., Lu~B.~Y., Yu~B, et al.} Noninvasive prenatal screening for fetal common sex chromosome aneuploidies from maternal blood // J Int Med Res.~--- 2017.~--- Vol.~45, No~2.~--- Р.~621--630.~--- DOI:~10.1177/0300060517695008.

\bibitem{}\BibAuthor{Zhu~Y., Shan~Q., Zheng~J., et al.} Comparison of Efficiencies of Non-invasive Prenatal Testing, Karyotyping, and Chromosomal Micro-Array for Diagnosing Fetal Chromosomal Anomalies in the Second and Third Trimesters // Front Genet.~--- 2019.~--- No~10.~--- Р.~69.~--- DOI:~10.3389/fgene.2019.00069
\end{thebibliography}
\thispagestyle{empty}
