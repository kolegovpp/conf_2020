
\procTitle{Коллекция чешуекрылых в
фондах Магаданского областного краеведческого музея}

\procAuthorI{Тридрих~Н.\,Н.}
\procEmailI{tridrih\_nik@mail.ru}
\procOrganizationI{ИСИЭЖ СО РАН}
\procCityI{Новосибирск}

\procAuthorII{Кораблёва~Н.\,С.}
\procEmailII{nina-1112@mail.ru}
\procOrganizationII{МОКМ}
\procCityII{Магадан}

\makeProcTitleII
\index{t@Тридрих~Н.\,Н.}
\index{k@Кораблёва~Н.\,С.}


Алексей Петрович Васьковский родился 18 июня 1911\,г. в Ленинграде, умер 16~августа 1979\,г., похоронен на Марчеканском кладбище. В 1930\,г. он поступает учиться на курсы младшего геологического персонала при Институте цветных металлов Главного геолого-разведочного управления \mbox{(ГГРУ)}. С 1931\,г. он выезжает на Колыму в составе II\,Колымской экспедиции под руководством Юрия Александровича Билибина и до 1933\,г. работает старшим коллектором, затем начальником партии. С 1933 по 1937\,г. от Дальстроя работает геологом-петрографом в сводной группе и одновременно учится в Ленинградском горном институте по специальности геолог. С 1937 по 1941\,г. работает в Индигирской геологоразведочной экспедиции, сначала начальником группы рудно-поисковых партий, потом начальником геолого-поискового отдела Индигирского геологоразведочного управления, а~с~1939\,г.~--- главный геолог экспедиции. В этот период работы сделал свое первое открытие россыпного месторождения золота Туора-Тас, на базе которого впоследствии работал один из крупнейших приисков Дальстроя <<Ольчан>>.

Зимой 1942\,г. в блокадном Ленинграде, в очень трудных условиях, сдаёт государственные экзамены в институте и получает диплом первой степени (с~отличием) и выезжает на Колыму. С этого времени работает в геологоразведочном отделе Дальстроя начальником научно-исследовательского отдела. В 1949\,г. Алексей Петрович назначен начальником отдела сводных карт.

В 1972--1979\,гг. заведовал лабораторией ландшафтоведения и охраны природы в Институте биологических проблем Севера.

Будучи профессиональным геологом и учёным с широким кругозором, он занимался исследованиями в различных областях геоморфологии, металлогении, формировании россыпных месторождений полезных ископаемых, стратиграфии и палеонтологии, истории четвертичного периода, как редактор и автор возглавлял работы по созданию обобщающих геологических, геоморфологических и стратиграфических карт Северо-Востока России и~Арктики. Является автором более 80 печатных работ и 18 изданных геологических карт Северо-Востока России.

Другое яркое направление научных интересов и исследований Алексея Петровича Васьковского было связано с изучением живой природы Севера (флоры, фауны, почв), а~также географии ландшафтов. В 1968\,г. Владимир Иванович Горазеев, руководитель областной секции озеленения Магаданского областного отделения Всероссийского общества охраны природы, прислал в~комитет по радиовещанию и телевидению Магаданской области предложение о введении раздела <<Природа родного края>>, при этом советовал пригласить для участия А.\,П.\,Васьковского. В документе, присланном в комитет, он так написал о Васьковском: <<\textit{\mbox{\dotsОн~--- геолог,} но\,хорошо знает нашу растительность и человек с~большим кругозором\dots}>> [8].

Много лет А.\,П.\,Васьковский, помимо своей работы, занимался фенологией нашего края, публиковал брошюры «Ход сезонных явлений в окрестностях города Магадана» с результатами наблюдений, а\,также статьи с данными исследованиями в «Краеведческих записках Магаданского областного краеведческого музея».

В 1973\,г. вышла его брошюра <<Состояние охраны природы в Магаданской области>>. В~ней он подробно описывает отрицательные последствия периода освоения области, а также все недостатки и просчёты в охране природы, обращает внимание на отсутствие заповедной зоны: <<\textit{\dotsНеобходимо отметить, что Северо-Восток, ландшафты и биоценозы которого так интенсивно разрушались, не имеет на своей территории ни одного заповедника и~национального парка, охраняющего природу\dots}>> и указывает на необходимость выбора территорий, подлежащих охране в целях сохранения характерных ландшафтов и~редких видов растений и животных, а~также организации заповедников и национальных парков на территории Магаданской области. И уже в 1974\,г. он выступает с инициативой создания заповедника <<Магаданский>>.

За открытие крупных месторождений полезных ископаемых и научный вклад в изучение Северо-Востока А.\,П.\,Васьковский был награждён орденом Ленина, орденами <<Красной Звезды>>, <<Знак Почёта>>, а также медалями и значками [2].

В 1990\,г. в фонды музея Людмилой Ивановной Пастуховой, женой Алексея Петровича, была передана коллекция бабочек. Коллекция чешуекрылых собрана в окрестностях г.\,Магадана в период с 6 июля по 5 августа 1965\,г. и насчитывает 22 экз. Коллекция представлена 16 видами бабочек из 7 семейств. Определением видового состава коллекции в~1967\,г. занимался Алексей Иванович Куренцов (1896--1975). Им определено 14 видов.
Алексей Иванович~--- выдающийся биолог, энтомолог и биогеограф, доктор биологических наук, профессор, заслуженный деятель науки РСФСР, лауреат Государственной премии СССР, основатель дальневосточной школы энтомологов.

В 1932\,г. переехал на Дальний Восток и посвятил изучению природы его территории почти 50 лет. В более чем 80 экспедициях обследовал территории Приморского и Хабаровского краёв, Амурской, Сахалинской, Камчатской, Магаданской, Читинской областей, Чукотского и Корякского\,АО, Якутии, Курильские острова. Исследовал речные долины Амура, Хора, Большой Уссурки, Бикина, горных рек и ручьёв, склоны Сихотэ-Алиня, хребты Джугджур и Яблоновый, Колымское нагорье.

А.\,И.\,Куренцов занимался насекомыми~--- вредителями лесных насаждений. Основное внимание он уделял жукам, особенно короедам. Но его любимыми насекомыми были бабочки. Во время своего первого путешествия на Дальний Восток в 1920\,г. он посвятил изучению этой группы более 40 работ, а монография <<Булавоусые чешуекрылые Дальнего Востока СССР>>, богато иллюстрированная цветными рисунками и картами ареалов, сразу же стала библиографической редкостью. В течение всей жизни Алексей Иванович в теоретических исследованиях постоянно возвращался к бабочкам для иллюстрации положений и выводов [4].
\enlargethispage{\baselineskip}

Нами была переэтикетирована коллекция с учётом современной систематики. Наибольший научный интерес в коллекции представляет \textit{Parnassius stubbendorfi}. Впервые в Магаданской области встреча с бабочкой \textit{Parnassius stubbendorfi} (Аполлон Штуббендорфа) Menetries, 1849, была описана О.\,Э.~Костериным [3, 5] в 1992\,г. по результатам экспедиции на п-ов\,Кони в 1989\,г. Была поймана всего пара экземпляров, и эта находка сразу стала жемчужиной экспедиции. Дело в том, что впервые данный вид был отмечен на 59-й параллели, самой крайней северной точке ареала. Подвид магаданских Парусников Штуббендорфа получил имя О.\,Э.\,Костерина и звучит как <<\textit{Parnassius stubbendorfi kosterini}>>. Отмеченная популяция встречается на Ольском участке заповедника <<Магаданский>>. До 2019\,г. в Магаданской области бабочка имела статус 3-й\,категории краснокнижной, как узкоареальный подвид североазиатского вида, представленный в области популяцией, вероятно, реликт климатического оптимума голоцена [1]. В настоящее время \textit{Parnassius stubbendorfi} исключён из списка охраняемых видов\quadи\quadне\quadвошёл во второе издание Красной книги Магаданской области.\\Основанием для исключения послужили новые сведения о распространении и численности, которые выявили отсутствие опасений о состоянии популяций на территории области.

Длина переднего крыла 25--30\,мм. Крылья белые с контрастными чёрными жилками. На~передних крыльях самцов имеются две отчётливые сероватые (полупрозрачные) перевязи у внешнего края и узкое дискальное пятно. У самок эти элементы рисунка в среднем шире и более размытые [1, с.\,10]. Бабочка имеет бореальный сибирско-дальневосточный ареал. В~нашей стране бабочка встречается в Среднесибирском, Южно-Западносибирском, Красноярском, Предалтайском, Горно-Алтайском, Тувинском, Прибайкальском,\;\;Забайкальском, Северо-Охо\-то\-морс\-ком, Средне-Охотоморском, Средне-Амурском, Нижне-Амурском и Приморском регионах [6].

В 2018\,г. данная коллекция чешуекрылых в составе материалов А.\,П.\,Вась\-ковского, в которые входили предметы палеонтологии, орнитологии, а также личные вещи, была принята на постоянное хранение в фонды Магаданского областного краеведческого музея.

\textbf{Аннотированный список видов.}

В описании видов после видового названия идут: дата сбора; место сбора; количество экземпляров в коллекции; биотопическая приуроченность; ареал [7]; примечания.
\vspace{-6pt}
\begin{center}
\textbf{Класс Insecta (Насекомые)}

\textbf{Отряд Lepidoptera (Чешуекрылые)}
\end{center}

\textcenter{Семейство \MakeUppercase{Arctiidae}~--- Медведицы}
\vspace{-8pt}
\begin{enumerate}[noitemsep, leftmargin=0cm]

      \item \textit{Setina irrorella insignata} (Staudinger, 1881)~--- 06.07.1965~г. Окрестности Магадана; 1~экз.; на~приморских склонах, на лугах на границе леса на высоте около 1000 м н.~у.~м.

\textcenter{Семейство \MakeUppercase{NOCTUIDAE}~--- Совки}

      \item \textit{Syngrapha} af. \textit{transbaicalensis} (Staudinger, 1892)~--- 06.07.1965~г. Окрестности Магадана; 1~экз.; в~лиственничниках, на~торфяниках, верховых болотах; бореальный сибирско-дальневосточный.

      \item \textit{Rhyacia ledereri} (Erschoff, 1870)~--- 06.07.1965~г. Окрестности Магадана; 2~экз.; на~приморских склонах, в~лиственничниках; бореальный восточнопалеарктический. Cиноним \textit{Agrotis ipsilon} (Hufnagel, 1766).

      \item \textit{Agrotis characteristica} (Alpheraky, 1892)~--- 06.07.1965 г. Окрестности Mагадана; 3 экз.; степные биотопы. Синоним \textit{Agrotis robusta} (Eversmann, 1856).

\textcenter{Семейство \MakeUppercase{Hesperiidae}~--- Толстоголовки}

      \item \textit{Carterocephalus palaemon} (Pallas, 1771)~--- 16.07.1965~г. Окрестности Магадана; 1~экз.; в~долинах смешанных лесов и редколесьях; полизональный (кроме Арктики) голарктический.

      \item \textit{Carterocephalus silvicola} (Meigen, 1828)~--- 16.07.1965~г. Окрестности Магадана; 1~экз.; на~приморских склонах, морях, полянах, реже в лиственничниках с берёзой, горных тундрах; полизональный (кроме Арктики) транспалеарктический.

\textcenter{Семейство \MakeUppercase{papilionidae}~--- Парусники}

      \item \textit{Parnassius phoebus} (Fabricius, 1793)~--- 06.07.1965~г. Окрестности Магадана; 3 экз.; на~травяных склонах, болотах, в горных тундрах; арктобореальный бореомонтанный голарктический.

    \item \textit{Parnassius stubbendorfi} (Menetrie, 1848)~--- 06.07.1965~г. Окрестности Магадана; 1 экз.; на приморских склонах, на лугах, в лиственничных редколесьях; бореальный сибирско-дальневосточный.

\textcenter{Семейство \MakeUppercase{Pieridae}~--- Белянки}

      \item \textit{Colias palaeno} (Linnaeus, 1761)~--- 05.08.1965~г. Окрестности Магадана (р.~Каменушка); 1~экз.; в поймах, на лиственничных марях, в тундрах; арктобореальный циркумголарктический.

      \item \textit{Pontia callidice} (Hubner, 1799)~--- 16.07.1965~г. Окрестности Mагадана; 1~экз.; на лугах, в горных, реже в низинных тундрах; арктобореальный арктоальпийский транспалеарктический. Синоним \textit{Synchloe callidice} (Hubner, [1800]).

\textcenter{Семейство \MakeUppercase{Satyridae}~--- Бархатницы}

      \item \textit{Coenonympha tullia} (Muller, 1764)~--- 05.08.1965~г. Окрестности Магадана (р.~Каменушка); 1~экз. и 16.07.1965~г. Окрестности Mагадана; 1~экз.; на болотах, в лиственничных редколесьях; полизональный циркумголарктический.

      \item \textit{Erebia ajanensis} (Menetries, 1857)~--- 06.07.1965~г. Окрестности Mагадана; 1~экз.; на~приморских склонах, в лиственничных редколесьях; бореальный дальневосточный.

      \item  \textit{Erebia embla} (Thunberg, 1791)~--- 06.07.1965~г. Окрестности Магадана; 1~экз.; на болотах, марях, в~лиственничниках; арктобореальный транспалеарктический.

      \item \textit{Oeneis magna} (Graeser, 1888)~--- 16.07.1965~г. Окрестности Mагадана; 1~экз.; на лугах, болотах, в пойменных заболоченных и горных лиственничных редколесьях; бореальный сибирско-дальневосточный.

      \item \textit{Oeneis jutta} (Hübner, 1806)~--- 06.07.1965~г. Окрестности Магадана; 1 экз.; на лугах, болотах, в пойменных лесах, в заболоченных лиственничниках, лесотундрах и горных тундрах; субарктобореальный циркумголарктический.

\textcenter{Семейство \MakeUppercase{NYmphalidae}~--- Переливницы}

      \item \textit{Boloria angarensis} (Ershov, 1870)~--- 05.08.1965~г. Окрестности Mагадана (р.~Каменушка); 1~экз.; на приморских склонах, лугах, болотах, в~лиственничных редколесьях, зарослях ольхи и кедрового стланика; арктобореальный палеарктический. Синоним \textit{Clossiana angarensis} (Ershov, 1870).
\end{enumerate}


\begin{thebibliography}{99}
%TODO: Оформить!

\bibitem{}\BibAuthor{Берман~Д.\,И., Горбунов\,П.\,Ю., Катаев\,Б.\,М.} Красная книга Магаданской области. Редкие и~находящиеся под угрозой исчезновения виды растений и животных.~--- Магадан~: ИБПС ДВО РАН, 2008.~--- 429\,с.
\bibitem{}\BibAuthor{Васьковский\,А.\,П.} Состояние охраны природы в Магаданской области.~--- Магадан~: ИБПС ДВНЦ АН СССР, 1973.~--- 10\,с.
\bibitem{}\BibAuthor{Костерин\,О.\,Э.} Булавоусые чешуекрылые (Lepidoptera, Diurna) полуострова Кони. Магаданская область // Actias.~--- 1994.~--- №\,1.~--- C.\,77--81.
\bibitem{}\BibAuthor{Купянская\,А.\,Н., Лелей\,А.\,С., Стороженко\,С.\,Ю.} Алексей Иванович Куренцов // Вестник ДВО РАН.~--- 2007.~--- №\,6.~--- С.\,155--161.
\bibitem{}\BibAuthor{Лейто\,А., Мянд\,Р., Оя\,Т. и др.} Исследование экосистем полуострова Кони. Магаданский заповедник.~--- Таллин~: АН Эстонии, 1991.~--- 224~с.
\bibitem{}\BibAuthor{Рябухин\,А.\,С., Засыпкина\,И.\,А.} Наземные и пресноводные насекомые побережья Тауйской губы // Биологическое разнообразие Тауйской губы Охотского моря.~--- Владивосток~: Дальнаука, 2005.~--- С.\,290--476.
\bibitem{}\BibAuthor{Синев\,С.\,Ю., Козлов\,М.\,В.} Micropterigidae\,: каталог чешуекрылых (Lepidoptera) России.~--- СПб.\,; М.\,: Товарищество научных изданий КМК, 2008.~--- С.\,296--423.

\textbf{Архивные источники}

\bibitem{}Предложение Горазеева\,В.\,И. комитету по радиовещанию и телевидению ввести раздел <<Природа родного края>> с привлечением геолога Васьковского\,А.\,П. и др.~--- Магадан, 1968.~--- 1~л.~--- Фонды МОКМ.
\end{thebibliography}
