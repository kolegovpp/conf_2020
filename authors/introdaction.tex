\begin{center}
  \textbf{В честь 60-летия СВКНИИ ДВО РАН}
\end{center}

Возникновение\,\,\,академической\,\,\,науки\,\,на\,\,\,Северо-Востоке\,\,\,России относится к~1960~г., когда по решению Президиума АН СССР в составе Сибирского отделения был создан Северо-Восточный комплексный научно-исследовательский институт (СВКНИИ), директором которого был назначен тогда еще кандидат наук, будущий академик и~председатель ДВО АН СССР Н.~А.~Шило. В отличие от существовавших уже в это время в Магадане ведомственных научных подразделений, вновь созданный институт сразу же был нацелен на~решение фундаментальных проблем сначала в области геолого-геофизических, историко-археологических и~социально-экономических, а затем космофизических и~биологических наук. На этот же период пришлось и~образование Совета молодых учёных и~специалистов. Молодые специалисты того времени приезжали в Магадан со всех уголков страны, они были амбициозны, предприимчивы и~полны надежд на светлое будущее региона, даже в~тяжёлых условиях климата. Так как Магаданская область является отдалённым регионом, то~молодые учёные устраивали различные научные мероприятия для обмена опытом.

Одной из главных задач Совета было сплочение инициативной молодежи не только в~стенах СВКНИИ ДВО РАН, но и~за его пределами. Так, в~2006~г. состоялся межинститутский семинар молодых ученых и~специалистов СВКНИИ ДВО РАН и~НИЦ <<Арктика>>, которые позднее вырос до межрегиональной конференции <<Научная молодёжь~--- Северо-Востоку России>>. Конференция проходит раз в два года, и~в её рамках существует школьная конкурс-выставка научно-исследовательский работ <<Будущее начинается сегодня>>. В~этом юбилейном для СВКНИИ ДВО РАН году с 26 по 27 ноября проходит восьмая по счёту конференция и~четвёртая конкурс-выставка школьных работ. Тематика научных направлений конференции охватывает достаточно широкий диапазон: региональную геологию и~геофизические методы исследований, изучение и~освоение минерально-сырьевых ресурсов, анализ и~состояние объектов окружающей среды, проблемы рационального природопользования, физико-математические и~компьетерные исследования, проблемы биоразнообразия и~состояния экосистем северных регионов, история освоения и~развития Северо-Востока России, особо охраняемые природные территории и~экология культуры, правовая наука и~социально-экономическое развитие Северо-Востока России, медико-экологические проблемы северных территорий.

По мнению Совета, одним из путей сохранения, а главное~--- восполнения интеллектуального и~научного потенциала Магаданской области как основы профессионального роста молодых учёных и~специалистов, является реализация комплексных мер по привлечению молодёжи к научному знанию, способствующему всестороннему и~полноценному развитию интеграции образования и~науки, что благотворно скажется на научном росте молодых сотрудников институтов, задействованных в подобных мероприятиях, и~формированию сильного научного сообщества.


\begin{flushright}
\textit{Организационный комитет конференции\\
<<Научная молодёжь~--- Северо-Востоку России>>}
\end{flushright}
