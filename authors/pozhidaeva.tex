\procTitle{Палеомагнитные исследования позднеплейстоценовых осадков Чукотского моря}
\procAuthor{Пожидаева~Д.\,К.}
\procEmail{pozhidaeva@neisri.ru}
\procOrganization{СВКНИИ ДВО РАН} \procCity{Магадан}

\makeProcTitle
\index{p@Пожидаева~Д.\,К.}

Арктические бассейны являются важным архивом палеоклиматических
изменений.

Однако, для этих объектов важнейшей проблемой является стратиграфия и
хронология в связи с низкими скоростями осадконакопления, низкой
биологической продуктивностью, растворением органических остатков,
неоднозначностью интерпретации данных физических и геохимических методов
[2, 3, 4].

Реконструкция характеристик геомагнитного поля по морским осадкам в
Арктике позволяет получить возрастные стратиграфические реперы,
являющиеся основой для создания надежных возрастных моделей

Изучение синхронных изменений параметров геомагнитного поля~--- наклонение (J), склонение (D) и интенсивность естественной остаточной намагниченности (In) на участках из одной и той же области может привести к выявлению региональных хроностратиграфических маркеров. Палеомагнитные исследования проводились для осадков колонок, расположенных на континентальных шельфах и склонах арктических окраин морей. Так как эти районы характеризуются высокими скоростями осаждения, то именно они являются ключевыми областями для рассмотрения изменчивости геомагнитного поля тысячелетнего и столетнего масштабов для голоцена [2].

Исследованы ориентированные образцы колонок Чукотского моря LV83~--- 7, 8,
11 и 14, отобранные непрерывно по длине керна в кубики с ребром 2~см.

Магнитные характеристики измерены на оборудовании российского и чешского
производства: Казанский университет (магнитные весы Фарадея,
коэрцитивный спектрометр [1], фирма AGICO (JR-5A, LDA-2A, AMU-1A,
MFK–FA).

Характеристическая остаточная\quadнамагниченность (ChRM)\quadвы\-де\-ле\-на из~ес\-тест\-вен\-ной остаточной намагниченности (NRM) со ступенчатым размагничиванием в переменном поле 2,5, 5, 7,5, 10, 20, 25, 30, 35, 40, 45, 50, 60, 70 и~80~мТл. Безгистерезисную остаточную намагниченность (ARM) определяли путём применения затухающего поля AF (пиковое поле 80~мТл) с использованием размагничивающего устройства LDA-2A, AMU-1A.

Полученные результаты  исследований позволили построить кривые изменения магнитных параметров  ChRM, ARM, MS по длине колонок и вычислить значения относительной палеонапряжённости (RPI) методом нормировки характеристической намагниченности на магнитную восприимчивость и безгистерезисную остаточную намагниченность [5]. Анализируя изменения полученных данных, предварительно был сделан вывод, что разрезы кернов донных осадков Чукотского моря представлены позднеплейстоценовыми отложениями.



\begin{thebibliography}{99}
%1
\bibitem{}\BibAuthor{Буров~Б.~В., Нургалиев~Д.~К., Ясонов~П.~Г.} Палеомагнитный анализ.~--- Казань~: КДУ. ун-та, 1986.~--- 167~с.
\bibitem{}\BibAuthor{Darby~D.~A., Polyak~L., Bauch~H.~A.} Past glacial and interglation conditions in the Arctic Ocean and marginal sea~--- a revive // Progress in Oceanolography.~--- 2006.~--- Vol.~71.~--- P.~129--144.
\bibitem{}\BibAuthor{Lize-Pronovost~A., St-Onge~G., Brachfeld~S., et al.} Paleomagnetic constraints on the Holocene stratigraphy of the Arctic Alaskan margin // Global and Planetary Change.~--- 2009.~--- Vol.~68.~--- P.~85--99.~--- DOI:~10.1016/j.gloplacha.2009.03.015.
\bibitem{}\BibAuthor{Polyak~L., Bischof~J., Ortiz~J.~D. et al.} Late Quaternary and sedimentation patterns in the western Arctic Ocean // Global and Planetary Change.~--- 2009.~--- Vol.~68.~--- P.~5--17.~--- DOI:~10.1016/j.gloplacha.2009.03.014
\bibitem{}\BibAuthor{Tauxe~L.} Sedimentary records of relative paleointensity: theory and practice // Reviews of Geophysics.~--- 1993.~--- Vol.~32.~--- P.~319--354.~--- DOI:~10.1029/93RG01771.

\end{thebibliography}
\thispagestyle{empty}
