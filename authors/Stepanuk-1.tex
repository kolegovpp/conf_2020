\procTitle{Четырёхвершинники полуплевой проективной плоскости порядка~16}
\procAuthor{Степанюк~Г.\,А.}
\procEmail{glebstepanyk1998@gmail.com}
\procOrganization{СВГУ} \procCity{Магадан}

\makeProcTitle
\index{s@Степанюк~Г.\,А.}


Проективной плоскостью называется инцидентностная структура $(P, L, I)$, если выполняются следующие аксиомы:
\begin{description}[noitemsep]\vspace{-6pt}
\item[А1.] Для любых двух различных точек существует единственная инцидентная им прямая;
\item[А2.] Для любых двух различных прямых существует единственная инцидентная им точка;
\item[А3.] Существует хотя по крайней мере точки, любые три из которых не инцидентны.
\end{description}\vspace{-8pt}

Проективная плоскость называется конечной, если множество её точек (прямых) является конечным. Будем говорить, что проективная плоскость имеет порядок $n>1$, если некоторой прямой проективной плоскости инцидентна в точности $n+1$ точка.

Пусть $F_4 = \lbrace0,1,\upalpha,\upbeta\rbrace$; $\langle F_4,+,\cdot \rangle$~--- поле, содержащее 4~элемента,\\ $S_{16} = \lbrace (x,y)|x,y \in F_4 \rbrace$, $\forall x,y,u,v, \in F_4$, $(x,y)+(u,v)=(x+u,y+v)$. Следуя [3] определим бинарную операцию $\circ$ на $S_{16} : (x,y)\circ(u,v) = (xu+y^2v, yu+x^2v+y^2v^2)$. Множество $S_{16}$
относительно бинарных операций $+$ и $\circ$ бразует полуполе порядка 16. Бинарная операция $\circ$ в полуполе $S_{16}$ описана в таблице 1.



% \clearpage
% \begin{wrapfigure}{C}{6cm}
%   \centering
% \input{authors/Stepanuk-1-tab-1-1.tex}
% \end{wrapfigure}
% \clearpage



\begin{equation}
  \Theta_{TO-BE} = \langle \Psi, P_p, M, Q_p, Q_M, O_M \rangle,
\end{equation}



\begin{table}[H]
\caption{Таблица Кэли полуполя $S^1_{16}=(S_{16},+,\circ)$}
\label{tab:Stepanuk-1-tab-1}
\begin{changemargin}{-0.5cm}{-1cm}
\begin{tabular}{p{0.7cm}|p{0.5cm}p{0.5cm}p{0.5cm}p{0.5cm}p{0.5cm}p{0.5cm}p{0.5cm}p{0.5cm}p{0.5cm}p{0.5cm}p{0.5cm}p{0.5cm}p{0.5cm}p{0.5cm}p{0.5cm}p{0.5cm}}
 \makebox[0.5cm][c]{$\circ$}    & (0;0) & (0;1) & (0;$\upalpha$) & (0;$\upbeta$) & (1;0) & (1;1) & (1;$\upalpha$) & (1;$\upbeta$) & ($\upalpha$;0) & ($\upalpha$;1) & ($\upalpha$;$\upalpha$) & ($\upalpha$;$\upbeta$) & ($\upbeta$;0) & ($\upbeta$;1) & ($\upbeta$;$\upalpha$) & ($\upbeta$;$\upbeta$) \\\hline


(0;0) & (0;0) & (0;0) & (0;0) & (0;0) & (0;0) & (0;0) & (0;0) & (0;0) & (0;0) & (0;0) & (0;0) & (0;0) & (0;0) & (0;0) & (0;0) & (0;0) \\
(0;1) & (0;0) & (1;1) & ($\upalpha$;$\upbeta$) & ($\upbeta$;$\upalpha$) & (0;1) & (1;0) & ($\upalpha$;$\upalpha$) & ($\upbeta$;$\upbeta$) & (0;$\upalpha$) & (1;$\upbeta$) & ($\upalpha$;1) & ($\upbeta$;0) & (0;$\upbeta$) & (1;$\upalpha$) & ($\upalpha$;0) & ($\upbeta$;1) \\
(0;$\upalpha$) & (0;0) & ($\upbeta$;$\upbeta$) & (1;$\upalpha$) & ($\upalpha$;1) & (0;$\upalpha$) & ($\upbeta$;1) & (1;0) & ($\upalpha$;$\upbeta$) & (0;$\upbeta$) & ($\upbeta$;0) & (1;1) & ($\upalpha$;$\upalpha$) & (0;1) & ($\upbeta$;$\upalpha$) & (1;$\upbeta$) & ($\upalpha$;0) \\
(0;$\upbeta$) & (0;0) & ($\upalpha$;$\upalpha$) & ($\upbeta$;1) & (1;$\upbeta$) & (0;$\upbeta$) & ($\upalpha$;1) & ($\upbeta$;$\upalpha$) & (1;0) & (0;1) & ($\upalpha$;$\upbeta$) & ($\upbeta$;0) & (1;$\upalpha$) & (0;$\upalpha$) & ($\upalpha$;0) & ($\upbeta$;$\upbeta$) & (1;1) \\
(1;0) & (0;0) & (0;1) & (0;$\upalpha$) & (0;$\upbeta$) & (1;0) & (1;1) & (1;$\upalpha$) & (1;$\upbeta$) & ($\upalpha$;0) & ($\upalpha$;1) & ($\upalpha$;$\upalpha$) & ($\upalpha$;$\upbeta$) & ($\upbeta$;0) & ($\upbeta$;1) & ($\upbeta$;$\upalpha$) & ($\upbeta$;$\upbeta$) \\
(1;1) & (0;0) & (1;0) & ($\upalpha$;1) & ($\upbeta$;1) & (1;1) & (0;1) & ($\upbeta$;0) & ($\upalpha$;0) & ($\upalpha$;$\upalpha$) & ($\upbeta$;$\upalpha$) & (0;$\upbeta$) & (1;$\upbeta$) & ($\upbeta$;$\upbeta$) & ($\upalpha$;$\upbeta$) & (1;$\upalpha$) & (0;$\upalpha$) \\
(1;$\upalpha$) & (0;0) & ($\upbeta$;$\upalpha$) & (1;0) & ($\upalpha$;$\upalpha$) & (1;$\upalpha$) & ($\upalpha$;0) & (0;$\upalpha$) & ($\upbeta$;0) & ($\upalpha$;$\upbeta$) & (1;1) & ($\upbeta$;$\upbeta$) & (0;1) & ($\upbeta$;1) & (0;$\upbeta$) & ($\upalpha$;1) & (1;$\upbeta$) \\
(1;$\upbeta$) & (0;0) & ($\upalpha$;$\upbeta$) & ($\upbeta$;$\upbeta$) & (1;0) & (1;$\upbeta$) & ($\upbeta$;0) & ($\upalpha$;0) & (0;$\upbeta$) & ($\upalpha$;1) & (0;$\upalpha$) & (1;$\upalpha$) & ($\upbeta$;1) & ($\upbeta$;$\upalpha$) & (1;1) & (0;1) & ($\upalpha$;$\upalpha$) \\
($\upalpha$;0) & (0;0) & (0;$\upbeta$) & (0;1) & (0;$\upalpha$) & ($\upalpha$;0) & ($\upalpha$;$\upbeta$) & ($\upalpha$;1) & ($\upalpha$;$\upalpha$) & ($\upbeta$;0) & ($\upbeta$;$\upbeta$) & ($\upbeta$;1) & ($\upbeta$;$\upalpha$) & (1;0) & (1;$\upbeta$) & (1;1) & (1;$\upalpha$) \\
($\upalpha$;1) & (0;0) & (1;0) & ($\upalpha$;$\upbeta$) & ($\upbeta$;0) & ($\upalpha$;1) & ($\upbeta$;$\upbeta$) & (0;$\upbeta$) & (1;$\upalpha$) & ($\upbeta$;$\upalpha$) & ($\upalpha$;0) & (1;1) & (0;$\upalpha$) & (1;$\upbeta$) & (0;1) & ($\upbeta$;1) & ($\upalpha$;$\upalpha$) \\
($\upalpha$;$\upalpha$) & (0;0) & ($\upbeta$;0) & (1;$\upbeta$) & ($\upalpha$;$\upbeta$) & ($\upalpha$;$\upalpha$) & (1;$\upalpha$) & ($\upbeta$;1) & (0;1) & ($\upbeta$;$\upbeta$) & (0;$\upalpha$) & ($\upalpha$;0) & (1;0) & (1;1) & ($\upalpha$;1) & (0;$\upalpha$) & ($\upbeta$;$\upalpha$) \\
($\upalpha$;$\upbeta$) & (0;0) & ($\upalpha$;1) & ($\upbeta$;0) & (1;1) & ($\upalpha$;$\upbeta$) & (0;$\upalpha$) & (1;$\upbeta$) & ($\upbeta$;$\upalpha$) & ($\upbeta$;1) & (1;0) & (0;1) & ($\upalpha$;0) & (1;$\upalpha$) & ($\upbeta$;$\upbeta$) & ($\upalpha$;$\upalpha$) & (0;$\upbeta$) \\
($\upbeta$;0) & (0;0) & (0;$\upalpha$) & (0;$\upbeta$) & (0;1) & ($\upbeta$;0) & ($\upbeta$;$\upalpha$) & ($\upbeta$;$\upbeta$) & ($\upbeta$;1) & (1;0) & (1;$\upalpha$) & (1;$\upbeta$) & (1;1) & ($\upalpha$;0) & ($\upalpha$;$\upalpha$) & ($\upalpha$;$\upbeta$) & ($\upalpha$;1) \\
($\upbeta$;1) & (0;0) & (1;$\upbeta$) & ($\upalpha$;0) & ($\upbeta$;$\upbeta$) & ($\upbeta$;1) & ($\upalpha$;$\upalpha$) & (1;1) & (0;$\upalpha$) & (1;$\upalpha$) & (0;1) & ($\upbeta$;$\upalpha$) & ($\upalpha$;1) & ($\upalpha$;$\upbeta$) & ($\upbeta$;0) & (0;$\upbeta$) & (1;0) \\
($\upbeta$;$\upalpha$) & (0;0) & ($\upbeta$;1) & (1;1) & ($\upalpha$;0) & ($\upbeta$;$\upalpha$) & (0;$\upbeta$) & ($\upalpha$;$\upbeta$) & (1;$\upalpha$) & (1;$\upbeta$) & ($\upalpha$;$\upalpha$) & (0;$\upalpha$) & ($\upbeta$;$\upbeta$) & ($\upalpha$;1) & (1;0) & ($\upbeta$;0) & (0;1) \\
($\upbeta$;$\upbeta$) & (0;0) & ($\upalpha$;0) & ($\upbeta$;$\upalpha$) & (1;$\upalpha$) & ($\upbeta$;$\upbeta$) & (1;$\upbeta$) & (0;1) & ($\upalpha$;1) & (1;1) & ($\upbeta$;1) & ($\upalpha$;$\upbeta$) & (0;$\upbeta$) & ($\upalpha$;$\upalpha$) & (0;$\upalpha$) & (1;0) & ($\upbeta$;0)
\end{tabular}
\end{changemargin}
\end{table}




\begin{thebibliography}{99}
%1
\bibitem{}\BibAuthor{Вендров\,А.\,М.} Проектирование программного обеспечения экономических информационных систем / 2-е изд.~--- М.\,: Финансы и статистика, 2005. ~--- 544\,с.

\bibitem{}\BibAuthor{Ясенев\,В.\,Н.} Информационные системы и технологии в экономике / 3-е изд., перераб. и доп.~--- М.\,: ЮНИТИ, 2012.~--- 560\,с.

\bibitem{}\BibAuthor{Васин С.\,Г.} Искусственный интеллект в управлении государством // Управление.~--- 2017.~--- №\,3 (17).~--- URL: https://cyberleninka.ru/article/n/iskusstvennyy-intellekt-v-upravlenii-gosudarstvom (дата обращения: 16.03.2020).

\bibitem{}Указ Президента РФ от 09.05.2017 № 203 <<О\,Стратегии развития информационного общества в Российской Федерации на 2017--2030 годы>> // Консультант Плюс.~--- URL: http://www.consultant.ru/document/cons\_doc\_LAW\_216363/ (дата обращения: 16.03.2020).

\bibitem{}Указ Президента Российской Федерации от 10.10.2019 № 490 <<О развитии искусственного интеллекта в Российской Федерации>> // Консультант Плюс.~--- URL: http://www.consultant.ru/document/cons\_doc\_LAW\_335184/ (дата обращения: 16.03.2020).

\bibitem{}Указ Президента РФ от 30.01.2019 № 30 <<О грантах Президента Российской Федерации, предоставляемых на развитие гражданского общества>> // Законы, кодексы и нормативно-правовые акты в Российской Федерации.~--- URL: https://legalacts.ru/doc/ukaz-prezidenta-rf-ot-30012019-n-30-o-grantakh/\#100014 (дата обращения: 16.03.2020).

\bibitem{}\BibAuthor{Копченко\,В.\,К., Сироткин\,А.\,В.} Применение систем искусственного интеллекта для распределения грантов предпринимателям магаданской области // Студенческий: электрон. научн. журн. - №\,11 (55).~--- С.\,20--24.

\bibitem{}\BibAuthor{Шушкевич\,Н.\,А., Мохов\,А.\,И., Крупский\,А.\,Ю., Нургазиева\,А.\,С.} Экспертиза проектов на основе информационной модели // Вестник евразийской науки.~--- 2010.~--- №\,4.~--- URL: https://cyberleninka.ru/article/n/ekspertiza-proektov-na-osnove-informatsionnoy-modeli (дата обращения: 20.02.2020).

\bibitem{}\BibAuthor{Марон\,А.\,И., Марон\,М.\,А.} Метод конкурсного отбора проектов // Открытое образование.~--- 2011.~--- №\,2. URL: https://cyberleninka.ru/article/n/metod-konkursnogo-otbora-proektov (дата обращения: 20.02.2020).

\bibitem{}\BibAuthor{Семиглазов\,В.\,А.} Развитие методики отбора инновационных проектов в условиях полной неопределенности // Инновации.~--- 2006.~--- №\,11.~--- URL: https://cyberleninka.ru/article/n/razvitie-metodiki-otbora-innovatsionnyh-proektov-v-usloviyah-polnoy-neopredelennosti (дата обращения: 20.02.2020).

\bibitem{}\BibAuthor{Мельник\,П.\,Б.} Математическая постановка задачи формирования реестра экспертов // Инноватика и экспертиза.~--- №\,2 (13).~---С. 69--81.

\bibitem{}\BibAuthor{Мельник\,П.\,Б.} Методика формирования экспертных пулов и групп для проведения экспертно-аналитических исследований // Инноватика и экспертиза.~--- №\,1 (19).~--- С.\,39--54.

\bibitem{}\BibAuthor{Мельник\,П.\,Б.} Реестр экспертов как система массового обслуживания: модель и параметры входящего потока заявок // Инноватика и экспертиза.~--- №\,1 (22).~--- С.\,67--78.

\bibitem{}\BibAuthor{Морозова\,Т.\,В.} Экспертные технологии в оценке инновационных проектов // Известия ЮФУ. Технические науки.~--- 2011.~--- №\,11.~--- URL: https://cyberleninka.ru/article/n/ekspertnye-tehnologii-v-otsenke-innovatsionnyh-proektov (дата обращения: 20.02.2020).

\bibitem{}\BibAuthor{Горчакова\,Е.\,А.} О необходимости унификации критерия отбора проектов кооперации промышленных предприятий в кластере // Известия СПбГЭУ.~--- 2017.~--- №\,5 (107).~--- URL: https://cyberleninka.ru/article/n/o-neobhodimosti-unifikatsii-kriteriya-otbora-proektov-kooperatsii-promyshlennyh-predpriyatiy-v-klastere (дата обращения: 20.02.2020).

\bibitem{}\BibAuthor{Ильин\,И.\,В.} Критерии отбора проектов для их эффективной реализации на условиях проектного финансирования // Финансы и кредит.~--- 2008.~--- №\,35 (323).~--- URL: https://cyberleninka.ru/article/n/kriterii-otbora-proektov-dlya-ih-effektivnoy-realizatsii-na-usloviyah-proektnogo-finansirovaniya-2 (дата обращения: 20.02.2020).

\bibitem{}\BibAuthor{Косоруков\,А.\,А.} Технологии искусственного интеллекта в современном государственном управлении // Социодинамика.~--- 2019.~--- №\,5.~--- С.\,43--58.~--- DOI: 10.25136/2409-7144.2019.5.29714

\bibitem{}\BibAuthor{Новикова\,Т.\,Г.} Теоретические основы экспертизы инновационной деятельности в образовании~: Автореф. дис. $\dots$ д-ра пед. наук : 13.00.01 / Акад. повышения квалификации и переподгот. работников образования М-ва образования РФ.~--- Москва, 2006.~--- 47\,с.

\bibitem{}\BibAuthor{Сироткин\,А.\,В, Копченко\,В.\,К.} Концепция разработки автоматизированной системы распределения грантов // Сборник статей XXI международной научно-практической конференции.~--- М.\,: <<Научно-издательский центр ,,Актуальность.рф‘‘>>, 2019. С.\,107--109.

\bibitem{}\BibAuthor{Сироткин\,А.\,В., Старикова\,О.\,А.} Отраслевая идентификация заявок в автоматизированной экспертной системе распределения грантов // Современные наукоёмкие технологии.~---2019.~--- №\,7.~--- С.\,99--103.

\bibitem{}III\,Международная научно-практическая конференция <<На перекрестке Севера и Востока (методологии и практики регионального развития)>> / Северо-Восточный Государственный Университет.~--- URL: http://svkonf.svgu.ru (дата обращения: 21.02.2020).

\bibitem{}\BibAuthor{Broder\,A.} Identifying and Filtering Near-Duplicate Documents, COM’00 // Proceedings of the 11th Annual Symposium on Combinatorial Pattern Matching.~---2000.~---P.\,1--10.
\end{thebibliography}
\thispagestyle{empty}
