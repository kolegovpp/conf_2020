\procTitle{Распределение скорости смещения обломочного чехла в аккумулятивных частях коллювиальных конусов в~горах Дел-Урэкчэн (Северное Приохотье), по~лихенометрическим данным}
\procAuthor{Колегов~П.\,П.}
\procEmail{kolegovpp@gmail.com}
\procOrganization{СВКНИИ ДВО РАН} \procCity{Магадан}

\makeProcTitle
\index{k@Колегов~П.\,П.}

Рассматривая процессы склонового морфолитогенеза на территории Северного Приохотья
можно выделить следующие типы: обваливание, осыпание, десерпция, плоскостной смыв, солифлюкция.
Их изучением на Северо-Востоке Азии занимались разные исследователи с переменным успехом.
Стоит отметить работы Т.\,И.\,Каплиной посвященные криогенным образованиям
и солифлюкции в частности [4]; Э.\,Э.\,Титова описавшего типы коллювиальных образований
и их взаимосвязь с криолитозоной [9, 10]; В.\,Л.\,Суходровского разработавшего классификацию рельефообразующийх процессов криолитозоны [8].

Начиная с 2000-х гг. стали применять новые количественные и качественные методы (лихенометрия, анализ космических снимков высокого разрешения) при изучении динамики склоновых процессов,
что возобновило интерес к исследованиям в данной области. Здесь можно выделить работы В.\,Н.\,Смирнова и А.\,А.\,Галанина [1--3].

Нами проведены исследования по динамики и цикличности процессов формирующие коллювиальные конусы выноса на территории Северного Приохотья (центральная часть гор Дел-Урэкчэн), в ходе которых получены количественные данные по скорости смещения обломочного чехла (0,2--2,0\,м/год), динамического возраста (300--500\,лет) и выявлены геопространственные закономерности в размещении данных форм [5--7].

Целью данных исследований являлось выявление закономерности в распределении скорости смещения обломочного чехла в разных частях аккумулятивной зоны.

Участок исследования расположен в правом борту р.~Нельканджа (басс. р.~Армань) близ слияния с р.~Нанкалой (60°16’48”\,с.\,ш., 150°56’42”\,в.\,д.). Рельеф участка среднегорный, абсолютные отметки высот водоразделов составляют 850--900\,м, долины 530--560\,м (рис.\,1, слева). Ширина долины составляет 1500\,м, поймы до 200\,м. Форма долины U-образная. Вдоль бортов долин сохранились реликты моренные накопления последних оледенений. Склоны крутые (25--35\dg), покрытые маломощным обломочным чехлом, коренные выходы на них редки.

Для описание рельефа использовалась новая цифровая модель~--- ArcticDEM [11], которая имеет разрешение 2 м/пиксель.

Изученный нами коллювиальный конус имеет следующие морфометрические характеристики, м: длина~---  450, ширина транзитной части~---  15--25, аккумулятивной~--- 115; угол наклона транзитной части 19--26\dg, аккумулятивной~--- 12--18\dg.
Обломочный чехол сложен крупнощебнистыми и мелкоглыбовым материалом в аккумулятивной зоне, и мелко-, срденещебнистые отложениями с дресвяным заполнителем в транзитной. Петрографический состав обломков представлен риолитами.

%TODO: масштабная линейка на космоснике
\begin{figure}[H]
  \centering
  \includegraphics[width=1\textwidth, page=1]{authors/Kolegov-fig.pdf}
  \caption{Цифровая модель рельефа [11] участка Нелькаджа (слева) и космоснимок фронтальной части конуса (справа). Условные обозначения: 1~--- лихеометрическая площадка и её номер, 2~--- направления транзита обломков и их скорость в м/год; на левом рисунке~--- горизонтали проведены через 50\,м, на правом~--- толстые через 10\,м, тонкие через 1\,м. На левой врезке~--- географическое положение участка исследования}
  \label{fig:kolegov-fig}
\end{figure}

В основании конуса, а также в транзитной части были заложены 4 лихенометрических площадки (см. рис.\,1, справа). Методика лихенометрической съёмки описана нами ранее в [5, 6], но имеются небольшие различия в количестве замеров. На площадке 20$\times$20\,м проводилось измерение самой крупной особи \textit{Rhizocarpon}~sp. на каждом из 25 случайно выбранном обломке горной породы.

Проведенные статистические исследования базы данных методом Монте-Карло по возможности измерения 25\,шт. лишайников на одной площадке показали, что ошибка в определении возраста в интервале 200--500\,лет составляет от 8 до 15\,\%. База данных состояла из 70 площадко по 100 измерений в каждой. Сущность метода заключалась в том, что из генеральной совокупности измерений по каждой площадки делалась выборка из 25 шт., всего таких выборок создавалось 100 ед. После чего отстраивались графики распределения этих выборок и графики погрешностей (рис. 2). Полученные показатели позволяют производить измерения по 25 шт. лишайников.



Диаметр таллома пересчитан в возраст по уравнению [6]:

$$t = 1000\cdot ln{\left(-\frac{1}{d-230}\right)}+5438,02$$

где $t$~--- возраст таллома; $d$~--- максимальный диаметр лишайника рассчитанный по убывающему логарифмическому тренду выборки из 25 измерений.

Северо-Восточная край дистальной части, по данным [11], имеет высоту 555--560\,м, а Юго-Западный край~--- 550--554\,м.

...

...

...



\begin{thebibliography}{99}
%1
\bibitem{}\BibAuthor{Галанин\,А.\,А.} Лихенометрия: современное состояние и направление развития
метода (аналитический обзор).~--- Магадан~: СВКНИИ ДВО РАН, 2002.~--- 74~с.
%2
\bibitem{}\BibAuthor{Галанин\,А.\,А.} Каменные глетчеры Северо-Востока России: строение, генезис,
возраст, географический анализ~: Дисс. $\dots$ доктора наук~: 11.00.04 / Северо-Восточный комплексный НИИ ДВО РАН.~--- Владивосток, 2009.~--- 303~с.
%3
\bibitem{}\BibAuthor{Галанин\,А.\,А., Смирнов\,В.\,Н.} Динамика гравитационных склоновых процес-
сов в горах Северного Приохотья в позднем голоцене и лихенометрическая методика их моделирования и прогноза // Геоморфология.~--- 2004.~--- №~3.~--- С.~67--75.
%4
\bibitem{}\BibAuthor{Каплина\,Т.\,И.} Криогенные склоновые процессы.~--- М.~: Наука, 1965.~--- 296~с.
%5
\bibitem{}\BibAuthor{Колегов\,П.\,П.} Динамика коллювиальных процессов в хребте Дел-Урэкчэн (Се-
верное Приохотье) на основе лихенометрических данных // Вестник СВНЦ ДВО РАН.~--- 2016.~--- №~2.~--- С.~10--18
%6
\bibitem{}\BibAuthor{Колегов\,П.\,П.} Динамика осыпей и каменных глетчеров Ольского плато (Северное Приохотье) на основании лихенометрического и фотометрического гранулометрического анализов // Вестник СВНЦ ДВО РАН.~--- 2019.~--- №~3.~--- С.~54–62.~--- DOI: 10.34078/1814-0998-2019-3-54-62.
%7
\bibitem{}\BibAuthor{Колегов\,П.\,П.} Геопространственный анализ коллювиальных конусов выноса центральной части гор Дел-Урэкчэн (Северное Приохотье) // Форум <<Наука Северо-Востока России: фундаментальные и прикладные исследования в Северной Пацифике и Арктике>>. Магадан, 5--6 марта 2020~г. / Отв. ред. Н.\,А.\,Горячев.~--- Магадан~: СВКНИИ ДВО РАН, 2020.~--- С.~41--44.
%8
\bibitem{}\BibAuthor{Суходровский\,В.\,Л.} Экзогенное рельефообразование в криолитозоне.~--- М.~:
Наука, 1979.~--- 280~с.
%9
\bibitem{}\BibAuthor{Титов\,Э.\,Э.} Скорости перемещения обломочного материала на склонах гор
Северо-Востока СССР // Вестник МГУ. География.~--- 1970.~--- №~4.~--- С.~95--98.
%10
\bibitem{}\BibAuthor{Титов\,Э.\,Э.} Строение и развитие склонов гор Северо-Востока СССР~: Автореф.
дис. $\dots$ канд. георг. наук~: 693 / Э.\,Э.\,Титов; МГУ.~--- Мoсква, 1971.~--- 35~с.

\bibitem{}ArcticDEM~--- Polar Geospatial Center.~--- 2018.~--- URL: https://www.pgc.umn.edu/data/arcticdem/ (ref. date: 20.02.2020).

\end{thebibliography}
\thispagestyle{empty}
