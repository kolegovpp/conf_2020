\procTitle{Распределение скорости смещения обломочного чехла в~аккумулятивных частях коллювиальных конусов в~горах Дел-Урэкчэн (Северное Приохотье) по~лихенометрическим данным}
\procTitleNewLine{Распределение скорости смещения обломочного чехла в~аккумулятивных частях коллювиальных конусов\\в~горах Дел-Урэкчэн (Северное Приохотье)\\по~лихенометрическим данным}
\procAuthor{Колегов~П.\,П., Кондратьев~М.\,Н.}
\procEmail{kolegovpp@gmail.com, mkondratyev85@gmail.com}
\procOrganization{СВКНИИ ДВО РАН} \procCity{Магадан}

\makeProcTitleNewLine
\index{k@Колегов~П.\,П.}
\index{k@Кондратьев~М.\,Н.}

Рассматривая процессы склонового морфолитогенеза на территории Северного Приохотья,
можно выделить следующие типы: обваливание, осыпание, десерпция, плоскостной смыв, солифлюкция.
Их изучением на Северо-Востоке Азии занимались разные исследователи с переменным успехом.
Стоит отметить работы Т.\,И.\,Каплиной, посвящённые криогенным образованиям
и солифлюкции в частности [4]; Э.\,Э.\,Титова описавшего типы коллювиальных образований
и их взаимосвязь с криолитозоной [9, 10]; В.\,Л.\,Суходровского, разработавшего классификацию рельефообразующих процессов криолитозоны [8].

Начиная с 2000-х гг. стали применять новые количественные и качественные методы (лихенометрия, анализ космических снимков высокого разрешения) при изучении динамики склоновых процессов,
что возобновило интерес к исследованиям в данной области. Здесь можно выделить работы В.\,Н.\,Смирнова и А.\,А.\,Галанина [1--3].

Нами проведены исследования по динамике и цикличности процессов, формирующих коллювиальные конусы выноса на территории Северного Приохотья (центральная часть гор Дел-Урэкчэн), в ходе которых получены количественные данные по скорости смещения обломочного чехла (0,2--2,0\,м/год), динамического возраста (300--500\,лет) и выявлены геопространственные закономерности в размещении данных форм [5--7].

Целью данных исследований было выявление закономерности в распределении скорости смещения обломочного чехла в разных частях аккумулятивной зоны.

Участок исследования расположен в правом борту р.~Нельканджа (басс. р.~Армань) близ слияния с р.~Нанкала (60\dg16’48”\,с.\,ш., 150\dg56’42”\,в.\,д.). Рельеф участка среднегорный, абсолютные отметки высот водоразделов составляют 850--900\,м, долины~--- 530--560\,м (рис.\,1, слева). Ширина долины составляет 1500~м, поймы до 200~м. Форма долины U-образная. Вдоль бортов долин сохранились реликты моренных накоплений последних оледенений. Склоны крутые (25--35\dg), покрытые маломощным обломочным чехлом, коренные выходы на них редки.

Для описание рельефа использовалась новая цифровая модель~--- ArcticDEM [11], которая имеет разрешение 2 м/пиксель.

Изученный нами коллювиальный конус имеет следующие морфометрические характеристики, м: длина~---  450, ширина транзитной части~---  15--25, аккумулятивной~--- 115; угол наклона транзитной части~--- 19--26\dg, аккумулятивной~--- 12--18\dg. Превышение области питания над дистальной частью (высота морфоскульптуры)~--- 200~м. Правая часть аккумулятивной зоны (если ориентироваться вниз по склону) имеет превышение в 5~м относительно левой (анализ цифровой модели).

Обломочный чехол сложен крупнощебнистым и мелкоглыбовым материалом в аккумулятивной зоне и мелко-, среднещебнистыми отложениями с дресвяным заполнителем в~транзитной. Петрографический состав обломков представлен риолитами.

%TODO: масштабная линейка на космоснике
\begin{figure}[H]
  \centering
  \includegraphics[width=1\textwidth, page=1]{authors/Kolegov-fig.pdf}
  \caption{Цифровая модель рельефа [11] участка Нельканджа (слева) и космоснимок фронтальной части конуса (справа). Условные обозначения: 1~--- лихеометрическая площадка и её номер; 2~--- направления транзита обломков и их скорость, м/год; на левом рисунке~--- горизонтали проведены через 50\,м, на правом~--- толстые через 10\,м, тонкие через 1\,м. На левой врезке~--- географическое положение участка исследования}
  \label{fig:kolegov-fig}
\end{figure}

В основании конуса, а также в транзитной части были заложены 4 площадки (см. рис.\,1, справа). Методика работ описана нами ранее в [5, 6], но имеются небольшие различия, а~именно были выбраны следующие параметры лихенометрической съёмки:  размер площадки~--- 20\,$\times$\,20\,м; лишайник-индикатор~--- \textit{Rhizocarpon}~sp.; количество замеров на площадке~--- 25 случайно выбранных обломков горной породы; метод измерения~--- самый крупный таллом на обломке; точность измерения~--- 1~мм.

Данное количество замеров на одной площадке обусловлено проведёнными нами статистическими исследованиями методом Монте-Карло базы данных лихенометрической съёмки (70 площадок по 100 замеров) за 2010--2017 гг. Из генеральной совокупности измерений по каждой площадке делались 100 выборок по 25 замеров (рис. 2,\textit{а}). Средняя величина ошибки в определении возраста по сравнению с измерением 100 талломов составляет от 8 до 18\,\%, для интервала 200--500\,лет (рис. 2,\textit{б,в}).  Полученные показатели ошибки признаны допустимыми для многопрофильной съёмки в аккумулятивных частях конусов выноса.



Диаметр таллома пересчитан в возраст по уравнению [6]:

$$t = 1000\cdot ln{\left(-\frac{1}{d-230}\right)}+5438,02,$$

где $t$~--- возраст таллома; $d$~--- максимальный диаметр лишайника, рассчитанный по убывающему логарифмическому тренду выборки из 25 измерений.

Основные показатели лихенометрического анализа приведены в таблице. Отметим, что полученные значения характерны для поверхностного слоя мощностью не~более 20~см.

\begin{figure}[H]
  \begin{center}
    \includegraphics[width=0.95\textwidth]{authors/kolegov-fig2.jpg}
  \end{center}
  \vspace{-0.5cm}
  \caption{График распределения выборок (\textit{а}). Красными точками показаны истинные значения,
полученные по 100 замерам, чёрные боксы и кружки~--- выборки из 25 шт. Графики ошибок: \textit{б}~--- график зависимости истинных значений от выборки в 25 ед; \textit{в}~--- погрешность измерений, получаемая при замере 25 талломов}
  \label{fig:kolegov}
\end{figure}



 %\parbox[c][8em][c]{0.06\textwidth}{ \centering Про\-филь} &

 \begin{table}[H]
   \begin{center}
 \caption*{\textbf{Основные показатели лихенометрического анализа по профилям}}
 \label{tab:kolegov-tab}
 \begin{tabular}{ccccccc}
    \toprule
 \multirow{2}{*}{\parbox[c][1.25cm][c]{0.06\textwidth}{ \centering Про\-филь}} &
  \multirow{2}{*}{\parbox[c][1.25cm][c]{0.1\textwidth}{ \centering Пло\-щад\-ка}} &
   \multicolumn{2}{l}{\parbox[c][][c]{0.25\textwidth}{ \centering Диаметр таллома, мм}} &
    \multirow{2}{*}{\parbox[c][1.25cm][c]{0.1\textwidth}{ \centering Возраст, лет}} &
     \multirow{2}{*}{\parbox[c][1.25cm][c]{0.12\textwidth}{ \centering Рас\-стоя\-ние,~м}} &
      \multirow{2}{*}{\parbox[c][][c]{0.1\textwidth}{ \centering Ско\-рость, м/год}} \\

\cmidrule(r){3-4}                          &                           & \parbox[c][][c]{0.12\textwidth}{ \centering наб\-лю\-даемый}       & \parbox[c][][c]{0.12\textwidth}{ \centering тео\-ре\-ти\-чес\-кий}       &                          &                             &                           \\
  \toprule
 \multirow{2}{*}{1}       & 02-Нж                     & 82                & 90                  & 502                      & \multirow{2}{*}{64}         & \multirow{2}{*}{0,17}     \\
                          & 05-Нж                     & 25                & 28                  & 130                      &                             &                           \\
  \midrule
 \multirow{2}{*}{2}       & 03-Нж                     & 47                & 54                  & 268                      & \multirow{2}{*}{71}         & \multirow{2}{*}{0,53}     \\
                          & 05-Нж                     & 25                & 28                  & 130                      &                             &                           \\
  \midrule
 \multirow{2}{*}{3}       & 04-Нж                     & 65                & 61                  & 313                      & \multirow{2}{*}{60}         & \multirow{2}{*}{0,33}     \\
                          & 05-Нж                     & 25                & 28                  & 130                      &                             &                           \\
  \midrule
 \multirow{2}{*}{общ.*}    & 02--04-Нж                  & 82                & 96                  & 545                      & \multirow{2}{*}{65}         & \multirow{2}{*}{0,16}     \\
                          & 05-Нж                     & 25                & 28                  & 130                      &                             &\\


                          \bottomrule
 \end{tabular}
\end{center}
\textit{* Для расчёта времени экспонирования и скорости транспортировки обломочного чехла были объединены выборки замеров талломов площадок 02, 03, 04-Нж.}

 \end{table}


\clearpage

Анализ полученных данных позволяет сделать следующие выводы:

\begin{enumerate}[noitemsep]\vspace{-8pt}
  \item допустимо при лихенометрической съёмке производить 25 замеров талломов, при этом погрешность относительно обычной методики (100~замеров) составит 8--15\,\%;
  \item скорость транспортировки обломков в различных частях аккумулятивной зоны варьирует от~0,17 до~0,53~м/год, при среднем значении 0,34~м/год;
  \item динамический возраст поверхности аккумулятивной зоны варьирует от 268 до 502 лет в разных её частях.
\end{enumerate}



\begin{thebibliography}{99}
%1
\bibitem{}\BibAuthor{Галанин\,А.\,А.} Лихенометрия: современное состояние и направление развития
метода (аналитический обзор).~--- Магадан~: СВКНИИ ДВО РАН, 2002.~--- 74~с.
%2
\bibitem{}\BibAuthor{Галанин\,А.\,А.} Каменные глетчеры Северо-Востока России: строение, генезис,
возраст, географический анализ~: дис. $\dots$ д-ра наук.~--- Владивосток, 2009.~--- 303~с.
%3
\bibitem{}\BibAuthor{Галанин\,А.\,А., Смирнов\,В.\,Н.} Динамика гравитационных склоновых процессов в горах Северного Приохотья в позднем голоцене и лихенометрическая методика их моделирования и прогноза // Геоморфология.~--- 2004.~--- №~3.~--- С.~67--75.
%4
\bibitem{}\BibAuthor{Каплина\,Т.\,И.} Криогенные склоновые процессы.~--- М.~: Наука, 1965.~--- 296~с.
%5
\bibitem{}\BibAuthor{Колегов\,П.\,П.} Динамика коллювиальных процессов в хребте Дел-Урэкчэн (Северное Приохотье) на основе лихенометрических данных // Вестник СВНЦ ДВО РАН.~--- 2016.~--- №~2.~--- С.~10--18.
%6
\bibitem{}\BibAuthor{Колегов\,П.\,П.} Динамика осыпей и каменных глетчеров Ольского плато (Северное Приохотье) на основании лихенометрического и фотометрического гранулометрического анализов // Там же.~--- 2019.~--- №~3.~--- С.~54–62.~--- DOI: 10.34078/1814-0998-2019-3-54-62.
%7
\bibitem{}\BibAuthor{Колегов\,П.\,П.} Геопространственный анализ коллювиальных конусов выноса центральной части гор Дел-Урэкчэн (Северное Приохотье) // Форум <<Наука Северо-Востока России: фундаментальные и прикладные исследования в Северной Пацифике и Арктике>>. Магадан, 5--6 марта 2020~г. / отв. ред. Н.\,А.\,Горячев.~--- Магадан~: СВКНИИ ДВО РАН, 2020.~--- С.~41--44.
%8
\bibitem{}\BibAuthor{Суходровский\,В.\,Л.} Экзогенное рельефообразование в криолитозоне.~--- М.~:
Наука, 1979.~--- 280~с.
%9
\bibitem{}\BibAuthor{Титов\,Э.\,Э.} Скорости перемещения обломочного материала на склонах гор
Северо-Востока СССР // Вестник МГУ. География.~--- 1970.~--- №~4.~--- С.~95--98.
%10
\bibitem{}\BibAuthor{Титов\,Э.\,Э.} Строение и развитие склонов гор Северо-Востока СССР~: автореф.
дис. $\dots$ канд. геогр. наук.~--- Мoсква, 1971.~--- 35~с.

\bibitem{}ArcticDEM~--- Polar Geospatial Center // University of Minnesota.~--- 2018.~--- URL: https://www.pgc.umn.edu/\-data/arcticdem (ref. date: 20.02.2020).

\end{thebibliography}
\thispagestyle{empty}
