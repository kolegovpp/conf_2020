\procTitle{Систематики золоторудных месторождений (принципы построений и современное состояние проблемы)}
\procAuthor{Малиновский~М.\,А.}
\procEmail{Max\_malinovsky@neisri.ru}
\procOrganization{СВКНИИ ДВО РАН} \procCity{Магадан}

\makeProcTitle
\index{m@Малиновский~М.\,А.}

Вопрос типизации золоторудных месторождений всегда был актуальным. И главной трудностью в этом было выделение критериев или признаков, по которым удавалось бы произвести разделение. В данной работе сделана попытка проследить за развитием идей и современным состоянием данной проблемы. Одной из первых систематик, предложенной ещё в начале прошлого века и оказавшей огромное влияние на направление работ по~этой тематике, была систематика Вальдемара Линдгрена [3]. По своей сути она была генетической и сразу ставила вопрос о происхождении рудного вещества.  Основными критериями, по которым происходило разделение месторождений, была глубина их формирования и температура, а также приблизительная оценка давления. Идеи, заложенные в этом труде, получили поддержку и развитие в трудах известного <<магматиста>> своего времени Ганса Шнейдерхена [7]. В них критерии расположения и формирования месторождений относительно магматического источника получили более дробное деление.  В дальнейшем набирающее популярность учение о рудных формациях также внесло свои коррективы в~типизацию золоторудных объектов. И такие критерии рудноформационного анализа, как геохимия и~геотектонические условия образования месторождений, нашли своё отражение в систематиках того времени. Примером этого могут являться взгляды В.~И.~Смирнова и~Н.~В.~Петровской [4, 6]. В последующем такой подход так или иначе находил отражение в~последующих классификациях, и примером этого может служить систематика Ю.~М.~Щепотьева [8].

Новым шагом в этом направлении служит добавление признаков геолого-промышленного деления руд (морфология рудных тел, залегание), как, например, у М.~М.~Константинова с~соавторами [2]. С течением времени единственным отличием стал переход от геосинклинальной теории к теории тектоники плит, где всем типам золоторудных месторождений находилась <<своя>> геотектоническая обстановка (А.~А.~Сидоров с~соавторами [5]). Похожие тенденции наблюдались и у зарубежных учёных. Одним из значимых трудов в этой области является типизации Кокса и Сингера [9], а также Гровса [10]. В них производится разделение на~тектонически стабильные и нестабильные территории с соответствующей привязкой к~ним магматизма различного характера. В отечественной литературе аналогом такого подхода является систематика, приведённая в трудах Н.~А.~Горячева [1], где уже влияние рудноформационных идей отсутствует. И одним из последних взглядов на классифицирование золоторудных месторождений являются труды Филлипса и Повела [11]. В~них авторы задаются следующими вопросами: с одной стороны, разделяя золоторудные месторождения на те, из которых добывается золото, и те, из которых извлекается ещё какой-либо полезный металл, и имея в виду при этом критерии минералогии и флюидов. А~с другой~--- рассматривают валентное состояние золота для данной группировки и его геохимию. И уже с~этих позиций делят месторождения на различные типы с учётом их поисковых признаков.

Таким образом, в систематизации золоторудных месторождений вопрос об обстановках формирования месторождений всегда занимал центральное место, в~то время как минералого-геохимические и морфологические критерии были вторичными. Со временем эти критерии стали следствием обстановок формирования.

\begin{thebibliography}{99}
\bibitem{}\BibAuthor{Горячев~Н.~А.} Золоторудообразующие системы орогенных поясов // Вестник СВНЦ ДВО РАН.~--- 2006.~--- №~1.~--- С.~2--16.
\bibitem{}\BibAuthor{Константинов~М.~М., Аристов~В.~В., Наталенко~М.~В., Стружков~С.~Ф.} Геолого-промышленная группировка золоторудных месторождений // Минеральные ресурсы России. Экономика и управление.~--- 2007.~--- №~4.~--- С.~15--18.
\bibitem{}\BibAuthor{Линдгрен~В.} Геология рудных месторождений западных штатов США.~--- Л.;\,М.~: ОНТИ НКТП СССР, 1937.~--- 644~с.
\bibitem{}\BibAuthor{Петровская~Н.~В.} Самородное золото. Общая характеристика, типоморфизм, вопросы генезиса.~--- М.~: Наука, 1973.~--- 347~с.
\bibitem{}\BibAuthor{Сидоров~А.~А., Старостин~В.~И., Томсон~И.~Н., Волков~А.~В.} Проблемы рудноформационного анализа // Вестник СВНЦ ДВО РАН.~--- 2011.~--- №~2.~--- С.~17--29.
\bibitem{}\BibAuthor{Смирнов~В.~И.} Геология полезных ископаемых.~--- М.~: Недра, 1976.~--- 486~с.
\bibitem{}\BibAuthor{Шнейдерхен~Г.} Рудные месторождения.~--- М.~: Иностранная литература, 1958.~--- 504~с.
\bibitem{}\BibAuthor{Щепотьев Ю. М., Вартанян С. С., Новиков В. П.} Геолого-промышленные типы золоторудных месторождений // Изв. вузов. МГРИ.~--- 1994.~--- №~3.~--- С.~175--190.
\bibitem{}\BibAuthor{Cox~D.~P., Singer~D.~A.} Mineral Deposit Models // U.\,S. geological survey bulletin~--- 1992.~--- Vol.~1693.~--- 379~p.
\bibitem{}\BibAuthor{Groves~D.~I., Bierlein~F.~P.} Geodynamic settings of mineral deposit systems // Journal of the Geological Society.~--- 2007.~--- Vol.~164~--- P.~19--30.
\bibitem{}\BibAuthor{Phillips~N., Powell~R.} A practical classification of gold deposits, with a theoretical basis // Ore Geology Reviews.~--- Vol.~65.~--- Part~3.~--- P.~568--573.
\end{thebibliography}
\thispagestyle{empty}
