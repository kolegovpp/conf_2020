
\procTitle{Литолого-петрографическая характеристика пермских отложений Омолонского массива}

\procAuthorI{Морозова~В.\,С.}
\procEmailI{veronika.morozova.2019@inbox.ru}
\procOrganizationI{СВГУ}
\procCityI{Магадан}

\procAuthorII{Брынько~И.\,В.}
\procEmailII{ibrynko@mail.ru}
\procOrganizationII{СВКНИИ ДВО РАН, СВГУ}
\procCityII{Магадан}


\makeProcTitleII
\index{b@Брынько~И.\,В.}
\index{m@Морозова~В.\,С.}

Пермские отложения широко распространены в Магаданской области. Они представлены различными фациями, разнообразие которых обусловлено различием условий их образования. Отложения имеют большую мощность, нередко с ними связаны проявления и месторождения полезных ископаемых, в частности золота. Наиболее полные и хорошо охарактеризованные фауной разрезы находятся на Омолонском массива [2, 3]. По этому данная работа посвящена литолого-петрографическому изучению пермских отложений Омолонского массива. Так в~2014-2015~гг. нами было изучено три разреза пермских отложений по р.~Русская-Омолонская, руч.~Водопадный и р.~Мунугуджак. При макро и микроскопического изучения пермских отложений нами выделено 8 основных литотипов, участвующих в строении этих разрезов: колымиевые известняки, диамиктиты, серые туффиты, зелёные туффиты, серые туфоалевролиты, зелёные туфоалевролиты, серые песчаники, зелёные песчаники.

\textbf{Колымиевые известняки} встречаются в рулонской толще, джигдалинской, омолонской, фолькской и хивачской свитах. Эти известняки имеют серый цвет, с запахом сероводорода, плитчатые, массивные, крепкие, (глинистые, кремнистые). В~шлифах основная масса породы состоит из разрозненных призмочек кальцита размером от 0,02\,$\times$\,0,2~мм до 0,05\,$\times$\,0,7~мм, преобладающая фракция 0,05\,$\times$\,0,3~мм, слагает от 70 до 95\,\%. Терригенная примесь представлена кварцем, полевым шпатом, плагиоклазом, реже халцедоном, слагает от 1 до 15\,\% породы. Форма этих зёрен идиоморфная, размер от 0,05 до 0,4~мм. Цемент известняков различный: хлорит-карбонатного, карбонатного, глинисто-хлорит-карбонатного состава, слагает от 5 до 15\,\% в породе.

\textbf{Диамиктиты} слагают среднюю часть гижигинской свиты. Диамиктиты имеют тёмно-серый цвет, слабоизвестковистые, щебенчатые, со скорлуповатой отдельностью, массивной текстуры, без какой либо ориентировки изометричных включений [1]. Рассеянный обломочный материал (включения) представлен вулканитами и, в меньшей степени – серыми алевролитами и гранитоидами\,(?). Включения имеют размер от 0,5 до 30\,см в длину, содержание в породе может варьировать, но не менее 10--15\,\%. В шлифах диамиктит на более чем 50\,\% сложен плохо сортированным, серицит-кварц-полевошпат-глинистым материалом с размером 0,005--0.025\,мм, который составляет основную массу породы. Обломочные частицы имеют удлинённую, остроугольную и изометричную форму, плохую сортировку, они плохо окатаны, часто угловатые. Аллотигенные компоненты представлены клисталлокластами полевого шпата, кварца, редко плагиоклаза с размером обломков от 0,05 до 0,7\,мм, слагающими до 20\,\% породы. В породе присутствуют литокласты, представленные эффузивами среднего~--- основного состава с фельзитовой, порфировой и гиалопиловой текстурами, размер обломков 0,2--2,0\,мм, которые составляют около 20\,\% породы. В некоторых шлифах  встречаются рогульчатые обломки вулканического стекла. Аутигенные минералы~--- магнетит, гетит, пирит. Вторичные минералы представлены хлоритом, карбонатом, серицитом.
\clearpage
\textbf{Зелёные туффиты} слагают нижнюю часть гижигинской свиты. Туффит зелёного цвета, известковистый, интенсивно хлоритизированный, песчанистый, щебенчатый. При петрографическом изучении туффитов, мы видим, что преобладает несортированный, глинисто-серицитовый материал с~тонкодисперсной примесью полевых шпатов и кварца и множеством пепловых рогулек, слагающих до 70\,\% породы, с размерностью фракции 0,025--0,01\,мм. Частицы обломочной фракции имеют удлинённую и изометричную морфологию, среднюю сортировку, угловатую степень окатанности. Аллотигенная часть представлена клисталлокластами полевого шпата и кварца, с размерами зёрен от 0,005 до 0,4\,мм, слагающими 15\,\% породы и литокластами эффузивных пород среднего состава размерностью 0,1--0,7\,мм со стекловатой и гиалопилитовой текстурой, слагающими 10\,\% породы. В~породе также отмечаются единичные <<окатыши>>, формирование, которых связывают c перемещением по дну комочков нелитифицированного осадка [2]. Аутигенные минералы~--- барит, гетит [2]. Вторичные минералы: серицит, хлорит, кальцит, мусковит, стиломелан.

\textbf{Серые туффиты} встречаются в фолькской свите. Туффиты серо-зеленого цвета, алеврито-песчаные, массивные, грубо-неяснослоистые, карбонатные. Основная масса породы – кремнисто–хлоритового состава, с примесью сфена (лейкоксена) и пепловыми обломками кислого состава, слагающая 50--60\,\% породы. Аллотигенные компоненты представлены двумя группами: кристаллокластами~--- кварца, альбита, плагиоклаза, полевого шпата, размер зёрен~--- от 0,05 до 0,3\,мм, слагающими 30--40\,\% породы; литокластами~--- обломками эффузивных пород среднего и кислого состава с~пилотакситовой, гиалопиловой и фельзитовой текстурами, размер зёрен от 0,25 до 0,7 мм, составляющими до 10\,\% породы. Форма зёрен остроугольная, округлая, призматическая; сортировка зёрен средняя, степень окатанность~--- преимущественно угловатая. Характерно присутствие сферул (порядка 3\,\% породы) заполненных кварцем, хлоритом, карбонатом, цеолитом\,(?), размером около 0,1\,мм. Аутигенные минералы~--- магнетит, пирит. Вторичные минералы представлены кальцитом, серицитом, гидроокислами железа.

\textbf{Зелёные туфоалевролиты} встречаются в хивачской свите в разрезе по руч.~Водопадный. Породы зеленовато-серого цвета, хлоритизированные, песчаные, щебенчатые, слабоизвестковистые, с зёрнами хлорита. Основная масса породы представлена гидрослюдистыми минералами с примесью дисперсных частиц полевых шпатов, слагающими до 60\,\% породы. Аллотигенные компоненты представлены двумя группами: первая~--- зерна полевых шпатов, кварца, плагиоклаза, с содержанием в породе от 10 до 30\,\%. Вторая группа~--- единичные обломки эффузивных пород основного состава с пилотакситовой и фельзитовой текстурами. Размер зёрен~--- от 0,05 до~0,7\,мм, угловатой, удлинённой, призматической формы, хорошей сортировки. В породе встречаются <<окатыши>> которые оформились при перемещении по дну комочков нелитифицрованного осадка. Аутигенными минералами являются пирит, гетит, гематит, хлорит. Зерна хлорита чаще всего имеют округлую форму, размер 0,1--0,3\,мм, процентное содержание зёрен увеличивается вверх по разрезу от 1--2 до 30\,\%. В некоторых зёрнах мы наблюдаем органику непонятного происхождения имеющего форму сеточки, паутины. Вторичные минералы: кальцит, хлорит.

\textbf{Серые туфоалевролиты} слагают джигдалинскую свиту. Породы серого цвета, известковистые, щебенчатые,  в туфоалевролитах мы наблюдали кремнистые стяжения, следы биотурбации осадка. Иногда наблюдается интенсивно хлоритизированный и кальцитизированный пирокластический материал основного состава [2]. Основная масса глинисто-кремнистого состава, слагает 40--50\,\% породы. Аллотигенные компоненты можно разделить на две группы: первая представлена  зёрнами кварца, полевого шпата, опала, размер зёрен от 0,05 до 0,3 мм, слагает 20--30\,\% породы; вторая группа~--- сильно хлоритизироваными эффузивными обломками основного, среднего состава, размер зерен от 0,3 до 0,7\,мм, слагают менее 1\,\% в породе, зерна остроугольной, призматической формы, средней сортировки, плохоокатанные. Пепловый материал~--- остроугольной, клиновидной формы, размером 0,4--0,5\,мм. Аутигенные минералы: магнетит, хлорит. Вторичные минералы: пирит, кальцит, хлорит, гидроокислы железа.

\textbf{Серые песчаники} слагают мунугуджакскую свиту. Песчаники зе\-ле\-но\-ва\-то-серые, вулканомиктовые, мелко-среднезернистые, слоистые, известковистые. Петрографическое изучение показало, что терригенная примесь представлена зёрнами кварца, полевых шпатов, и эффузивными обломками кисло-среднего состава, размером~--- от 0,05 до 0,4 мм, форма~--- остроугольная, таблитчатая, призматическая, плохоокатанные, слагают 75--80\,\% породы. Цемент кремнисто-хлоритовый, слагает до 15\,\% породы. Аутигенные минералы: гематит, магнетит, составляют до 3\,\% породы. Вторичные минералы: хлорит, кальцит.

Известковистые песчаники имеют микроорганогенно-детритовую структуру. Терригенная примесь представлена теми же зёрнами, которые слагают 15--20\,\% породы. Органические остатки представлены спикулами губок, составляющими 15--20\,\%. Все это цементируется бурым, почти непрозрачным аморфным веществом, главной составной частью которого являются фосфаты с примесью микрочешуйчатой глины, землистого карбоната и гидроокислов железа, составляющие 60--80\,\% породы. Аутигенные минералы~--- фосфат, магнетит. Вторичные минералы~--- гематит, хлорит, кальцит.

\textbf{Зелёные песчаники}, слагают фолькскую свиту. Песчаники массивные, среднеслоистые, зелёные и серые, средне- грубозернистые, полимиктовые, известковистые. Терригенная примесь представлена кварцем, полевым шпатом, альбитом, эффузивными породами средне-кислого состава, размер зёрен от 0,03 до 0,3\,мм, призматической и оскольчатой формы, составляющие порядка 40--50\,\%. Под микроскопом наблюдается смесь мелких карбонатных обломков призматического слоя раковин колымий, размером 0,05\,$\times$\,0,3\,мм, слагающих до 30\,\%. Цемент поровый, представлен микрочешуйчатым хлоритом, слагает до 3\,\% породы. Вторичные минералы~--- кальцит, гидроокислы железа.

Таким образом, по комплексу литологических типов описанных выше мы можем сказать, что пермское время характеризуется преимущественно мелководно-морским осадконакоплением, главным образом терригенным отложениями, а среднепермские отложения~--- карбонатными. Особенностью этих литотипов является туфовая примесь и довольно частая встречаемость остатков фауны.

\begin{thebibliography}{99}
%1
\bibitem{}\BibAuthor{Брынько~И.~В.} Основные литотипы пермских отложений юго-восточной части Омолонского массива // Идеи, гипотезы, поиск…: [сб.ст. по материалам XXIV регион. науч. конф. аспирантов, соискателей и молодых исследователей] / Сев.-Вост. гос. ун-т.~--- Красноярск~: Научно-инновационный центр, 2018.~--- С.~172--177.
\bibitem{}\BibAuthor{Кашик~Д.~С., Ганелин~В.~Г., Караваева~Н.~И., и др.} Опорный разрез перми Омолонского массива.~--- Л.~: Наука, 1990.~--- 200 с.
\bibitem{}\BibAuthor{Терехов~М.~И.} Стратиграфия и тектоника южной части Омолонского массива.~--- М.~: Наука, 1979.~--- 113~с.
\end{thebibliography}
