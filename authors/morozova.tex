
\procTitle{Литолого-петрографические особенности пермских отложений Омолонского массива}

\procAuthorI{Морозова~В.\,С.}
\procEmailI{veronika.morozova.2019@inbox.ru}
\procOrganizationI{СВГУ}
\procCityI{Магадан}

\procAuthorII{Брынько~И.\,В.}
\procEmailII{ibrynko@mail.ru}
\procOrganizationII{СВКНИИ ДВО РАН, СВГУ}
\procCityII{Магадан}


\makeProcTitleII
\index{b@Брынько~И.\,В.}
\index{m@Морозова~В.\,С.}

\vspace{-10pt}
Пермские отложения широко распространены в~Магаданской области. Они представлены различными фациями, состав которых обусловлено различием условий их образования. Отложения имеют большую мощность, нередко с~ними связаны проявления и~месторождения полезных ископаемых, в~частности золота. Наиболее полные и~хорошо охарактеризованные фауной разрезы находятся на Омолонском массиве [1, 2]. Поэтому данная работа посвящена литолого-петрографическому изучению пермских отложений Омолонского массива. Так, в~2014--2015~гг. нами были изучены три разреза пермских отложений~--- по~р.~Русская-Омолонская, руч.~Водопадный и~р.~Мунугуджак. При макро- и~микроскопического изучения пермских отложений нами выделено 8~основных литотипов, участвующих в~строении этих разрезов: колымиевые известняки, диамиктиты, серые туффиты, зелёные туффиты, серые туфоалевролиты, зелёные туфоалевролиты, серые известковистые песчаники, зелёные песчаники.

\textbf{Колымиевые известняки} встречаются в~рулонской толще, джигдалинской, омолонской, фолькской и~хивачской свитах. Эти известняки имеют серый цвет, с~запахом сероводорода, плитчатые, массивные, крепкие (глинистые, кремнистые). В шлифах основная масса породы состоит из разрозненных призмочек кальцита размером 0,05\,$\times$\,0,3~мм, слагает от~70 до~95~\%. Терригенная примесь представлена кварцем, полевым шпатом, плагиоклазом, реже халцедоном, слагает от 1 до 15~\% породы, размер зёрен от~0,05 до~0,4~мм. Цемент известняков различный: хлорит-карбонатного, карбонатного, глинисто-хлорит-карбонатного состава, слагает от~5 до~15~\% в~породе.

\textbf{Диамиктиты} слагают среднюю часть гижигинской свиты, имеют тёмно-серый цвет, щебёнчатые, со скорлуповатой отдельностью, массивной текстуры, содержат включения. Рассеянный обломочный материал (включения) представлен вулканитами и в~меньшей степени~--- серыми алевролитами и~гранитоидами~(?). Включения имеют размер от~0,5 до~30~см в~длину, содержание в~породе может варьировать, но составляет не менее 10--15~\%. В шлифах диамиктит более чем на 50~\% сложен плохо сортированным серицит-кварц-полевошпат-глинистым материалом размером 0,005--0,025~мм, который составляет основную массу породы. Аллотигенные компоненты представлены кристаллокластами полевого шпата, кварца, редко плагиоклаза с~размером обломков от~0,05 до~0,7~мм, слагающими до~20~\% породы. В~породе присутствуют литокласты, представленные эффузивами среднего-основного состава, размер обломков 0,2--2,0~мм, которые составляют около 20~\% породы. В некоторых шлифах встречаются рогульчатые обломки вулканического стекла.

\textbf{Зелёные туффиты} слагают нижнюю часть гижигинской свиты. Туффит зелёного цвета, интенсивно хлоритизированный, песчанистый, щебёнчатый. Основная масса представлена несортированным, глинисто-серицитовым материалом с~тонкодисперсной примесью полевых шпатов и~кварца и~множеством пепловых рогулек, слагающих до 70~\% породы, с~размерностью фракции 0,025--0,01~мм. Аллотигенная часть представлена клисталлокластами полевого шпата и~кварца, с~размерами зерен от~0,005 до~0,4~мм, слагающими 15~\% породы и~литокластами эффузивных пород среднего состава размерностью 0,1--0,7~мм, слагающими 10~\% породы. \enlargethispage{2\baselineskip}

\textbf{Серые туффиты} встречаются в~фолькской свите. Туффиты серо-зелёного цвета, алеврито-песчаные, массивные, грубо-неяснослоистые. Основная масса породы~--- крем\-ни\-сто-хло\-ри\-то\-во\-го состава, с~примесью сфена (лейкоксена) и~пепловыми обломками кислого состава, слагающая 50--60~\% породы. Аллотигенные компоненты представлены двумя группами: кристаллокластами~--- кварца, альбита, плагиоклаза, полевого шпата, размер зёрен от~0,05 до~0,3~мм, слагающими 30--40~\% породы; литокластами~--- обломками эффузивных пород среднего и~кислого состава, размер зёрен от~0,25 до~0,7~мм, составляющими до 10~\% породы. Характерно присутствие сферул (около 3~\% породы), заполненных кварцем, хлоритом, карбонатом, цеолитом~(?), размером около 0,1~мм.


\textbf{Зелёные туфоалевролиты} встречаются в~хивачской свите в~разрезе по руч.~Водопадный. Породы зеленовато-серого цвета, хлоритизированные, песчаные, щебёнчатые. Основная масса породы представлена гидрослюдистыми минералами с~примесью дисперсных частиц полевых шпатов, слагающими до~60~\% породы. Аллотигенные компоненты представлены двумя группами: первая~--- зёрна полевых шпатов, кварца, плагиоклаза, с~содержанием в~породе от~10 до~30~\%. Вторая группа~--- единичные обломки эффузивных пород основного состава. Размер зёрен от~0,05 до~0,7~мм. В~породе встречаются <<окатыши>>, которые оформились при перемещении по дну комочков нелитифицированного осадка. Аутигенные зерна хлорита чаще всего имеют округлую форму, размер 0,1--0,3~мм, процентное содержание зёрен увеличивается вверх по разрезу от 1--2 до 30~\%. В некоторых зёрнах, которые имеют форму сеточки, паутины, мы наблюдаем органику непонятного происхождения.

\textbf{Серые туфоалевролиты} слагают джигдалинскую свиту. Породы серого цвета, известковистые, щебёнчатые, содержат кремнистые стяжения и~следы биотурбации осадка. Иногда наблюдается интенсивно хлоритизированный и~кальцитизированный пирокластический материал основного состава [1]. Основная масса глинисто-кремнистого состава слагает 40--50~\% породы. Аллотигенные компоненты можно разделить на две группы: первая представлена зёрнами кварца, полевого шпата, опала, размер зёрен от~0,05 до~0,3~мм, слагает 20--30~\% породы; вторая группа~--- сильно хлоритизированные эффузивные обломки основного, среднего состава, размер зёрен от~0,3 до~0,7~мм, слагают менее 1~\% в~породе. Пепловый материал~--- остроугольной, клиновидной формы, размером 0,4--0,5~мм.

\textbf{Серые известковистые песчаники} слагают мунугуджакскую свиту. Песчаники зе\-ле\-но\-ва\-то-серые, вулканомиктовые, мелко-, среднезернистые, слоистые. Они имеют микро\-ор\-га\-но\-ген\-но-детритовую структуру. Терригенная примесь представлена зёрнами кварца, полевых шпатов, и~эффузивными обломками кисло-среднего состава, размером от~0,05 до~0,4~мм, слагают 15--75~\% породы. Органические остатки представлены спикулами губок, составляющими 1--15~\%. Всё это цементируется бурым, почти непрозрачным аморфным веществом, главной составной частью которого являются фосфаты с~примесью микрочешуйчатой глины, землистого карбоната и~гидроксидов железа, составляющие 15--60~\% породы.

\textbf{Зелёные песчаники} слагают фолькскую свиту. Песчаники массивные, среднеслоистые, зелёные и~серые, средне-, грубозернистые, полимиктовые, известковистые. Терригенная примесь представлена кварцем, полевым шпатом, альбитом, эффузивными породами средне-кислого состава, размер зёрен от~0,03 до~0,3~мм, призматической и~оскольчатой формы, составляет около 40--50~\%. Под микроскопом наблюдается смесь мелких карбонатных обломков призматического слоя раковин колымий, размером 0,05 $\times$ 0,3~мм, слагающих до 30~\%. Цемент поровый, представлен микрочешуйчатым хлоритом, слагает до 3~\% породы.

Изучение литологических особенностей из пермских отложений Омолонского бассейна уточняет реконструкцию обстановок седиментации. Особенностью этих литотипов является туфовая примесь и~довольно частая встречаемость остатков фауны.
\vspace{-8pt}
\begin{thebibliography}{99}
\vspace{-6pt}
\bibitem{}\BibAuthor{Кашик~Д.~С., Ганелин~В.~Г., Караваева~Н.~И. и~др.} Опорный разрез перми Омолонского массива.~--- Л.~: Наука, 1990.~--- 200 с.\enlargethispage{2\baselineskip}
\bibitem{}\BibAuthor{Терехов~М.~И.} Стратиграфия и~тектоника южной части Омолонского массива.~--- М.~: Наука, 1979.~--- 113~с.
\end{thebibliography}
