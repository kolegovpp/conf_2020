\procTitle{Информационно-поисковая система <<Стратиграфия и фауна верхнего палеозоя~--- нижнего триаса Северо-Востока России>>}
\procAuthor{Шиберт~Е.\,Ю.}
\procEmail{elizavetashibert@gmail.com}
\procOrganization{СВКНИИ ДВО РАН} \procCity{Магадан}

\makeProcTitle
\index{x@Шиберт~Е.\,Ю.}

В последнее десятилетие во всём мире широко разрабатываются и создаются информационные базы данных и поисковые системы различного рода, обеспечивающие учёным в любом уголке мира доступ к геологической информации разного содержания. В этом отношении не является исключением и информация палеонтолого-стратиграфических исследований. Примером таких ресурсов могут служить <<Fossilworks>>, <<Earthtime>>, <<PaleobiologyDa\-ta\-base>> и пр. [13, 14, 17].

В настоящее время в лаборатории региональной геологии и геофизики собраны многочисленные материалы по позднему палеозою и раннему триасу не только Северо-Востока России [1, 3--8, 10--12], но и многих прилегающих территорий: Восточного Забайкалья, Приморья, Монголии, обширных территорий Российской Арктики~--- Севера Сибири, Таймыра, Новой Земли, Севера Русской платформы [2, 9]. Большому массиву собранных данных необходима систематизация, которая будет способствовать возможности структурирования и обеспечения свободного доступа большого числа учёных, специалистов и любых заинтересованных лиц всего мира.

Первой частью рассматриваемой работы может являться создание информационно-поисковой системы <<Стратиграфия и фауна верхнего палеозоя~--- нижнего триаса Северо-Востока России>>. Эта поисковая система должна обеспечивать группировку, хранение и~быстрый поиск доступной палеонтологической и стратиграфической информации, необходимой для выполнения различных междисциплинарных проектов по~геологии и смежным направлениям исследований. Свободный доступ через интернет сделает возможным быстрый доступ к необходимым данным для большого числа пользователей. Информационно-поисковая система упростит получение данных также для тех, кому необходимо затребовать информацию из другого города или страны, без ожидания почтовых отправлений.

Часто многие материалы разобщены или недоступны для многих потенциальных пользователей, многие не опубликованы или не существуют на цифровых носителях, а имеются только в печатном или рукописном варианте. Применение доступных форматов даст возможность объединения информации в общие, более обширные, информационно-поисковые системы геологического содержания, в частности, в имеющейся в СВКНИИ ДВО РАН ГИС~<<Природные ресурсы Магаданской области>> [15, 16].
Структура такой системы будет содержать несколько блоков, каждый из которых будет нести свою необходимую информацию см. рисунок.

В первом блоке будет представлена непосредственно картографическая информация, в которой содержатся географически привязанные точки наблюдений, номера опорных и~стратотипических разрезов, схемы структурно-фациального районирования и т.\,д. Во второй блок будут внесены базы данных по характеристике стратонов, точкам наблюдений, спискам определений палеонтологических объектов и пр. В~третий блок предполагается включить совокупность электронных документов с различными необходимыми научными публикациями, связанных между собой специальными ссылками (гиперссылками) для~быстрого перехода от одного документа в заданное место другого и перемещений внутри них~--- гипертекстовые документы. А~четвёртый блок будет представлять собой информацию в виде цифровых снимков обнажений, окаменелостей и др. для наглядного пояснения и расшифровки текстовой информации.\enlargethispage{\baselineskip}

\begin{figure}[H]
  \centering
  \includegraphics[width=0.9\textwidth, page=1]{authors/shibert_IPS_struktura.pdf}
  \caption*{\textbf{Структура информационно-поисковой системы}}
  \label{fig:shibert_IPS_struktura.pdf}
\end{figure}



Источниками размещаемой информации будут опубликованные статьи, монографии, научные отчёты, различные коллекции ископаемой фауны и другие материалы. На начальном этапе предлагаемого проекта планируется внесение данных по более чем 500 опорным разрезам верхнего палеозоя -- нижнего триаса, 500-600 точкам наблюдений, включающим несколько тысяч образцов ископаемой фауны.

Геолого-картографической основой, разрабатываемой информационно-поисковой системой, будут разномасштабные ГИС-проекты всей территории Магаданской области.

В данный момент нами систематизирована бóльшая часть палеонтологических и стратиграфических материалов, а также сфотографированы несколько сотен палеонтологических объектов, начата их цифровая обработка.

\textit{Исследование выполнено при финансовой поддержке РФФИ в рамках научного проекта №~20-05-00604.}


\begin{thebibliography}{99}
%1
\bibitem{}\BibAuthor{Бяков А.~С.} Пермские отложения Балыгычанского поднятия.~--- Магадан~: СВКНИИ ДВО РАН, 2004.~--- 87 с.

\bibitem{}\BibAuthor{Бяков А.~С.} Новые представления о системе пермских иноцерамоподобных двустворок востока бореальной зоны // Палеонтологический журнал.~--- 2008.~---  №~3.~--- С.~12--23.
\bibitem{}\BibAuthor{Бяков А.~С.} Зональная стратиграфия, событийная корреляция, палеобиогеография перми Северо-Востока Азии (по двустворчатым моллюскам).~-- Магадан~: СВКНИИ ДВО РАН, 2010.~--- 262~с.
\bibitem{}\BibAuthor{Бяков А.~С.} Новая зональная схема пермских отложений Северо-Востока Азии по двустворчатым моллюскам. Статья~1. Зональное расчленение // Тихоокеан. геол.~---  2012.~--- Т.~31, №~5.~--- С.~13--40.
\bibitem{}\BibAuthor{Бяков А.~С.} Новая зональная схема пермских отложений Северо-Востока Азии по двустворчатым моллюскам. Статья~2. Вопросы корреляции // Там же.~---  2013.~---  Т.~32, №~1.~--- С.~3--17.
\bibitem{}\BibAuthor{Бяков А.~С.} Пермские биосферные события на Северо-Востоке Азии // Стратиграфия. Геол. корреляция.~---  2012.~---  №~2.~---  С.~88--100.
\bibitem{}\BibAuthor{Кашик Д.С. и др.} Опорный разрез перми Омолонского массива / Д.~С.~Кашик, В.~Г.~Ганелин, Н.~И.~Караваева, А.~С.~Бяков, О.~А.~Миклухо-Маклай, Г.~А.~Стукалина, Н.~В.~Ложкина, Л.~А.~Дорофеева, Ю.~К.~Бурков, Е.~И.~Гутенева, Л.~Н.~Смирнова~--- Л.~: Наука, 1990.~---  200~с.
\bibitem{}\BibAuthor{Кутыгин Р.В. и др.} Опорный разрез дулгалахского и хальпирского горизонтов Западного Верхоянья / Р.~В.~Кутыгин, И.~В.~Будников, А.~С.~Бяков, А.~Г.~Клец, В.~С.~Гриненко // Тихоокеан. геология.~---  2003.~---  Т.~22, №~6.~---  С.~82--97.
\bibitem{}\BibAuthor{Захаров Ю.~Д.,  Бяков А.~С., Хорачек М.} Глобальная корреляция базальных слоев триаса в~свете первых изотопно-углеродных свидетельств по границе перми и триаса на Северо-Востоке Азии // Там же.­--  2014.~---  №~1.~---  C.~3--19.
\bibitem{}\BibAuthor{Biakov A.~S.} Permian bivalve mollusks of Northeast Asia // Journ. of Asian Earth Sciences.~--- 2006.~--- Vol.~26, No.~3--4.~--- P.~235--242.
\bibitem{}\BibAuthor{Biakov A.~S., Shi G.~R.} Palaeobiogeography and palaeogeographical implications of Permian marine bivalve faunas in Northeast Asia (Kolyma-Omolon and Verkhoyansk-Okhotsk regions, northeastern Russia) // Palaeogeography, Palaeoclimatology, Palaeoecology.~--- 2010.~--- Vol.~298.~--- Iss.~1--2.~--- P.~42--53.
\bibitem{}\BibAuthor{Davydov V.~I., Biakov A.~S., Schmitz~M.~D., Silantiev~V.~V.} Radioisotopic calibration of the~Guadalupian (middle Permian) Series: review and updates // Earth-Science Reviews.~--- 2018.~--- Vol.~176.~--- P.~222--240.
\bibitem{}Earthtime.~--- 2018.~--- URL: https://earthtime.org  (ref. date: 20.02.2020).
\bibitem{}Fossilworks.~--- 1998.~--- URL: http://fossilworks.org  (ref. date: 20.02.2020).
\bibitem{}\BibAuthor{Golubenko I.~S.} Quantitative Estimation of Gold Mineralization in Degdekan-Arga-Yuryakh District (Magadan Region, Russia)// Mineral Prospectivity, current approaches and future innovations~--- Orléans, France, 24-26 October 2017: Book of abstracts.~--- Orléans~: N.\,pl., 2017.~--- P.~82--83.
\bibitem{}\BibAuthor{Golubenko I.~S., Goryachev N.~A.} Bank of geospatial information of the geological structure of the territory Magadan region (Northeast of Russia) // Информационные технологии для наук о~Земле и приложения для геологии, горной промышленности и экономики <<ITES\&MP-2019>>~: Материалы V~Международной конференции.~--- М.~: ВНИИгеосистем, 2019.~--- С.~52.

\bibitem{}PaleobiologyDatabase.~--- 1998.~--- URL: https://paleobiodb.org  (ref. date: 20.02.2020).

\end{thebibliography}
